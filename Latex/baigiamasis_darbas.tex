\documentclass[12pt]{article}
\usepackage{indentfirst}
\usepackage[utf8x]{inputenc}
\usepackage[T1]{fontenc}
\usepackage[english,lithuanian]{babel}
\usepackage{array}
\usepackage{caption}
\usepackage{subcaption}
\usepackage{makecell}
\usepackage[euler]{textgreek}
\usepackage{multirow}
\usepackage{boldline}
\usepackage{floatrow}
\floatsetup[table]{capposition=top}
\usepackage{amsmath, amsthm, amssymb}
\usepackage{graphicx}
\usepackage{setspace}
\usepackage{verbatim}
\usepackage[left=3cm,top=2cm,right=1.5cm,bottom=2cm]{geometry}
\usepackage{floatrow}
\newfloatcommand{capbtabbox}{table}[][\FBwidth]
\usepackage{blindtext}
\onehalfspacing
\usepackage[hidelinks, unicode]{hyperref}
\usepackage{textcomp}
\usepackage{amsmath}

\newcommand{\EE}{\mathbb{E}\,} % Mean
\newcommand{\ee}{{\mathrm e}}  % nice exponent
\newcommand{\dd}{{\mathrm d}}
\newcommand{\RR}{\mathbb{R}}

\begin{document}
\selectlanguage{lithuanian}

\begin{titlepage}
\vskip 20pt
\begin{center}
\includegraphics[scale=0.5]{MIF}
\end{center}

%%%%%%%%%%%%%%%%%%%%%%%
% TITULINIS PUSLAPIS
%%%%%%%%%%%%%%%%%%%%%%%

\vskip 20pt
\centerline{\bf \large \textbf{VILNIAUS UNIVERSITETAS}}
\bigskip
\centerline{\large \textbf{MATEMATIKOS IR INFORMATIKOS FAKULTETAS}}
\bigskip
\centerline{\large \textbf{BIOINFORMATIKOS BAKALAURO STUDIJŲ PROGRAMA}}

\vskip 90pt
\begin{center}
    {\bf \LARGE Pavadinimas lietuviškai}
\end{center}
\begin{center}
    {\bf \Large Pavadinimas angliškai}
\end{center}
\vskip 20pt
\centerline{\bf \large \textbf{Bakalauro baigiamasis darbas}}
\bigskip
\vskip 40pt

\hskip 140pt {\large Autorė: Danielė Stasiūnaitė}

\hskip 140pt{\large VU el. p.: daniele.stasiunaite@mif.stud.vu.lt}
\bigskip
\vskip 20pt

\hskip 140pt {\large Darbo vadovė: J. m. d. Kotryna Kvederavičiūtė}
\vskip 60pt
\vskip 40pt
\centerline{\large \textbf{Vilnius}}
\centerline{\large \textbf{2023}}
\newpage
\end{titlepage}

\selectlanguage{lithuanian}

%%%%%%%%%%%%%%%%%%%%%
% TURINIO PUSLAPIS
%%%%%%%%%%%%%%%%%%%%%

\tableofcontents
\newpage

%%%%%%%%%%%%%%%%%%%%%%%%%%%%%%%%%%%%
% LIETUVIŠKOS SANTRAUKOS PUSLAPIS
%%%%%%%%%%%%%%%%%%%%%%%%%%%%%%%%%%%%

\section*{Santrauka}
\hfill \break
\textbf{Raktiniai žodžiai:} R, \emph{Shiny}, \emph{Homo sapiens},
\emph{Mus musculus}, ChIP sekoskaita

\newpage

%%%%%%%%%%%%%%%%%%%%%%%%%%%%%%%%%%
% ANGLIŠKOS SANTRAUKOS PUSLAPIS
%%%%%%%%%%%%%%%%%%%%%%%%%%%%%%%%%%

\section*{Summary}
\hfill \break
\textbf{Keywords:} R, \emph{Shiny}, \emph{Homo sapiens},
\emph{Mus musculus}, ChIP-seq

\newpage

%%%%%%%%%%%%%%%%%%%
% ĮVADO PUSLAPIS
%%%%%%%%%%%%%%%%%%%

\section{Įvadas}
\subsection*{Darbo temos aktualumas}
\subsection*{Darbo tikslas}
\subsection*{Uždaviniai}
\newpage

%%%%%%%%%%%%%%%%%%
% TEORINĖ DALIS
%%%%%%%%%%%%%%%%%%

\section{ChIP sekoskaita ir jos vykdymo ypatumai}

\textbf{DNR sekoskaita} - deoksiribonukleorūgšties nukleotidų sekos nustatymas.
Nukleotidų sekų nustatymui gali būti naudojami du pagrindiniai DNR sekvenavimo 
metodai:

\begin{itemize}
    \item \emph{Sendžerio} metodas;
    \item Didelio našumo arba naujos kartos sekoskaita
    (angl. \emph{NGS - \textbf{N}ext \textbf{G}eneration \textbf{S}equencing}).
\end{itemize}

Šiuolaikiniuose tyrimuose labai dažnai taikomos modernios DNR sekos - naujos 
kartos sekoskaitos - technologijos, kurios leidžia nuskaityti didelį kiekį
DNR arba RNR sekų daug sparčiau ir pigiau nei klasikinė \emph{Sendžerio} 
sekoskaita. Pastaroji sekoskaita dažniau naudojama, tiriant mažus duomenų
rinkinius.

\subsection{Transkripcijos faktoriai ir jų prasmė}

\textbf{Transkripcijos faktorius} - ypatingas baltymų tipas. Šio tipo baltymai
atpažįsta specifines DNR sekas (genų promotoriuose esančias sekas) ir tokiu būdu
kontroliuoja chromatino struktūros kondensacijos laipsnį bei atitinkamų genų
ekspresijos procesus, inicijuojant arba slopinant genų transkripciją. Šių
baltymų sąveika su DNR įtakoja daug biologiškai svarbių procesų: ląstelių
diferenciaciją, ląstelės ciklo eigą, genų transkripciją, DNR replikaciją,
imuninio atsako valdymą ir daugelį kitų procesų\cite{ARTICLE10, ARTICLE11}.

Nepaisant to, transkipcijos faktoriai bei sekos, prie kurių jie jungiasi, gali
mutuoti. Šios pastarųjų baltymų mutacijos nulemia įvairių ligų išsivystymą.
Pavyzdžiui, atsiradusios mutacijos AIRE (autoimuniniame reguliatoriuje)
transkripcijos faktoriuje sukelia I tipo autoimuninį poliendokrinopatijos
sindromą\cite{ARTICLE9} (angl. \emph{APS1 - \textbf{A}utoimmune
\textbf{P}olyendocrinopathy \textbf{S}yndrome type \textbf{I}}). Pasireiškus
šiam sindromui organizmo imuninės ląstelės naikina sveikas, hormonus
išskiriančių liaukų ląsteles\cite{ASP1}.

\subsection{ChIP sekoskaitos apibūdinimas}
\textbf{ChIP sekoskaita} - chromatino imunoprecipitacijos sekoskaita (angl.
\emph{chromatin immunoprecipiration sequencing}). Tai yra viena iš svarbiausių
technologijų, atliekant epigenetikos tyrimus\cite{ARTICLE1}. Šiam metodui
būdingas klasikinio eksperimentinio chromatino precipitacijos metodo derinimas
su naujos kartos (angl. \emph{next generation (NGS) sequencing}) sekoskaita,
siekiant išsiaiškinti baltymų sąveikas su DNR ir nustatyti, kaip transkripcijos
faktoriai ir kiti, su chromatinu susiję baltymai, gali įtakoti įvairius fenotipo
pokyčių mechanizmus\cite{ARTICLE2}.

\newpage

Pagrindinės ChIP sekoskaitos taikymo sritys\cite{ARTICLE2}:
\begin{itemize}
    \item Histonų modifikavimas;
    \item Genų reguliacijos tyrimai;
    \item Transkripcijos komplekso surinkimas;
    \item DNR pažaidų taisymas;
    \item Vystymosi mechanizmų tyrimai;
    \item Ligų progresavimo tyrimai.
\end{itemize}

% KABUTĖS: „ “

\subsection{ChIP sekoskaitos vykdymo eiga}

DNR ir baltymų sąveikos tyrimo - ChIP sekoskaitos - eiga pateikta pirmame
paveiksle:

\begin{figure}[ht]
    \begin{center}
        \includegraphics[width=0.7\linewidth]{../Figures/ChIP-seq_workflow.jpg}
        \vspace{-1\baselineskip}
        \caption*{\small\textbf{1 pav. ChIP sekoskaitos vykdymo etapai}}
        \label{fig:birds}
    \end{center}
\end{figure}

Remiantis pateiktu paveikslu ChIP sekoskaita gali būti suskirstyta į du
pagrindinius etapus:

\begin{enumerate}
    \item Mėginių paruošimas ir sekvenavimas;
    \item Kompiuterinė analizė.
\end{enumerate}

\newpage

\subsubsection{Mėginių paruošimas ir sekvenavimas}
\begin{itemize}
    \item \textbf{Baltymo prijungimas prie DNR.} Priklausomai nuo to, kokio
        baltymo sąveika su DNR yra tiriama, baltymo ir DNR sąveika gali būti
        sustiprinta, naudojant įvairias chemines medžiagas (formaldehidas
        naudojamas dažniausiai).
    \item \textbf{DNR suskaidymas į fragmentus.} \emph{NGS} bibliotekos 
        paruošimui reikalingas DNR suskaidymo į fragmentus etapas. Šiame etape
        DNR dažniausiai suskaidoma į 150 - 500 nukleotidus turinčius fragmentus,
        naudojant ultragarso bangas - sonifikacijos mechanizmą. Pastarojo
        mechanizmo sukurtas vibracijos rezonansas suskaido DNR į fragmentus.
        Fragmentų ilgis gali būti kontroliuojamas sonifikacijos įrenginio
        naudojimo dažniu. Pavyzdžiui, kuo ilgiau trunka vienas ultra garso
        panaudojimo ciklas, tuo trumpesni DNR fragmentai
        gaunami\cite{SONICATION}.
    \item \textbf{Imunoprecipitacijos procesas.} Suskaidytos DNR fragmentai
        inkubuojami su specifiniu antikūnu, galinčiu atpažinti prie DNR
        prisijungusį baltymą - transkripcijos faktorių. Tam, jog ChIP
        sekoskaitos rezultatai būtų patikimi ir tinkami, tinkamas antikūno
        parinkimas ir jo kokybės užtikrinimas yra vienas iš svarbiausių ChIP
        sekoskaitos mėginių paruošimo etapų\cite{ARTICLE3}. Testuojant
        skirtingus antikūnus pasirenkamas tas, kurį panaudojus gaunamas
        didesnis DNR sekų, prie kurių prisijungęs transkripcijos faktorius,
        praturtinimas nei praturtinimas, kuris gautas, naudojant nespecifinį
        antikūną\cite{ARTICLE4} (pavyzdžiui, naudojant tipinį imunoglobulino G
        (IgG) antikūną).
    \item \textbf{Sekvenavimas.} Neretai sekvenavimo įrenginių pritaikymui
        reikalingi trumpų adapterių prijungimo prie gautų DNR fragmentų ir PGR
        amplifikacijos etapai - reikalingas bibliotekos sukonstravimas, kuris
        gali skirtis, priklausomai nuo pasirinktos sekvenavimo platformos
        ir joms specifinių bibliotekos paruošimo protokolo\cite{ARTICLE5}.
        Įvykdžius šiuos etapus gali būti gautas „triukšmas“\cite{ARTICLE4}
        (angl. \emph{bias}), kuris gali būti mažesnis, atliekant mažiau DNR
        amplifikacijos (padauginimo) ciklų. Sukonstravus biblioteką atliekamas
        \emph{NGS} sekvenavimas.
        
\end{itemize}

\subsubsection{Kompiuterinė analizė}
\begin{itemize}
    \item \textbf{DNR nuskaitymų kartografavimas.} Nusekvenuoti DNR 
        fragmentai išsaugomi \emph{FASTQ} arba \emph{CSFSATQ} formatais. Šie DNR
        nuskaitymai (angl. \emph{reads} arba \emph{tags}) perkeliami ant genomo,
        naudojant kartografavimo (angl. \emph{mapping}) įrankius, pavyzdžiui,
        \emph{Bowtie}\cite{ARTICLE6}, \emph{Burrows-Wheeler}, kurie leidžia
        nustatyti nuskaityto DNR fragmentp poziciją genome, esant kelių
        nukleotidų neatitikimui\cite{ARTICLE7} (angl. \emph{mismatch}). Atlikus
        DNR nuskaitymų priskyrimą gaunami \emph{SAM}, \emph{BAM} (dažniausiai
        naudojamas formatas), \emph{CRAM} arba \emph{tagAlign} formato failai.
    \item \textbf{Normalizavimas.} 
    \item \textbf{Pikų nustatymas (angl. \emph{peak calling}).} Šiame etape
        nustatomi reikšmingai praturtinti genomo lokusai - pikai
        (angl. \emph{peaks}). Įgyvendinus šį etapą dažniausiai sugeneruojami
        \emph{BED} formato failai\cite{ARTICLE8}, kuriuose pateikiamos genominės
        pikų pozicijos, įvairūs statistiniai įverčiai bei identifikacijos kodai,
        kuriuos naudojant galima vykdyti įvairias analizes.
    \item \textbf{Analizių atlikimas.} Dažniausiai atliekamos analizės yra
        motyvų analizė bei genų ontologijos analizė\cite{ARTICLE8}
        (angl. \emph{GO - \textbf{G}ene \textbf{O}ntology enrichment analysis}),
        pateikianti biologinių procesų, ląstelinių komponentų ir molekulinių
        funkcijų, kuriose dalyvauja genas, sąrašą.
\end{itemize}

% Paaiškinti, kas yra ChIP sekoskaita ir koks jos tikslas, kokie jos privalumai.
% Kokie egzistuoja metodai, leidžiantys įvertinti TF taikinius kitame organizme?
% Kas yra transkripcijos faktorius?
% Ką daro TF?
% Kuo yra svarbūs TF?
% Kas yra ChIP sekoskaita?
% Kokiais metodais gali būti nustatyti pikai?
% Kokie yra šių metodų privalumai ir trūkumai?
% Kodėl naudojamas MACS2?

\newpage

\subsection{Pikų nustatymo algoritmai}
Pikų nustatymas yra vienas iš svarbiausių etapų, atliekant DNR ir reguliatorinių
baltymų - transkripcijos faktorių arba histonų - sąveikos tyrimų analizes.
Kuriant pikų nustatymo algoritmus sprendžiamos dvi pagrindinės problemos:
regionų, kurie, tikėtina, yra pikai, nustatymas bei tikėtinų pikų statistinio
reikšmingumo tikrinimas.

Pagrindinė pikų nustatymo algoritmų įvestis yra kartografavimo metu su genomu
išlyginti DNR fragmentų nuskaitymai. Antrajame paveiksle šie fragmentai pažymėti
raudona ir žalia spalvomis.

\begin{figure}[ht]
    \begin{center}
        \includegraphics[width=0.6\linewidth]{../Figures/Read_mapping.png}
        \vspace{-1\baselineskip}
        \caption*{\small\textbf{2 pav. DNR nuskaitymų kartografavimas}}
        \label{fig:birds}
    \end{center}
\end{figure}

Nustačius DNR fragmentų nuskaitymų pozicijas genome kai kurios pastebimos
nuskaitymų sankaupų grupės gali indikuoti, jog toje pozicijoje yra galimas
transkripcijos faktoriaus prisijungimas (nuskaitymų sankaupa yra reikšminga),
tačiau neretai tokios sankaupos - pikai - gali būti laikomos molekuliniu arba
eksperimentiniu „triukšmu“. Taigi kuriant pikų nustatymo algoritmus yra
svarbu, jog algoritmas gebėtų įvertinti, ar pikas yra biologiškai reikšmingas,
ar tai tėra „triukšmas“.

Yra sukurta daugiau nei 30 skirtingų pikų nustatymo algoritmų (angl.
\emph{peak caller}), kurie sprendžia anksčiau paminėtas problemas,
tačiau šių problemų sprendimo būdai yra skirtingi. Konkretaus algoritmo
pasirinkimas labai priklauso nuo atliekamo eksperimento tipo ir specialisto,
atliekančio analizę, patirties\cite{ARTICLE13}.

\subsubsection{MACS2}
\textbf{MACS2} - \textbf{M}odel-based \textbf{A}nalysis of
\textbf{C}hIP-\textbf{S}eq. Tai yra populiariausias ir bene seniausias
pikų nustatymo algoritmas.
Transkripcijos faktorių jungimosi prie DNR saitai nustatomi, atsižvelgus į
nuskaitymų pozicijas bei kryptį. Neretai konkrečioje genomo pozicijoje gali
būti priskirti keli nuskaitymai. MACS2 metode yra numatyta palikti tik po
vieną nuskaitymą konkrečioje pozicijoje (pašalinti duplikatus). Duplikatai
nėra šalinami, jeigu tikimasi, kad transkripcijos faktorius jungsis keliose
skirtingose genomo pozicijose. Tose genomo pozicijose, kurioje, tikėtina,
jungiasi transkripcijos faktorius turi būti pastebimas \emph{Watsono} ir
\emph{Kriko} nuskaitymų išsidėstymas arba \textbf{bimodalinis pasiskirtymas},
kurio grafikas pavaizduotas paveiksle:

\begin{figure}[ht]
    \begin{center}
        \includegraphics[width=0.4\linewidth]{../Figures/Bimodal_pattern.png}
        \vspace{-1\baselineskip}
        \caption*{\small\textbf{3 pav. Bimodalinis pasiskirstymas}}
        \label{fig:birds}
    \end{center}
\end{figure}

Tam, jog panaši pikų struktūra būtų surasta, MACS algoritmas skanuoja visą
kartografuotų nuskaitymų duomenų rinkinį. Algoritmas naudoja dydį, kuris nurodo,
į kokio ilgio nukleotidų fragmentus buvo skaidoma DNR sonifikacijos proceso
metu (angl. \emph{bandwidth}), bei \emph{mfold} vertę. Vykdant algoritmą
atliekamas \emph{2 * bandwidth} ilgio poslinkis ir ieškoma tokių genomo
pozicijų, kuriose nuskaitymų yra daugiau nei naudojant atsitiktinį nuskaitymų
rinkinį (daugiau už \emph{mfold} vertę).

Nustačius aukštos kokybės pikus yra atsitiktinai parenkama 1000 pikų. Turint
šiuos pikus yra atskiriami jų \emph{Watsono ir Kriko} (teigiamos ir neigiamos
grandinės) nuskaitymai.
Šių teigiamų ir neigiamų grandinių pikų grupės yra išlyginamos pagal jų
centrus, kaip pavaizduota 4 paveiksle. Atstumas tarp išlygintų nuskaitymų
modų (\emph{d}) nurodo, kokio ilgio yra piko fragmentas.

\begin{figure}[ht]
    \begin{center}
        \includegraphics[width=0.4\linewidth]{../Figures/Tag_alignment.png}
        \vspace{-1\baselineskip}
        \caption*{\small\textbf{4 pav. Nuskaitymų išlyginimas}}
        \label{fig:birds}
    \end{center}
\end{figure}

Algoritmas visiems nuskaitymams atlieka \emph{d/2} 3' DNR galo link link
tikėtiniausio DNR ir transkripcijos faktoriaus sąveikos regiono. Atlikus šį
poslinkį atliekamas \emph{2 * d} poslinkis, jog būtų surastas statistiškai
reikšmingas nuskaitymų praturtinimas, naudojant \emph{Puasono} skirstinį, kurio
parametras \(\lambda_{BG}\) yra tikėtinas nuskaitymų skaičius atlikus poslinkį.
Nepaisant to, \(\lambda_{BG}\) parametras naudojamas,
neatsižvelgus į galimą „triukšmą“, kuris galėjo kilti dėl chromatino struktūros,
DNR aplifikacijos arba sekvenavimo, todėl yra naudojamas parametras
\(\lambda_{local}\), kuris skaičiuojamas kiekvienam tikėtinam
pikui:

\begin{equation} \label{lambda_local}
    \lambda_{local} = max(\lambda_{BG}, \lambda_{5k}, \lambda_{10k})
\end{equation}

čia \(5k\), \(10k\) yra poslinkio plotis.

Parametro \(\lambda_{local}\) naudojimas leidžia aptikti \emph{false positive},
pikus (pikus, kurie atsirado dėl „triukšmo“) ir nustatyti tik tuos pikus,
kurie indikuoja svarbų DNR ir baltymo sąveikos regioną\cite{ARTICLE12}.

\textbf{Metodo privalumai} \newline
  
\subsubsection{GEM}
\textbf{GEM} - \textbf{G}enome wide \textbf{E}vent finding and \textbf{M}otif
discovery. Tai yra vienas iš naujesnių (sukurtas 2012 metais) algoritmų, kuris
išsiskiria tuo, jog jame yra kombinuojama pikų paieška bei motyvų analizė, jog
būtų pagerinta finalinių pikų rezoliucija\footnote{\textbf{Rezoliucija - }
genetikoje aukšta rezoliucija reiškia, jog yra žinoma itin daug molekulinių
detalių apie DNR.}.

\emph{GEM} algoritmas sudarytas iš šešių skirtingų etapų\cite{ARTICLE15}:
\begin{enumerate}
    \item \textbf{Baltymo ir DNR sąveikos regionų nustatymas.} Pradiniai
          regionai nustatomi, taikant \emph{GPS} algoritmą\cite{ARTICLE14},
          kuris naudoja \emph{Dirichlė} skirstinį.
    \item \textbf{Praturtintų \emph{k - merų} nustatymas.} Jie nustatomi,
          lyginant \emph{k - merų} dažnius tarp teigiamų sekų ir neigiamų
          kontrolinių sekų. Teigiamos sekos - sekos, kurios sudarytos iš 61
          bazių poros ir yra išsidėsčiusios spėjamų baltymo ir DNR sąveikos
          regionų (gautų pirmajame etape) centruose. Neigiamos kontrolinės
          sekos - 61 bazių porą turinčios sekos, kurios yra nutolusios nuo
          teigiamų sekų per 300 bazių porų. Be to, šios sekos nepersidengia su
          sekomis, esančiomis baltymo - DNR sąveikos sekų centruose. Šiame
          etape yra skaičiuojami \emph{k - merų} fragmentų atitikimai teigiamų
          ir neigiamų sekų rinkiniuose. \emph{K - meras} (sekos fragmentas)
          yra laikomas praturtintu, kai \emph{p} vertė yra mažesnė nei 0.001.
          % ir 3-fold enrichment in terms of positive/negative hit count?
    \item \textbf{Praturtintų \emph{k - merų} klasterizavimas.} Praturtinti
          \emph{k - merai} klasterizuojami į ekvivalentiškumo klases, kurios
          apibūdina panašias DNR sekas, prie kurių jungiasi transkripcijos
          faktorius. Seka atitinka \emph{k - mero} ekvivalentiškumo klasę, kai
          sekoje nustatomas fragmentas yra vienas iš ekvivalentiškumo klasės
          elementų.
    \item \textbf{Išankstinio pasiskirstymo nustatymas.} Labiausiai praturtinta
          \emph{k - merų} klasė yra naudojama \emph{Dirichlė} išankstinio
          pasiskirstymo paskaičiavimui. Šiame etape genomas yra suskaidomas į
          kelis tūkstančius bazių porų turinčius segmentus. Šie segmentai yra
          gaunami atskiriant DNR fragmentus, kuriuose yra daugiau nei 500 bazių
          porų turintys tarpai bei DNR regionai, kuriems buvo priskirta mažiau
          nei 6 DNR nuskaitymai (angl. \emph{reads}). Šie regionai yra
          skanuojami su DNR sekų fragmentais, kurie priklauso atrinktai
          \emph{k - merų} ekvivalentiškumo klasei. Šie \emph{k - merų}
          atitikimai yra skaičiuojami.
    \item \textbf{Tikslesnių baltymo - DNR sąveikos regionų spėjimas.} Tam yra
          panaudojamas 4 etape gautas išankstinis pozicijų pasiskirstymas.
    \item \textbf{2 - 3 etapų kartojimas.} Tam yra panaudojami 5 etape gauti
          patikslinti baltymo - DNR sąveikos regionai.
\end{enumerate}

% \subsubsection{CisGenome}
% Taikant šį pikų nustatymo įrankį DNR nuskaitymai yra kartografuojami. Baltymo -
% DNR sąveikos regionai yra identifikuojami kaip tos genomo sritys, kurioms
% būdingas itin didelis nuskaitymų skaičius. Priklausomai nuo to, ar kontroliniai
% mėginiai yra sekvenuojami, taikant šį įrankį pikai gali būti nustatomi dvejopai.

% Naudojant \emph{CisGenome} įrankį vykdomi šie etapai\cite{ARTICLE16}:
% \begin{enumerate}
%     \item \textbf{\emph{FDR} skaičiavimas.} Pasirinkto organizmo genomas yra
%           suskaidomas į nepersidegiančius pasirinkto dydžio fragmentus - langus
%           (angl. \emph{windows}). Atlikus genomo suskaidymą yra suskaičiuojama,
%           kiek kiekviename fragmente (lange) yra kartografuotų sekų nuskaitymų.
%           \emph{FDR} reikšmė gaunama spėjamą \(k\) nuskaitymų lange proporciją 
%           padalinus iš stebimos \(k\) nuskaitymų lange proporcijos, remiantis
%           neigiamu binominiu modeliu. Šiame etape nustatoma, kokiam
%           kartografuotų nuskaitymų skaičiui būdinga mažesnė nei 0.1 \emph{FDR}
%           reikšmė.
%     \item \textbf{Pikų nustatymas.} Jeigu du pikus skiria mažesnis bazių porų
%           skaičius nei maksimalus jų skaičius, šie du pikai yra sujungiami į
%           vieną piką. Taip pat pikai, kurie yra trumpesni nei nurodytas
%           minimalus piko ilgis, tada tokie pikai nėra įtraukiami į bendrą pikų
%           sąrašą.
% \end{enumerate}

\newpage

%%%%%%%%%%%%%%%%%%%%%%
% MĖGINIŲ APRAŠYMAS
%%%%%%%%%%%%%%%%%%%%%%

\section{Pasirinktų mėginių charakteristika}
Metodo patikimumui ir tikslumui įvertinti naudoti naminės pelės ir žmogaus
\emph{ChiIP} sekoskaitos duomenys, gauti iš GTRD (Gene Transcription Regulation
Database)\cite{GTRD} duomenų bazės, saugančios informaciją apie transkripcijos
sekas bei atviro chromatino regionus.

Metodo testavimui naudoti 7 skirtingi naminių pelių (lot. \emph{Mus musculus})
mėginiai bei išgauti iš 6 skirtingi žmogaus (lot. \emph{Homo sapiens}) mėginiai.
Abiejų organizmų mėginiai išgauti iš vienodo tipo \emph{Th1}, \emph{Th2} ir
širdies skilvelių ląstelių. 

ChIP sekoskaitos mėginiai, išgauti iš naminės pelės, aprašyti pirmoje
lentelėje.

\begin{table}[htb]
    \newcolumntype{M}[1]{>{\centering\arraybackslash}m{#1}}
    \small
    \caption*{\small\textbf{1 lentelė. Naminės pelės mėginių charakteristikos}}
    \begin{tabular}{|c|c|c|c|c|}
        \hline
        \textbf{Ląstelių tipas} & \textbf{\thead{Kamienas}} &
        \textbf{\thead{Poveikis}} & \textbf{Antikūnai} & \textbf{\thead{GTRD ID}} \\
        \hline
        \thead{Th1} & C57BL/6 & - &
        \thead{anti-Gata3} &
        \thead{\href{https://gtrd.biouml.org/\#!table/gtrd_current.experiments/Details/ID=EXP000787}{EXP000787}}\\ 
        \hline
        \thead{Th2} & C57BL/6 & - &
        \thead{anti-Gata3} &
        \thead{\href{https://gtrd.biouml.org/\#!table/gtrd_current.experiments/Details/ID=EXP000798}{EXP000798}}\\ 
        \hline
        \thead{Th1} & C57BL/6J & \thead{Poliarizacija (2 d.)} &
        \thead{T-bet\\(Tbx21 sinonimas)} &
        \thead{\href{https://gtrd.biouml.org/\#!table/gtrd_current.experiments/Details/ID=EXP000710}{EXP000710}}\\ 
        \hline
        \thead{Th1} & C57BL/6J & \thead{-} &
        \thead{anti-T-bet} &
        \thead{\href{https://gtrd.biouml.org/\#!table/gtrd_current.experiments/Details/ID=EXP031557}{EXP031557}}\\ 
        \hline
        \thead{Th1} & ? & \thead{-} &
        \thead{?} &
        \thead{\href{https://gtrd.biouml.org/\#!table/gtrd_current.experiments/Details/ID=EXP031709}{EXP031709}}\\ 
        \hline
        \thead{Širdies miocitai} & C57BL/6 & \thead{Kontrolė;\\kairysis prieširdis} &
        \thead{CTCF} &
        \thead{\href{https://gtrd.biouml.org/\#!table/gtrd\_current.experiments/Details/ID=EXP052222}{EXP052222}}\\ 
        \hline
        \thead{Širdies miocitai} & C57BL/6 & \thead{Pašalintas CTCF genas;\\kairysis prieširdis} &
        \thead{CTCF} &
        \thead{\href{https://gtrd.biouml.org/\#!table/gtrd\_current.experiments/Details/ID=EXP052223}{EXP052223}}\\ 
        \hline
    \end{tabular}
\end{table}

\begin{itemize}
    \item \textbf{TRE}: tetraciklino atsako elementas (angl. \emph{Tetracycline
        Response Element}). Tai yra 7 DNR sekos fragmentai,
        sudaryti iš 19 nukleotidų ir atskirti trumpesniais sekų fragmentais.
    \item \textbf{sb431542}: stipriai veikianti, selektyvi cheminė medžiaga;
        augimo faktoriaus inhibitorius.
    \item \textbf{xav939}: stipriai veikianti cheminė medžiaga; tankirazės,
        slopinančios specialaus baltymo, stabdančio telomerazės veiklą,
        jungimąsi prie telomerinių DNR sekų, inhibitorius.
    \item \textbf{AGHMT}: Transkripcijos faktorių, A - Akt1 kinazė, G - GATA4,
        H - HAND2, M - MEF2C, T - Tbx5, komplektas.
    \item \textbf{GHMT, GMT}: sumažintas transkripcijos faktorių
        komplektas.
    \item \textbf{sc-17866x/sc-17866}: žmonių, pelių ir žiurkių Tbx5 antigeną
        atpažįstantys antikūnai (išskirti iš ožkų).
\end{itemize}
\newpage

ChIP sekoskaitos mėginiai, išgauti iš žmogaus, aprašyti antroje lentelėje.

\begin{table}[htb]
    \newcolumntype{M}[1]{>{\centering\arraybackslash}m{#1}}
    \small
    \caption*{\small\textbf{2 lentelė. Žmogaus mėginių charakteristikos}}
    \begin{tabular}{|c|c|c|c|c|}
    \hline
    \textbf{Ląstelių tipas} & \textbf{\thead{Kamienas}} &
    \textbf{\thead{Poveikis}} & \textbf{Antikūnai} & \textbf{\thead{GTRD ID}} \\
    \hline
    \thead{Th1} & - & - &
    \thead{GATA-3} &
    \thead{\href{https://gtrd.biouml.org/\#!table/gtrd\_current.experiments/Details/ID=EXP031083}{EXP031083}}\\ 
    \hline
    \thead{Th2} & - & - &
    \thead{GATA-3} &
    \thead{\href{https://gtrd.biouml.org/\#!table/gtrd\_current.experiments/Details/ID=EXP031084}{EXP031084}}\\ 
    \hline
    \thead{Th1} & - & \thead{anti-CD3/CD28} &
    \thead{anti-T-bet} &
    \thead{\href{https://gtrd.biouml.org/\#!table/gtrd_current.experiments/Details/ID=EXP037440}{EXP037440}}\\ 
    \hline
    \thead{Th1} & - & \thead{-} &
    \thead{T-bet} &
    \thead{\href{https://gtrd.biouml.org/\#!table/gtrd_current.experiments/Details/ID=EXP031082}{EXP031082}}\\ 
    \hline
    \thead{Th1} & ? & \thead{-} &
    \thead{?} &
    \thead{\href{https://gtrd.biouml.org/\#!table/gtrd_current.experiments/Details/ID=EXP031709}{EXP031709}}\\
    \hline
    \end{tabular}
\end{table}

\newpage

%%%%%%%%%%%%%%%%%%%
% TYRIMO METODAI
%%%%%%%%%%%%%%%%%%%

\section{Tyrimo metodai}
Transkripcijos faktorių taikinių spėjimo pasirinktame organizme metodas
įgyvendintas su R programavimo kalba\cite{R} (4.2.3 versija).

\subsection{Duomenų kokybės įvertinimas}
Genominių duomenų kokybės įvertinimui panaudotos Kursinio darbo bei Kursinio
projekto metu atliktos duomenų kokybės įvertinimo analizės. Taip pat
duomenų kokybės įvertinimo analizių sąrašas papildytas 3 naujomis analizėmis.

\subsubsection*{Pikų skaičius mėginiuose}
Pirmajame duomenų kokybės įvertinimo etape panaudota Kursiniame darbe
įgyvendinta analizė, pikų skaičiaus nustatymui mėginiuose pritaikanti bazinę R
programavimo kalbos funkciją \emph{length()}, apskaičiuojančią, kiek kiekviename
mėginyje yra pikų regionų.

\subsubsection*{Pikų skaičius chromosomose}
Atliekant šį duomenų kokybės įvertinimo etapą Kursiniame darbe panaudota
funkcija \emph{facet\_wrap()}, kuri nustato, kiek pikų yra nustatyta skirtingose
chromosomose. Ankstesniame darbe buvo naudojamos tik naminės pelės
(lot. \emph{Mus musculus}) chromosomos, šiame darbe organizmų, kurių genominiai
duomenys gali būti analizuojami, sąrašas papildytas, todėl šiame duomenų
įvertinimo etape analizuojamos tik tos chromosomos, kurios yra būdingos
organizmams, iš kurių išgauti audinių mėginiai yra analizuojami.

\subsubsection*{Mėginių panašumas}
Mėginių tarpusavio panašumui įvertinti panaudota Kursinio darbo metu
realizuota modifikuota \emph{Jaccard()} funkcija. Mėginių panašumas
vizualizuotas, panaudojus spalvų intensyvumo grafiką - pritaikius \emph{ggplot2}
bibliotekos funkciją \emph{geom\_tile()}.

\subsubsection*{Genominė distribucija}
Šio Kursinio projekto metu duomenų kokybės vertinimo rezultatas, pikams
priskiriant artimiausio geno pavadinimą (anotuojant piką), pateiktas
lentelės pavidalu. Šiame darbe pastarasis rezultato atvaizdavimas patobulintas:
kiekvienam mėginiui sukurtas grafikas, vaizduojantis kiekvieno genominio
elemento procentinę dalį. Pastaroji vizualizacija sukurta, pritaikius
\emph{ChIPseeker}\cite{CHIPSEEKER}

\subsubsection*{Atstumas iki TSS}
Atstumo iki artimiausio transkripcijos pradžios taško
(angl. \emph{Transcription Start Site}) nustatymas įgyvendintas, pritaikius
\emph{ChIPseeker} bibliotekos funkciją \emph{annotatePeak()}, kuriai argumentų
pavidalu perduoti pikus aprašantys \emph{GRanges} objektai bei visus žinomus
konkretaus organizmo genus aprašantis \emph{TxDb} objektas. Atlikus pikų
anotavimą gaunama genominių elementų procentines dalis apibendrinanti lentelė,
kuri perduota funkcijai \emph{plotDistToTSS()}. Pastaroji funkcija šiame duomenų
kokybės vertinimo etape sukūrė grafiką, atvaizduojantį kiekvieno mėginio pikų
atstumą iki \emph{TSS} regiono.

\subsubsection*{Pikų profilio atvaizdavimas}
Prieš skaičiuojant pikų, kurie jungiasi prie \emph{TSS} regionų, profilį yra
paruošiami tie \emph{TSS} regionai, kurie yra vadinami šalia \emph{TSS}
einančiais regionais (angl. \emph{flanking sequences}). Šie regionai nustatyti
pritaikius \emph{ChIPseeker} bibliotekos funkciją \emph{getPromoters()}.
Pastarajai funkcijai perduotas konkretaus organizmo \emph{TxDb} objektas.
Nustačius šalia \emph{TSS} regionų esančių sekų regionus panaudota
\emph{getTagMatrix()} funkcija, kuri sukuria nuskaitymų, kurie patenka į šalia
\emph{TSS} sričių esančius sekų regionus, matricą. Gautą matricą perdavus
funkcijai \emph{plotAvgProf()} gautas pikų profilis.

\subsection{Duomenų analizės atlikimas}

\subsection{Transkripcijos faktoriaus taikinių spėjimas}

\subsection{Interaktyvios aplikacijos kūrimas}

Programa, apdorojanti naudotojo įvestus \emph{BED} formato failus ir atliekanti
papildomas analizes, sukurta su R programavimo kalbos biblioteka
\emph{Shiny}\cite{SHINY}. Pastarosios bibliotekos funkcijos leidžia sukurti
interaktyvias internetines aplikacijas (angl. \emph{Interactive Web App}),
naudojant R bei internetinių puslapių kūrimo kalbų - HTML, CSS, JavaScript -
funkcijas.

\subsubsection*{Duomenų įkėlimas}
Naudotojas gali įkelti vieną arba daugiau \emph{BED} formato failų, saugančių
genominę \emph{ChIP} sekoskaitos mėginių informaciją. Duomenų įkėlimo skiltyje
specifikuojami papildomi parametrai, kurie panaudojami atliekant duomenų kokybės
vertinimą, duomenų analizes bei konkretaus transkripcijos faktoriaus taikinių
spėjimą pasirinktame organizme. Programos naudotojas gali nurodyti
transkripcijos faktoriaus pavadinimą, pateikti nurodyto transkripcijos
faktoriaus pozicinę svorių matricą bei pasirinkti, kokio organizmo genome
norima atlikti transkripcijos faktoriaus taikinių spėjimą.

\subsubsection*{Duomenų kokybė}
Duomenų kokybės skiltyje pateikiama įkeltų \emph{ChIP} sekoskaitos duomenų
lentelė, kurios elementus naudotojas gali pasirinkti bei vykdyti duomenų
kokybės vertinimo etapus tik pasirinktiems mėginiams. Taip pat naudotojas
gali apjungti pasirinktus arba visus įkeltus mėginius į vieną duomenų rinkinį
(šis funkcionalumas įgyvendintas, pritaikius bazinę R funkciją \emph{union()})
arba analizuoti tik pasirinktų arba pažymėtų mėginių persidengiančius pikus
(įgyvendinta, panaudojus funkciją \emph{intersect()}).

\subsubsection*{Analizės}

Čia analizes apima: transkripcijos faktoriaus motyvo pavaizdavimas, PWM
matricos atitikimų skaičiavimas, GO analizė, motyvų paieška De-novo.

\subsubsection*{Taikinių spėjimas}

\newpage

%%%%%%%%%%%%%%%%%%%%%%%%%%%%%%%
% REZULTATAI IR JŲ APTARIMAS
%%%%%%%%%%%%%%%%%%%%%%%%%%%%%%%

\section{Rezultatai ir jų aptarimas}
\newpage

%%%%%%%%%%%%
% IŠVADOS
%%%%%%%%%%%%

\section{Išvados}
\newpage

%%%%%%%%%%%%%%%
% LITERATŪRA
%%%%%%%%%%%%%%%

\bibliographystyle{plain}
\begin{thebibliography}{99}

\bibitem{GTRD} GTRD: an integrated view of transcription regulation.
Kolmykov S, Yevshin I, Kulyashov M, Sharipov R, Kondrakhin Y, Makeev VJ,
Kulakovskiy IV, Kel A, Kolpakov F Nucleic Acids Res. 2021 Jan
8;49(D1):D104-D111.

\bibitem{ARTICLE1} Nakato R, Sakata T. Methods for ChIP-seq analysis: A
practical workflow and advanced applications. Methods. 2021 Mar;187:44-53.
doi: 10.1016/j.ymeth.2020.03.005. Epub 2020 Mar 30. PMID: 32240773.

\bibitem{ARTICLE2} \url{https://www.cd-genomics.com/chip-seq.html}

\bibitem{SONICATION}
\url{https://goldbio.com/articles/article/how-to-fragment-DNA-for-NGS}

\bibitem{ARTICLE3} This is article
\url{https://www.nature.com/articles/nmeth.f.247}

\bibitem{ARTICLE4} ARTICLE
\url{https://bmcbiol.biomedcentral.com/articles/10.1186/1741-7007-8-56}

\bibitem{ARTICLE5} \url{https://www.cd-genomics.com/the-advantages-and-workflow-of-chip-seq.html}

\bibitem{ARTICLE6} Langmead, B., Trapnell, C., Pop, M. et al. Ultrafast and
memory-efficient alignment of short DNA sequences to the human genome. Genome
Biol 10, R25 (2009). https://doi.org/10.1186/gb-2009-10-3-r25

\bibitem{ARTICLE7} Li H, Durbin R. Fast and accurate short read alignment with
Burrows-Wheeler transform. Bioinformatics. 2009 Jul 15;25(14):1754-60. doi: 10.
1093/bioinformatics/btp324. Epub 2009 May 18. PMID: 19451168; PMCID: PMC2705234.

\bibitem{ARTICLE8} \url{https://www.sciencedirect.com/science/article/pii/S1046202320300591#b0080}

\bibitem{ARTICLE9} Lee TI, Young RA. Transcriptional regulation and its
misregulation in disease. Cell. 2013 Mar 14;152(6):1237-51.
doi: 10.1016/j.cell.2013.02.014. PMID: 23498934; PMCID: PMC3640494.

\bibitem{ASP1} \url{https://myriad.com/womens-health/diseases/autoimmune-polyglandular-syndrome-type-1/}

\bibitem{ARTICLE10} \url{https://www.sciencedirect.com/science/article/pii/S0092867418301065}

\bibitem{ARTICLE11} Mundade R, Ozer HG, Wei H, Prabhu L, Lu T. Role of ChIP-seq
in the discovery of transcription factor binding sites, differential gene
regulation mechanism, epigenetic marks and beyond. Cell Cycle.
2014;13(18):2847-52. doi: 10.4161/15384101.2014.949201. PMID: 25486472;
PMCID: PMC4614920.

\bibitem{ARTICLE12} Zhang, Y., Liu, T., Meyer, C.A. et al. Model-based Analysis
of ChIP-Seq (MACS). Genome Biol 9, R137 (2008).
https://doi.org/10.1186/gb-2008-9-9-r137

\bibitem{ARTICLE13} \url{https://epigenie.com/wp-content/uploads/2013/02/Peak-Calling-for-ChIP-Seq.pdf}

\bibitem{ARTICLE14} Guo Y, Papachristoudis G, Altshuler RC, Gerber GK, Jaakkola
TS, Gifford DK, Mahony S. Discovering homotypic binding events at high spatial
resolution. Bioinformatics. 2010 Dec 15;26(24):3028-34.
doi: 10.1093/bioinformatics/btq590. Epub 2010 Oct 21. PMID: 20966006;
PMCID: PMC2995123.

\bibitem{ARTICLE15} Guo Y, Mahony S, Gifford DK. High resolution genome wide
binding event finding and motif discovery reveals transcription factor spatial
binding constraints. PLoS Comput Biol. 2012;8(8):e1002638. doi: 10.1371/journal.
pcbi.1002638. Epub 2012 Aug 9. PMID: 22912568; PMCID: PMC3415389.

\bibitem{ARTICLE16} Ji H, Jiang H, Ma W, Wong WH. Using CisGenome to analyze
ChIP-chip and ChIP-seq data. Curr Protoc Bioinformatics. 2011 Mar;
Chapter 2:Unit2.13. doi: 10.1002/0471250953.bi0213s33. PMID: 21400695;
PMCID: PMC3072298.

\bibitem{R} R Core Team (2022). R: A language and environment for statistical
computing. R Foundation for Statistical Computing, Vienna, Austria. URL
https://www.R-project.org/.

\bibitem{SHINY} Chang W, Cheng J, Allaire J, Sievert C, Schloerke B, Xie Y,
Allen J, McPherson J, Dipert A, Borges B (2023). shiny: Web Application
Framework for R. R package version 1.7.4.9002, \url{https://shiny.rstudio.com/}.

\bibitem{CHIPSEEKER} Wang Q, Li M, Wu T, Zhan L, Li L, Chen M, Xie W, Xie Z,
Hu E, Xu S, Yu G (2022). “Exploring epigenomic datasets by ChIPseeker.”
Current Protocols, 2(10), e585. doi: 10.1002/cpz1.585.

\end{thebibliography}
\newpage

%%%%%%%%%%%%
% PRIEDAI
%%%%%%%%%%%%

\section{Priedas} \label{Priedas}

Priedų sąraše pateikiamos tarpinių rezultatų puslapio, sugeneruoto su Scikick,
bei Git repozitorijos, kurioje saugomi analizei naudoti duomenų failai,
parašyti skriptai bei pagrindinė R programa, nuorodos.

\begin{itemize}
    \item \textbf{Kursinio projekto analizės Git repozitorija:}\\
        \url{https://github.com/dansta0804/Tbx5\_analysis\_II.git}
    \item \textbf{Kursinio darbo Scikick ataskaita:}\\
        \url{https://karklas.mif.vu.lt/\string~dast6577/KursinisDarbas/v2.0/peaks\_MM.html}
    \item \textbf{Kursinio darbo analizės Git repozitorija:}\\
        \url{https://github.com/dansta0804/Tbx5\_analysis.git}
\end{itemize}

\end{document}