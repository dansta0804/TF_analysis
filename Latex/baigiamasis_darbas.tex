\documentclass[12pt]{article}
\usepackage{indentfirst}
\usepackage[utf8x]{inputenc}
\usepackage[T1]{fontenc}
\usepackage[english,lithuanian]{babel}
\usepackage{array}
\usepackage{caption}
\usepackage{subcaption}
\usepackage{makecell}
\usepackage[euler]{textgreek}
\usepackage{multirow}
\usepackage{boldline}
\usepackage{floatrow}
\floatsetup[table]{capposition=top}
\usepackage{amsmath, amsthm, amssymb}
\usepackage{graphicx}
\usepackage{setspace}
\usepackage{verbatim}
\usepackage[left=3cm,top=2cm,right=1.5cm,bottom=2cm]{geometry}
\usepackage{floatrow}
\newfloatcommand{capbtabbox}{table}[][\FBwidth]
\usepackage{blindtext}
\onehalfspacing
\usepackage[hidelinks, unicode]{hyperref}
\usepackage{textcomp}
\usepackage{amsmath}

\newcommand{\EE}{\mathbb{E}\,}
\newcommand{\ee}{{\mathrm e}}
\newcommand{\dd}{{\mathrm d}}
\newcommand{\RR}{\mathbb{R}}

\begin{document}
\selectlanguage{lithuanian}

\begin{titlepage}
\vskip 20pt
\begin{center}
\includegraphics[scale=0.5]{MIF}
\end{center}

%%%%%%%%%%%%%%%%%%%%%%%
% TITULINIS PUSLAPIS
%%%%%%%%%%%%%%%%%%%%%%%

\vskip 20pt
\centerline{\bf \large \textbf{VILNIAUS UNIVERSITETAS}}
\bigskip
\centerline{\large \textbf{MATEMATIKOS IR INFORMATIKOS FAKULTETAS}}
\bigskip
\centerline{\large \textbf{BIOINFORMATIKOS BAKALAURO STUDIJŲ PROGRAMA}}

% Sukurti algoritmą bei jo implementaciją R \emph{Shiny}, kurie leistų nuspėti
% transkripcijos faktorių prisijungimo vietas, remiantis sekų paašumu.


\vskip 90pt
\begin{center}
    {\bf \LARGE Transkripcijos faktorių taikinių spėjimo metodo implementacija
    \emph{Shiny} aplikacijoje}
\end{center}
\begin{center}
    {\bf \Large The implementation of transcription factors' target prediction
    in \emph{Shiny} application}
\end{center}
\vskip 20pt
\centerline{\bf \large \textbf{Bakalauro baigiamasis darbas}}
\bigskip
\vskip 40pt

\hskip 140pt {\large Autorė: Danielė Stasiūnaitė}

\hskip 140pt{\large VU el. p.: daniele.stasiunaite@mif.stud.vu.lt}
\bigskip
\vskip 20pt

\hskip 140pt {\large Darbo vadovė: J. m. d. Kotryna Kvederavičiūtė}
\vskip 60pt
\vskip 40pt
\centerline{\large \textbf{Vilnius}}
\centerline{\large \textbf{2023}}
\newpage
\end{titlepage}

\selectlanguage{lithuanian}

%%%%%%%%%%%%%%%%%%%%%
% TURINIO PUSLAPIS
%%%%%%%%%%%%%%%%%%%%%

\tableofcontents
\newpage

%%%%%%%%%%%%%%%%%%%%%%%%%%%%%%%%%%%%
% LIETUVIŠKOS SANTRAUKOS PUSLAPIS
%%%%%%%%%%%%%%%%%%%%%%%%%%%%%%%%%%%%

\section*{Santrauka}
Šiais laikais plačiai atliekami biologinių procesų tyrimai yra itin svarbūs,
siekiant suprasti įvairiuose organizmuose vykstančius gyvybiškai svarbius
biologinius procesus. Pastarųjų
procesų mechanizmų supratimas daro įtaką šiuolaikiniui įvairių žmogaus ligų
gydymui bei prevencijai. Pavyzdžiui, tobulėjantys tyrimų metodai bei įrankiai
padėjo išsiaiškinti, kad vėžiniai susirgimai, neurologiniai sutrikimai, širdies
ir kraujagyslių ligos, diabetas ir net nutukimas gali būti reguliatorinių sekų, 
transkripcijos faktorių bei kitų, genų reguliavime dalyvaujančių elementų,
mutacijų pasekmė\cite{ARTICLE0}.

Genų reguliavimas yra vienas iš svarbiausių procesų, kurio metu yra
ekspresuojami konkretūs genai. Baltymus koduojančių genų ekspresijos metu yra
susintetinami baltymai, atliekantys įvairias funkcijas: įeinantys į struktūrinių
audinių sudėtį bei organizmų apsauginių elementų rinkinį, turintys medžiagų
pernašos bei organizmo homeostazės palaikymo funkcionalumą, dalyvaujantys
organizmų embriogenezės, tolimesnio vystymosi bei kituose biologiniuose
procesuose, užtikrinančiuose normalų organizmų funkcionavimą. Atitinkamų genų
aktyvavimas bei išaktyvinimas yra valdomas transkripcijos faktorių, kurie lemia
organizmui reikalingų baltymų sintezę, arba, įvykus transkripcijos faktorių
mutacijoms ar pasireiškus nepalankioms sąlygoms, ligų atsiradimą.

Šiame darbe, naudojantis R programavimo kalbos bibliotekomis bei
bioinformatiniais komandinės eilutės įrankiais, sukurtas transkripcijos faktorių
taikinius pasirinkto organizmo genome spėjantis metodas, pritaikantis pozicines
transkripcijos faktorių matricas (PWM). Šis metodas kartu su ChIP sekoskaitos
duomenis apdorojančiomis biologinėmis analizėmis implementuotas \emph{Shiny}
aplikacijoje. Naudojantis sukurta aplikacija galima analizuoti iš skirtingų
organizmų išgautus ChIP sekoskaitos duomenis bei, pagal poreikį, panaudoti
sugeneruotas rezultatų lenteles ir vizualizacijas.

\hfill \break
\textbf{Raktiniai žodžiai:} genų reguliavimas, transkripcijos faktorius, 
transkripcijos faktoriaus taikinys, R, \emph{Shiny} aplikacija, pozicinė svorių
matrica, ChIP sekoskaita.

\newpage

%%%%%%%%%%%%%%%%%%%%%%%%%%%%%%%%%%
% ANGLIŠKOS SANTRAUKOS PUSLAPIS
%%%%%%%%%%%%%%%%%%%%%%%%%%%%%%%%%%

\section*{Summary}
The investigation of biological processes is remarkably important in order to
understand the processes that occur in all living organisms. The understanding
of aforementioned processes and their mechanisms plays a crucial role in
disease treatment and prevention. For instance, the latest advances in research
methods and tools helped to determine that diseases like cancer, neurological
disorders, cardiovascular diseases, diabetes and even obesity can be caused
by mutations that occur in regulatory sequences, transcription factors and
other elements that are involved in gene regulation\cite{ARTICLE0}.

Gene regulation is one of the biological processes that invokes expression of
particular genes. During the expression of protein coding genes the proteins
that perform different functions are synthethised. The functions involve the
formation of structural frameworks, immune function, substance transportation,
homeostasis, embryogenesis, later development and other biological processes
that maintain proper functioning of all living organisms. The activation and
inactivation of certain genes is controlled by transcription factors that can
either start protein synthesis or, in case of being mutation - affected,
result in the development of the diseases.

In this work, a method that predicts transcription factor target sites in
specified genome by using positional weight matrix (PWM) was created using R
programming language libraries and bioinformatical command line tools. The
method along with biological analyses that process ChIP sequencing data was
implemented in \emph{Shiny} application. The application allows to analyze
sequencing data that is retrieved from various organisms' samples. Furthermore,
the application is capable of generating result tables and visualizations that
can be downloaded and used in research.

\hfill \break
\textbf{Keywords:} gene regulation, transcription factor, transcription factor's
target, R, \emph{Shiny} application, positional weight matrix, ChIP
sequencing.

\newpage

%%%%%%%%%%%%%%%%%%%
% ĮVADO PUSLAPIS
%%%%%%%%%%%%%%%%%%%

\section{Įvadas}
\subsection*{Darbo temos aktualumas}
Yra sukurtas ne vienas bioinformatinis įrankis\cite{BIOTOOLS} bei metodas,
leidžiantis apdoroti bei analizuoti gautus biologinius - DNR, RNR, baltymų sekų,
struktūrinius, genų ekspresijos, mutacijų, metabolinių kelių ir kitus -
duomenis. Sekoskaitos duomenys yra vienas iš biologinių duomenų tipų, kuris,
priklausomai nuo to, kokio tipo sekoskaita buvo atlikta ir kas yra tiriama, gali
būti analizuojamas, taikant aprašytas duomenų kompiuterines analizes
(pavyzdžiui, RNR sekoskaitos duomenų analizės eiga\cite{RNASEQ}).

Iš didžiulės biologinius duomenis galinčių apdoroti įrankių įvairovės galima
pasirinkti ir pritaikyti ne vieną įrankį ar metodą, kuris padėtų gauti
interpretuojamą rezultatą, tačiau skirtingų įrankių paieška bei taikymas yra ne
visada patogi vykdomo tyrimo dalis. Dėl šios priežasties šio darbo metu
sukurta populiarios ir neretai biologinius duomenis analizuojančių specialistų
naudojamos \emph{Shiny} aplinkos aplikacija. Ši aplikacija leidžia atlikti
didžiąją dalį ChIP sekoskaitos duomenų analizės etapų bei pritaikyti metodą,
nuspėjantį galimas transkripcijos faktoriaus prisijungimo sritis pasirinkto 
organizmo genome.

\subsection*{Darbo tikslas}
Sukurti algoritmą bei jo implementaciją R \emph{Shiny}, kurie leistų nuspėti
transkripcijos faktorių prisijungimo vietas, remiantis sekų panašumu.

\subsection*{Uždaviniai}
\begin{itemize}
    \item Išanalizuoti metodus, taikomus praturtintų nuskaitymų - pikų -
          nustatymui;
    \item Įgyvendinti transkripcijos faktorių taikinių spėjimo metodą;
    \item Modifikuoti ir adaptuoti sukurtas R programas, galinčias apdoroti
          ChIP sekoskaitos duomenis iš skirtingų organizmų;          
    \item Sukurti internetinę aplikaciją, leidžiančią įvertinti pateiktų
          ChIP sekoskaitos duomenų kokybę bei atlikti analizes.

\end{itemize}

\newpage

%%%%%%%%%%%%%%%%%%
% TEORINĖ DALIS
%%%%%%%%%%%%%%%%%%

\section{ChIP sekoskaita ir jos vykdymo ypatumai}

\textbf{DNR sekoskaita} - deoksiribonukleorūgšties nukleotidų sekos nustatymas.
Nukleotidų sekų nustatymui gali būti naudojami du pagrindiniai DNR sekvenavimo 
metodai:

\begin{itemize}
    \item \emph{Sendžerio} metodas;
    \item Didelio našumo arba naujos kartos sekoskaita
    (angl. \emph{NGS - \textbf{N}ext \textbf{G}eneration \textbf{S}equencing}).
\end{itemize}

Šiuolaikiniuose tyrimuose labai dažnai taikomos modernios DNR sekos - naujos 
kartos se\-kos\-kai\-tos - technologijos, kurios leidžia nuskaityti didelį kiekį
DNR arba RNR sekų daug sparčiau ir pigiau nei klasikinė \emph{Sendžerio} 
sekoskaita. Pastaroji sekoskaita dažniau naudojama, tiriant mažus duomenų
rinkinius.

\subsection{Transkripcijos faktoriai ir jų prasmė}

\textbf{Transkripcijos faktorius} - ypatingas baltymų tipas. Šio tipo baltymai
atpažįsta specifines DNR sekas ir tokiu būdu kontroliuoja chromatino struktūros
kondensacijos laipsnį bei atitinkamų genų ekspresijos procesus, inicijuojant
arba slopinant genų transkripciją. Šių baltymų sąveika su DNR įtakoja daug
biologiškai svarbių procesų: ląstelių diferenciaciją, ląstelės ciklo eigą, genų
transkripciją, DNR replikaciją, imuninio atsako valdymą ir daugelį kitų
procesų\cite{ARTICLE10, ARTICLE11}.

Transkripcijos faktoriai bei sekos, prie kurių jie jungiasi, gali
mutuoti. Šios pastarųjų baltymų mutacijos nulemia įvairių ligų išsivystymą.
Pavyzdžiui, atsiradusios mutacijos AIRE (autoimuniniame reguliatoriuje)
transkripcijos faktoriuje sukelia I tipo autoimuninį poliendokrinopatijos
sindromą\cite{ARTICLE9} (angl. \emph{APS1 - \textbf{A}utoimmune
\textbf{P}olyendocrinopathy \textbf{S}yndrome type \textbf{I}}). Pasireiškus
šiam sindromui organizmo imuninės ląstelės naikina sveikas, hormonus
išskiriančių liaukų ląsteles\cite{ASP1}, todėl transkripcijos faktorių veikimo
mechanizmų supratimas yra itin svarbus, siekiant diagnozuoti bei išgydyti ligas,
kurios gali būti susijusios su transkripcijos faktorių baltymų mutacijomis.

\subsection{ChIP sekoskaitos apibūdinimas}
\textbf{ChIP sekoskaita} - chromatino imunoprecipitacijos sekoskaita (angl.
\emph{chromatin immunoprecipitation sequencing}). Tai yra viena iš svarbiausių
technologijų, atliekant epigenetikos tyrimus\cite{ARTICLE1}. Šiam metodui
būdingas klasikinio eksperimentinio chromatino precipitacijos metodo derinimas
su naujos kartos (angl. \emph{next generation (NGS) sequencing}) sekoskaita,
siekiant išsiaiškinti baltymų sąveikas su DNR ir nustatyti, kaip transkripcijos
faktoriai ir kiti, su chromatinu susiję baltymai, gali įtakoti įvairius fenotipo
pokyčių mechanizmus\cite{ARTICLE2}.

ChIP sekoskaita neretai taikoma, atliekant: histonų modifikavimo, genų
reguliacijos, transkripcijos komplekso surinkimo, DNR pažaidų taisymo,
vystymosi mechanizmų bei ligų progresavimo tyrimus.

% KABUTĖS: „ “

\subsection{ChIP sekoskaitos vykdymo eiga}

DNR ir baltymų sąveikos tyrimo - ChIP sekoskaitos - eiga pateikta pirmame
paveiksle:

\begin{figure}[ht]
    \begin{center}
        \includegraphics[width=0.7\linewidth]{../Figures/ChIP-seq_workflow.jpg}
        \vspace{-1\baselineskip}
        \caption*{\small\textbf{1 pav. ChIP sekoskaitos vykdymo etapai}}
    \end{center}
\end{figure}

Remiantis pateiktu paveikslu ChIP sekoskaita gali būti suskirstyta į du
pagrindinius etapus:

\begin{enumerate}
    \item Mėginių paruošimas ir sekvenavimas;
    \item Kompiuterinė analizė.
\end{enumerate}

\subsubsection{Mėginių paruošimas ir sekvenavimas}
Mėginių paruošimo bei sekvenavimo etapai įprastai skirstomi į šiuos etapus:

\begin{itemize}
    \item \textbf{Baltymo prijungimas prie DNR.} Šiame etape transkripcijos faktorius susiejamas su DNR, naudojant įvairias chemines medžiagas (formaldehidas naudojamas dažniausiai). Ši baltymo ir DNR fiksacija,
        naudojant chemines medžiagas, padeda išlaikyti baltymo - DNR
        sąsają. Šios sąsajos išsaugojimas yra būtina sąlyga imunoprecipitacijos
        proceso vykdymui.
    \item \textbf{DNR suskaidymas į fragmentus.} \emph{NGS} bibliotekos 
        paruošimui reikalingas DNR suskaidymo į fragmentus etapas. Šiame etape
        DNR dažniausiai suskaidoma į 150 - 500 nukleotidus turinčius fragmentus,
        naudojant ultragarso bangas - sonifikacijos mechanizmą. Pastarojo
        mechanizmo sukurtas vibracijos rezonansas suskaido DNR į fragmentus.
        Fragmentų ilgis gali būti kontroliuojamas sonifikacijos įrenginio
        naudojimo dažniu. Pavyzdžiui, kuo ilgiau trunka vienas ultra garso
        panaudojimo ciklas, tuo trumpesni DNR fragmentai
        gaunami\cite{SONICATION}.
    \item \textbf{Imunoprecipitacijos procesas.} Suskaidytos DNR fragmentai
        inkubuojami su specifiniu antikūnu, galinčiu atpažinti prie DNR
        prisijungusį baltymą - transkripcijos faktorių. Tam, jog ChIP
        sekoskaitos rezultatai būtų patikimi ir tinkami, tinkamas antikūno
        parinkimas ir jo kokybės užtikrinimas yra vienas iš svarbiausių ChIP
        sekoskaitos mėginių paruošimo etapų\cite{ARTICLE3}. Testuojant
        skirtingus antikūnus pasirenkamas tas, kurį panaudojus gaunamas
        didesnis DNR sekų, prie kurių prisijungęs transkripcijos faktorius,
        praturtinimas nei praturtinimas, kuris gautas, naudojant nespecifinį
        antikūną\cite{ARTICLE4} (pavyzdžiui, naudojant tipinį imunoglobulino G
        (IgG) antikūną).
    \item \textbf{Sekvenavimas.} Neretai sekvenavimo įrenginių pritaikymui
        reikalingi trumpų adapterių prijungimo prie gautų DNR fragmentų ir PGR
        amplifikacijos etapai - reikalingas bibliotekos sukonstravimas, kuris
        gali skirtis, priklausomai nuo pasirinktos sekvenavimo platformos
        ir jai specifinės bibliotekos paruošimo protokolo\cite{ARTICLE5}.
        Įvykdžius šiuos etapus gali būti gautas \colorbox{red}{„triukšmas“}\cite{ARTICLE4}
        (angl. \emph{bias}), kuris gali būti mažesnis, atliekant mažiau DNR
        amplifikacijos (padauginimo) ciklų. Sukonstravus biblioteką atliekamas
        \emph{NGS} sekvenavimas.
\end{itemize}

Įgyvendinus mėginių paruošimo ir sekvenavimo etapus atliekama gautų duomenų
kompiuterinė analizė.

\subsubsection{Kompiuterinė analizė}
Gauti ChIP sekoskaitos duomenys apdorojami ir analizuojami, vykdant šiuos
etapus:

\begin{itemize}
    \item \textbf{DNR nuskaitymų kartografavimas.} Nusekvenuoti DNR 
        fragmentai išsaugomi \emph{FASTQ} arba \emph{CSFSATQ} formatais. Šie DNR
        nuskaitymai (angl. \emph{reads} arba \emph{tags}) perkeliami ant genomo,
        naudojant kartografavimo (angl. \emph{mapping}) įrankius, pavyzdžiui,
        \emph{Bowtie}\cite{ARTICLE6}, \emph{Burrows-Wheeler}, kurie leidžia
        nustatyti nuskaityto DNR fragmento poziciją genome, esant kelių
        nukleotidų neatitikimui\cite{ARTICLE7} (angl. \emph{mismatch}). Atlikus
        DNR nuskaitymų priskyrimą gaunami \emph{SAM}, \emph{BAM} (dažniausiai
        naudojamas formatas), \emph{CRAM} arba \emph{tagAlign} formato failai.
    \item \textbf{Normalizavimas.} Normalizavimas reikalingas, siekiant
        sumažinti techninį nuskaitymų variabilumą - sumažinti „triukšmą“,
        atsiradusį dėl sekvenavimo gylio skirtumų tarp mėginių. Normalizavimas
        atliekamas konkrečiose genomo pozicijose esančių nuskaitymų skaičių
        padalinus iš bendro nuskaitymų skaičiaus\cite{ARTICLE17}.
    \item \textbf{Pikų nustatymas (angl. \emph{peak calling}).} Šiame etape
        nustatomi reikšmingai praturtinti genomo lokusai - pikai
        (angl. \emph{peaks}). Įgyvendinus šį etapą dažniausiai sugeneruojami
        \emph{BED} formato failai\cite{ARTICLE1}, kuriuose pateikiamos genominės
        pikų pozicijos, įvairūs statistiniai įverčiai bei identifikacijos kodai,
        kuriuos naudojant galima vykdyti įvairias analizes.
    \item \textbf{Biologinės analizės.} Dažniausiai atliekamos analizės yra
        motyvų analizė bei genų ontologijos analizė\cite{ARTICLE1}
        (angl. \emph{GO - \textbf{G}ene \textbf{O}ntology enrichment analysis}),
        pateikianti biologinių procesų, ląstelinių komponentų ir molekulinių
        funkcijų, kuriose dalyvauja genas, sąrašą. Įvertinus motyvų dažnį,
        konservatyvumą bei biologines funkcijas, susijusias su transkripcijos
        faktorių prisijungimu prie šių motyvų, gali būti identifikuoti
        genominiai regionai, prie kurių gali jungtis transkripcijos faktoriai,
        konservatyvūs motyvai, galintys indikuoti baltymas - baltymas sąveiką,
        bei gali būti analizuojama genų evoliucija.
\end{itemize}

Įvykdžius kompiuterinės analizės etapus gaunami rezultatai, suteikiantys
įžvalgų apie genų reguliavimo mechanizmus, padedantys nustatyti
transkripcijos faktorių taikinius ir geriau suprasti ligų progresavimą bei
ieškoti ligos gydymo būdų.

\newpage

\subsection{Pikų nustatymo algoritmai}
Pikų nustatymas yra vienas iš svarbiausių etapų, atliekant DNR ir reguliatorinių
baltymų - transkripcijos faktorių arba histonų - sąveikos tyrimų analizes.
Kuriant pikų nustatymo algoritmus sprendžiamos dvi pagrindinės problemos:
regionų, kurie, tikėtina, yra pikai, nustatymas bei tikėtinų pikų statistinio
reikšmingumo tikrinimas.

Pagrindinė pikų nustatymo algoritmų įvestis yra kartografavimo metu su genomu
išlyginti DNR fragmentų nuskaitymai. Antrajame paveiksle šie fragmentai pažymėti
raudona ir žalia spalvomis.

\begin{figure}[ht]
    \begin{center}
        \captionsetup{justification=centering}
        \includegraphics[width=0.6\linewidth]{../Figures/Read_mapping.png}
        \vspace{-1\baselineskip}
        \caption*{\small\textbf{2 pav. DNR nuskaitymų kartografavimas. Žalia
                                spalva pavaizduoti 5' \(\rightarrow\) 3' DNR
                                galo link kartografuoti nuskaitymai. Raudona
                                spalva - 3' \(\rightarrow\) 5' nuskaitymai.}}
    \end{center}
\end{figure}

Nustačius DNR fragmentų nuskaitymų pozicijas genome kai kurios pastebimos
nuskaitymų sankaupų grupės gali indikuoti, jog toje pozicijoje yra galimas
transkripcijos faktoriaus prisijungimas (nuskaitymų sankaupa yra reikšminga),
tačiau neretai tokios sankaupos - pikai - gali būti laikomos molekuliniu arba
eksperimentiniu „triukšmu“. Taigi kuriant pikų nustatymo algoritmus yra
svarbu, jog algoritmas gebėtų įvertinti, ar pikas yra biologiškai reikšmingas,
ar tai tėra „triukšmas“.

Yra sukurta daugiau nei 30 skirtingų pikų nustatymo algoritmų (angl.
\emph{peak caller}), kurie sprendžia anksčiau minėtas problemas,
tačiau šių problemų sprendimo būdai yra skirtingi. Konkretaus algoritmo
pasirinkimas labai priklauso nuo atliekamo eksperimento tipo ir specialisto,
atliekančio analizę, patirties\cite{ARTICLE13}.

\subsubsection{MACS2}
\textbf{MACS2} - \textbf{M}odel-based \textbf{A}nalysis of
\textbf{C}hIP-\textbf{S}eq. Tai yra populiariausias ir bene seniausias
pikų nustatymo algoritmas.
Transkripcijos faktorių jungimosi prie DNR saitai nustatomi, atsižvelgus į
nuskaitymų pozicijas bei kryptį. Neretai konkrečioje genomo pozicijoje gali
būti priskirti keli nuskaitymai. MACS2 metode yra numatyta palikti tik po
vieną nuskaitymą konkrečioje pozicijoje (pašalinti duplikatus). Duplikatai
nėra šalinami, jeigu tikimasi, kad transkripcijos faktorius jungsis keliose
skirtingose genomo pozicijose. Tose genomo pozicijose, kuriose, tikėtina,
jungiasi transkripcijos faktorius, turi būti pastebimas \emph{Watsono} ir
\emph{Kriko} nuskaitymų išsidėstymas arba \textbf{bimodalinis pasiskirtymas},
kurio grafikas pavaizduotas paveiksle:

\begin{figure}[ht]
    \begin{center}
        \includegraphics[width=0.5\linewidth]{../Figures/Bimodal_pattern.png}
        \vspace{-1\baselineskip}
        \caption*{\small\textbf{3 pav. Bimodalinis pasiskirstymas}}
    \end{center}
\end{figure}

Tam, jog panaši pikų struktūra būtų surasta, MACS2 algoritmas skanuoja visą
kartografuotų nuskaitymų duomenų rinkinį. Algoritmas naudoja dydį, kuris nurodo,
į kokio ilgio nukleotidų fragmentus buvo skaidoma DNR sonifikacijos proceso
metu (angl. \emph{bandwidth}), bei \emph{mfold} vertę. Vykdant algoritmą
atliekamas \emph{2 * bandwidth} ilgio poslinkis ir ieškoma tokių genomo
pozicijų, kuriose nuskaitymų yra daugiau nei naudojant atsitiktinį nuskaitymų
rinkinį (daugiau už \emph{mfold} vertę).

Nustačius aukštos kokybės pikus yra atsitiktinai parenkama 1000 pikų. Turint
šiuos pikus yra atskiriami jų \emph{Watsono} ir \emph{Kriko} (teigiamos ir
neigiamos grandinės) nuskaitymai.
Šių teigiamų ir neigiamų grandinių pikų grupės yra išlyginamos pagal jų
centrus, kaip pavaizduota 4 paveiksle. Atstumas tarp išlygintų nuskaitymų
modų (\emph{d}) nurodo, kokio ilgio yra piko fragmentas.

\begin{figure}[H]
    \begin{center}
        \includegraphics[width=0.5\linewidth]{../Figures/Tag_alignment.png}
        % \vspace{-1\baselineskip}
        \caption*{\small\textbf{4 pav. Nuskaitymų išlyginimas}}
        \label{fig:birds}
    \end{center}
\end{figure}

Algoritmas visiems nuskaitymams atlieka \emph{d/2} 3' DNR galo link
tikėtiniausio DNR ir transk\-rip\-ci\-jos faktoriaus sąveikos regiono poslinkį.
Atlikus šį poslinkį atliekamas \emph{2 * d} poslinkis, jog būtų surastas
statistiškai reikšmingas nuskaitymų praturtinimas, naudojant \emph{Puasono}
skirstinį, kurio parametras \(\lambda_{BG}\) yra tikėtinas nuskaitymų skaičius
atlikus poslinkį. Nepaisant to, \(\lambda_{BG}\) parametras naudojamas,
neatsižvelgus į galimą „triukšmą“, kuris galėjo kilti dėl chromatino struktūros,
DNR amplifikacijos arba sekvenavimo, todėl yra naudojamas parametras
\(\lambda_{local}\), kuris skaičiuojamas kiekvienam tikėtinam
pikui:

\begin{equation} \label{lambda_local}
    \lambda_{local} = max(\lambda_{BG}, \lambda_{5k}, \lambda_{10k})
\end{equation}

čia \(5k\), \(10k\) yra poslinkio plotis.

Parametro \(\lambda_{local}\) naudojimas leidžia aptikti \emph{false positive},
pikus (pikus, kurie atsirado dėl „triukšmo“) ir nustatyti tik tuos pikus,
kurie indikuoja svarbų DNR ir baltymo sąveikos regioną\cite{ARTICLE12}.
  
\subsubsection{GEM}
\textbf{GEM} - \textbf{G}enome wide \textbf{E}vent finding and \textbf{M}otif
discovery. Šis 2012 metais sukurtas algoritmas išsiskiria tuo, jog jame yra
kombinuojama pikų paieška bei motyvų analizė, jog būtų pagerinta galutinių pikų
rezoliucija\footnote{\textbf{Rezoliucija - } genetikoje aukšta rezoliucija
reiškia, jog yra žinoma itin daug molekulinių detalių apie DNR.}.\\

GEM algoritmą sudaro šeši skirtingi etapai\cite{ARTICLE15}:
\begin{enumerate}
    \item \textbf{Baltymo ir DNR sąveikos regionų nustatymas.} Pradiniai
        regionai nustatomi, taikant \emph{GPS} algoritmą\cite{ARTICLE14},
        kuris naudoja \emph{Dirichlė} skirstinį.
    \item \textbf{Praturtintų \emph{k - merų} nustatymas.} Jie nustatomi,
        lyginant \emph{k - merų} dažnius tarp teigiamų sekų ir neigiamų
        kontrolinių sekų. Teigiamos sekos - sekos, kurios sudarytos iš 61
        bazių poros ir yra išsidėsčiusios spėjamų baltymo ir DNR sąveikos
        regionų (gautų pirmajame etape) centruose. Neigiamos kontrolinės
        sekos - 61 bazių porą turinčios sekos, kurios yra nutolusios nuo
        teigiamų sekų per 300 bazių porų. Be to, šios sekos nepersidengia su
        sekomis, esančiomis baltymo - DNR sąveikos sekų centruose. Šiame
        etape yra skaičiuojami \emph{k - merų} fragmentų atitikimai teigiamų
        ir neigiamų sekų rinkiniuose. \emph{K - meras} (sekos fragmentas)
        yra laikomas praturtintu, kai \emph{p} vertė yra mažesnė nei 0.001.
        % ir 3-fold enrichment in terms of positive/negative hit count?
    \item \textbf{Praturtintų \emph{k - merų} klasterizavimas.} Praturtinti
        \emph{k - merai} klasterizuojami į ekvivalentiškumo klases, kurios
        apibūdina panašias DNR sekas, prie kurių jungiasi transkripcijos
        faktorius. Seka atitinka \emph{k - mero} ekvivalentiškumo klasę, kai
        sekoje nustatomas fragmentas yra vienas iš ekvivalentiškumo klasės
        elementų.
    \item \textbf{Išankstinio pasiskirstymo nustatymas.} Labiausiai praturtinta
        \emph{k - merų} klasė yra naudojama \emph{Dirichlė} išankstinio
        pasiskirstymo paskaičiavimui. Šiame etape genomas yra suskaidomas į
        kelis tūkstančius bazių porų turinčius segmentus. Šie segmentai yra
        gaunami atskiriant DNR fragmentus, kuriuose yra daugiau nei 500 bazių
        porų turintys tarpai bei DNR regionai, kuriems buvo priskirta mažiau
        nei 6 DNR nuskaitymai (angl. \emph{reads}). Šie regionai yra
        skanuojami su DNR sekų fragmentais, kurie priklauso atrinktai
        \emph{k - merų} ekvivalentiškumo klasei, \emph{k - merų}
        atitikimai yra skaičiuojami.
    \item \textbf{Tikslesnių baltymo - DNR sąveikos regionų spėjimas.} Tam yra
        panaudojamas 4 etape gautas išankstinis pozicijų pasiskirstymas.
    \item \textbf{2 - 3 etapų kartojimas.} Tam yra panaudojami 5 etape gauti
        patikslinti baltymo - DNR sąveikos regionai.
\end{enumerate}

\newpage

%%%%%%%%%%%%%%%%%%%%%%%
% METODO REALIZACIJA
%%%%%%%%%%%%%%%%%%%%%%%

\section{Transkripcijos faktorių taikinių spėjimo metodas}
Transkripcijos faktorių taikinių spėjimo metodas leidžia nustatyti, kokiose
pasirinkto organizmo genomo srityse (genuose) yra įmanomas nurodyto
transkripcijos faktoriaus prisijungimas, naudojant kito organizmo genomines
sekas. Metodas realizuotas, įgyvendinus aprašytus etapus.

\subsection{Pikų anotavimas}
Šiame etape yra anotuojami ChIP sekoskaitos mėginiai, išgauti iš organizmo,
kurio duomenys panaudojami, atliekant transkripcijos faktorių taikinių spėjimą.
Taip pat anotuojami ir kontroliniai ChIP sekoskaitos mėginiai, kurie išgauti iš
organizmo, kurio genome spėjamos transkripcijos faktoriaus prisijungimo vietos.
Kontroliniai mėginiai reikalingi tam, jog būtų galima įvertinti metodo tikslumą.
ChIP sekoskaitos duomenys - pikų rinkiniai - anotuojami, pritaikant sukurtą
\emph{annotate\_peak()} funkciją. Šiai funkcijai turi būti perduoti šie
parametrai:

\begin{itemize}
    \item \textbf{\emph{Pikų rinkinys:}} ChIP sekoskaitos duomenų rinkinys,
        saugomas \emph{GRanges} objektų sąraše (\emph{GRangesList}).
    \item \textbf{\emph{Ensembl genomo anotacija:}} organizmų genominės
        anotacijos iš R \emph{annotables} bibliotekos.
    \item \textbf{\emph{TxDb objektas:}} pasirinktų organizmų duomenų bazės,
        kuriose saugoma informacija įprastai atitinka GTF/GFF failų informaciją.
    \item \textbf{\emph{org.x.eg.db anotacija:}} \emph{x} atitinka organizmų
        lotyniškų pavadinimų santrumpą (pavyzdžiui, \emph{Mus musculus} -
        \emph{Mm}).
\end{itemize}

Perduoti parametrai panaudojami \emph{ChIPseeker}\cite{CHIPSEEKER} bibliotekos
funkcijos \emph{annotatePeak()}, kuri pikams priskiria artimiausius genus.
Taip pat šiame etape anotuotiems pikams priskiriami genų identifikacijos
numeriai bei genų simboliai.

\subsection{Aminorūgščių sekų išgavimas}
Anotuotų pikų rinkinių genus atitinkančios aminorūgščių sekos išgaunamos,
panaudojus \emph{ensembldb}\cite{ENSEMBLDB} bibliotekos funkciją 
\emph{proteins()}. Šiai funkcijai perduoti šie parametrai:

\begin{itemize}
    \item \textbf{\emph{Ensembl duomenų bazės objektas:}} specifikavus
        konkretaus organizmo anotacijos objektą, įvykdomas SQL susiejimas su
        duomenų baze, kurioje saugoma konkretaus organizmo \emph{Ensembl}
        anotacija.
    \item \textbf{\emph{Filtras:}} jis taikomas, kad iš \emph{Ensembl} duomenų
        bazės būtų išgauta tik tam tikra informacija. Realizuojant
        transkripcijos faktorių taikinių paieškos metodą taikytas
        \emph{GeneNameFilter()}, kuriam perduodamas pikui priskirto geno
        pavadinimas.
    \item \textbf{\emph{Grąžinamo rezultato tipas:}} siekiant išgauti
        aminorūgščių sekas formatu, kuris būtų paprastai apdorojamas
        tolimesniuose etapuose, pasirinktas \emph{AAStringSet} grąžinamo
        rezultato tipas.
\end{itemize}

\subsection{\emph{Blastp} paieška}
Paieškos atlikimui panaudota \emph{rBlast}\cite{RBLAST} bibliotekos funkcija
\emph{blast()}, kuriai perduotas pasirinkto organizmo baltymų duomenų bazės
failas bei nurodytas \emph{Blast} paieškos tipas - \emph{blastp}.
\emph{rBlast} biblioteka leidžia naudotis lokaliai instaliuotu
\emph{BLAST+}\cite{BLAST}, naudojant \emph{Bioconductor} in\-fra\-struk\-tū\-rą.
Specifikavus duomenų bazę, kurioje atliekama paieška, taikoma funkcija
\emph{predict()}. Šiai funkcijai perduodamas duomenų bazės objektas, pradinio
organizmo genų aminorūgščių sekų rinkinys, \emph{Blast} paieškai naudojamų gijų
skaičius bei galutinio rezultato formatas - identifikacijos numerius, sekų
ilgius, sekų identiškumo bei ieškomų sekų padengimo procentus specifikuojantys
stulpeliai.

\subsection{Paieškos rezultatų apdorojimas}
Gauti \emph{Blast} paieškos rezultatai sugrupuojami pagal identifikacijos
numerius ir atrenkamos tos rezultatų eilutės, kurios atitinka dvejus \emph{dplyr}
bibliotekos filtrus: maksimalų ieškomų sekų padengimo procentą bei maksimalų
sekų identiškumą.

Po filtrų pritaikymo gavus mažesnį rezultatų rinkinį ieškomų sekų \emph{Ensembl}
identifikacijos numeriams priskiriami NCBI genų identifikacijos numeriai bei
genų simboliai. Taip pat šiame metodo realizavimo etape ieškomas sekas
atitinkančioms sekoms priskiriami NCBI identifikacijos numeriai bei genų
simboliai, naudojant bibliotekos \emph{clusterProfiler}\cite{CLUSTERPROFILER}
funkciją \emph{bitr}. Pastaroji funkcija surastoms sekoms priskiria
identifikacijos numerius, naudojant organizmo, kurio genome siekiama atlikti
transkripcijos faktorių taikinių spėjimą, genomo anotacijos objektą
\emph{org.x.eg.db}, kur \emph{x} - organizmo lotyniško pavadinimo pirmosios
raidės.

Ieškomų sekų informacija sujungiama su sekų, turinčių didelį sekų identiškumą
bei sekų padengimą, informacija.

\subsection{Duomenų rinkinio sekų išgavimas}
Šiame etape naudojami du \emph{TxDb} bibliotekos organizmų objektai,
saugantys nurodytų or\-ga\-niz\-mų genominę informaciją. Nukleotidų sekų
išgavimui iš konkretaus organizmo genomo naudojama \emph{BSgenome} bibliotekos
funkcija \emph{getSeq()}. Šiai funkcijai perduodamas pasirinkto organizmo
\emph{BSgenome} objektas bei \emph{GRanges} objektas, kuriame išsaugota
genominių regionų informacija - genominės pozicijos tų genų, kurie specifikuoti
apdorotame duomenų rinkinyje bei patenka į organizmų \emph{TxDb} objektų duomenų
rinkinį.

\subsection{PWM atitikimų nustatymas}
Išgauti kiekvieno organizmo sekų rinkiniai analizuojami atskirai. Kiekvienai
sekų rinkiniui priklausančiai sekai pritaikoma
\emph{Biostrings}\cite{BIOSTRINGS} bibliotekos
funkcija \emph{countPWM()}, kuriai pateikiama pasirinkto transkripcijos
faktoriaus pozicinė svorių matrica bei \emph{min.score} vertė procentais,
nurodanti minimalų sekos fragmento bei pozicinės svorių matricos atitikimą.
Grąžintas pozicinės svorių matricos atitikimų skaičius bei geno pavadinimas
patalpinamas į \emph{data.frame} duomenų struktūrą.

Nustačius, kiek genų sekose esama pozicijų, kuriose galimas konkretaus
transkripcijos faktoriaus prisijungimas, gautos duomenų struktūros sujungiamos
į bendrą duomenų struktūrą, kuriai sukuriamas papildomas stulpelis,
specifikuojantis užklausos sekose nustatytų transkripcijos faktorių
prisijungimo procentą, palyginus transkripcijos faktorių taikinių skaičių
užklausos sekose su skaičiumi, gautu \emph{Blast} surastose sekose. 

\newpage

%%%%%%%%%%%%%%%%%%%%%%
% MĖGINIŲ APRAŠYMAS
%%%%%%%%%%%%%%%%%%%%%%

\section{Pasirinktų mėginių charakteristika}
Metodo patikimumui ir tikslumui įvertinti naudoti naminės pelės ir žmogaus
ChIP se\-kos\-kai\-tos duomenys, gauti iš nuolat atnaujinamo bei papildomo
ChIP-Atlas\cite{CHIPATLAS} serverio, saugančio ChIP bei kitų sekoskaitų
epigenetinių duomenų rinkinius bei leidžiančio vizualizuoti praturtintų sekų -
pikų - regionus\cite{CHIPATLAS2}.

Metodo testavimui naudoti 4 skirtingi naminių pelių (lot. \emph{Mus musculus})
bei 4 skirtingi žmogaus (lot. \emph{Homo sapiens}) mėginiai, išgauti iš
skirtingų tipų ląstelių: pliuripotentinių kamieninių ląstelių, gebančių
diferencijuoti į visų tipų ląsteles, nervinių bei kraujo ląstelių.

Iš ChIP-Atlas duomenų bazės atsisiųsti tik statistiškai patikimi ChIP
sekoskaitos duomenys, atrinkti įvertinus statistinio \emph{q} įverčio reikšmes.
Įvertinus genominių regionų \emph{q} vertes gautas mažesnis duomenų rinkinys,
kuris gali būti apdorojamas, nesukeliant programos veikimo trikdžių bei
nesumažinant rezultatų generavimo spartos. Statistiškai patikimi duomenys
atsisiųsti BED formato pavidalu.

ChIP sekoskaitos mėginiai, išgauti iš naminės pelės ląstelių, aprašyti pirmoje
lentelėje.

\begin{table}[htb]
    \newcolumntype{M}[1]{>{\centering\arraybackslash}m{#1}}
    \small
    \caption*{\small\textbf{1 lentelė. Naminės pelės mėginių charakteristikos}}
    \begin{tabular}{|c|c|c|c|c|}
        \hline
        \textbf{Ląstelių tipas} & \textbf{\thead{Kamienas}} &
        \textbf{\thead{Poveikis}} & \textbf{Antikūnai} &
        \textbf{\thead{ChIP-Atlas ID}} \\
        \hline
        \thead{Nervinės iš mESC} & - & Laukinis tipas (\emph{wt}) &
        \thead{CTCF} &
        \thead{\href{https://chip-atlas.org/view?id=SRX13476140}{SRX13476140}}\\ 
        \hline
        \thead{Endoderminės\\iš mESC} & DKI &
        \thead{Išgavimas po 3 dienų;\\r2} &
        \thead{anti-Foxa2} &
        \thead{\href{https://chip-atlas.org/view?id=SRX4298470}{SRX4298470}}\\
        \hline
        \thead{Endoderminės\\iš mESC} & DKI &
        \thead{Išgavimas po 5 dienų;\\endoderminės ląstelės} &
        \thead{anti-Gata4} &
        \thead{\href{https://chip-atlas.org/view?id=SRX4298474}{SRX4298474}}\\ 
        \hline
        \thead{Nervinės\\progenitorinės} & C57BL/6 x DBA &
        \thead{Ląstelių išgavimas\\po disociacijos} & \thead{anti-Sox2} &
        \thead{\href{https://chip-atlas.org/view?id=SRX2749159}{SRX2749159}}\\
        \hline
    \end{tabular}
\end{table}

\begin{itemize}
    \item \textbf{mESC} - pelių embrioninės kamieninės ląstelės
          (angl. \emph{\textbf{M}ouse \textbf{E}mbryonic \textbf{S}tem
          \textbf{C}ells}).
    \item \textbf{Endoderminės ląstelės} - ląstelių tipas, iš kurių 
          susiformuoja įvairių organų epitelinis audinys.
    \item \textbf{DKI} - pelių kamienas, kuriam būdingas dvigubas
          pasirinktų genų sekų įterpimas arba specifinių genų sekų pakeitimas
          kitų genų sekomis.
    \item \textbf{Progenitorinės ląstelės} - ląstelės, gebančios diferencijuoti
          į tam tikro tipo ląsteles (tuo jos panašios į kamienines ląsteles),
          tačiau jų specifiškumo lygis yra didesnis nei mažo specifiškumo
          kamieninių ląstelių.
    \item \textbf{C57BL/6 x DBA} - sukryžminti dažniausiai naudojami inbrydingo
          būdu gauti pelių kamienai.
\end{itemize}

\newpage

ChIP sekoskaitos mėginiai, išgauti iš žmogaus ląstelių, aprašyti antroje
lentelėje.

\begin{table}[htb]
    \newcolumntype{M}[1]{>{\centering\arraybackslash}m{#1}}
    \small
    \caption*{\small\textbf{2 lentelė. Žmogaus mėginių charakteristikos}}
    \begin{tabular}{|c|c|c|c|}
    \hline
    \textbf{Ląstelių tipas} &
    \textbf{\thead{Poveikis}} & \textbf{Antikūnai} &
    \textbf{\thead{ChIP-Atlas ID}} \\
    \hline
    \thead{Nervinės iš hESC} & \thead{Laukinis tipas (\emph{wt});\\KCl-} &
    \thead{CTCF} &
    \thead{\href{https://chip-atlas.org/view?id=SRX4417526}{SRX4417526}}\\ 
    \hline
    \thead{Endoderminės\\iš hESC} & \thead{DE} &
    \thead{FOXA2} &
    \thead{\href{https://chip-atlas.org/view?id=SRX11080722}{SRX11080722}}\\ 
    \hline
    \thead{Endoderminės\\iš hESC} & Replika 1 & \thead{GATA4} &
    \thead{\href{https://chip-atlas.org/view?id=SRX701989}{SRX701989}}\\ 
    \hline
    \thead{Nervinės\\progenitorinės} & Laukinis tipas (\emph{wt}) &
    \thead{anti-SOX2} &
    \thead{\href{https://chip-atlas.org/view?id=SRX5716451}{SRX5716451}}\\ 
    \hline
    \end{tabular}
\end{table}

\begin{itemize}
    \item \textbf{hESC} - žmogaus embrioninės kamieninės ląstelės
          (angl. \emph{\textbf{H}uman \textbf{E}mbryonic \textbf{S}tem
          \textbf{C}ells}).
    \item \textbf{KCl (+/-)} - ląstelių stimuliavimas arba nestimuliavimas kalio
          chlorido tirpalu.
    \item \textbf{DE} - galutinė endoderma (angl. \emph{\textbf{d}efinitive
          \textbf{e}ndoderm}).
\end{itemize}

\newpage

%%%%%%%%%%%%%%%%%%%
% TYRIMO METODAI
%%%%%%%%%%%%%%%%%%%

\section{Tyrimo metodai}
Transkripcijos faktorių taikinių spėjimo pasirinktame organizme metodas
įgyvendintas su R programavimo kalba\cite{R} (4.2.3 versija).

\subsection{Duomenų kokybės vertinimas}
Genominių duomenų kokybės vertinimui panaudotos Kursinio darbo bei Kursinio
projekto metu atliktos duomenų kokybės įvertinimo analizės. Taip pat
duomenų kokybės įvertinimo analizių sąrašas papildytas 3 naujomis analizėmis.

\subsubsection*{Pikų skaičius mėginiuose}
Pirmajame duomenų kokybės įvertinimo etape panaudota Kursiniame darbe
įgyvendinta analizė pikų skaičiaus nustatymui mėginiuose pritaikanti bazinę R
programavimo kalbos funkciją \emph{length()}. Pastaroji funkcija leidžia
apskaičiuoti, kiek kiekviename mėginyje yra pikų regionų.

\subsubsection*{Pikų skaičius chromosomose}
Atliekant šį duomenų kokybės įvertinimo etapą panaudota funkcija
\emph{facet\_wrap()}, kuri nustato, kiek pikų yra nustatyta skirtingose
chromosomose. Pikų skaičiaus pasiskirstymą skirtingose chromosomose
vizualizuojančios stulpelinės diagramos sukurtos su \emph{ggplot2}\cite{GGPLOT2}
bibliotekos funkcija \emph{geom\_bar()}.

\subsubsection*{Mėginių panašumas}
Mėginių tarpusavio panašumui įvertinti panaudota Kursinio darbo metu
realizuota modifikuota \emph{Jaccard()} funkcija. Mėginių panašumas
vizualizuotas, panaudojus spalvų intensyvumo grafiką - pritaikius
\emph{ggplot2} bibliotekos funkciją \emph{geom\_tile()}.

\subsubsection*{Genominė distribucija}
Šio Kursinio projekto metu realizuoto duomenų kokybės vertinimo rezultatas,
pikams priskiriant artimiausio geno pavadinimą (anotuojant piką), pateiktas
lentelės pavidalu. Šiame darbe pastarasis rezultato atvaizdavimas patobulintas:
kiekvienam mėginiui sukurtas grafikas, vaizduojantis kiekvieno genominio
elemento procentinę dalį. Pastaroji vizualizacija sukurta, pritaikius
\emph{ChIPseeker}\cite{CHIPSEEKER} bibliotekos funkciją \emph{plotAnnoBar()}.

\subsubsection*{Atstumas iki TSS}
Atstumo iki artimiausio transkripcijos pradžios taško
(angl. \emph{Transcription Start Site}) nustatymas įgyvendintas, pritaikius
\emph{ChIPseeker} bibliotekos funkciją \emph{annotatePeak()}, kuriai argumentų
pavidalu perduoti pikus aprašantys \emph{GRanges}\cite{GRANGES} objektai bei
visus žinomus konkretaus organizmo genus aprašantis \emph{TxDb}\cite{TXDB_MM}
objektas. Atlikus pikų anotavimą gauta genominių elementų procentines dalis
apibendrinanti lentelė, kuri perduota \emph{ChIPseeker} funkcijai
\emph{plotDistToTSS()}. Pastaroji funkcija šiame duomenų kokybės vertinimo
etape sukūrė grafiką, atvaizduojantį kiekvieno mėginio pikų atstumą iki
artimiausio \emph{TSS} regiono.

\subsubsection*{Pikų profilio atvaizdavimas}
Prieš skaičiuojant pikų, kurie jungiasi prie \emph{TSS} regionų, profilį yra
paruošiami tie \emph{TSS} regionai, kurie yra vadinami „šalia \emph{TSS}
einančiais regionais“ (angl. \emph{flanking sequences}). Šie regionai nustatyti
pritaikius \emph{ChIPseeker} bibliotekos funkciją \emph{getPromoters()}.
Pastarajai funkcijai perduotas konkretaus organizmo \emph{TxDb} objektas.
Nustačius šalia \emph{TSS} regionų esančių sekų regionus panaudota
\emph{getTagMatrix()} funkcija, kuri sukuria nuskaitymų, kurie patenka į šalia
\emph{TSS} sričių esančius sekų regionus, matricą. Gautą matricą perdavus
\emph{ChIPseeker} funkcijai \emph{plotAvgProf()} gautas pikų profilis.

\newpage

\subsection{Biologinių analizių atlikimas}
Šiame etape buvo panaudotos patobulintos Kursinio darbo metu įgyvendintos
analizės, kurias galima atlikti su pasirinktais genominiais duomenimis.

\subsubsection*{Transkripcijos faktoriaus motyvo logotipas}
Transkripcijos faktoriaus motyvo sekos logotipas sukurtas su R bibliotekos
\emph{ggseqlogo}\cite{GGSEQLOGO} funkcija \emph{ggseqlogo()}, kuriai perduota
transponuota konkretaus transkripcijos faktoriaus pozicinė svorių matrica.
Šios matricos atsisiųstos iš HOCOMOCO\cite{HOCOMOCO} duomenų bazės (11.0
versija), saugančios 680 žmogaus bei 453 naminės pelės transkripcijos faktorių
pozicines svorių matricas.

\subsubsection*{PWM matricos atitikimų skaičiavimas}
Kursinio darbo metu įgyvendintas pozicinės matricos atitikimų skaičiavimo
metodas patobulintas ir automatizuotas. ChIP sekoskaitos duomenys išsaugoti
\emph{GRanges} objektų pavidalu. Pastarieji objektai, saugantys genomines
sekoskaitos duomenų pozicijas, panaudoti šias pozicijas atitinkančių nukleotidų
sekų išgavimui. Nukleotidai, patenkantys į tam tikrus regionus, išgauti
panaudojus bibliotekos \emph{BSgenome}\cite{BSGENOME} funkciją \emph{getSeq()}.
Pastarajai funkcijai perduotas organizmo genomo (metodas testuotas su naminės
pelės genomu) anotacijos objektas
\emph{BSgenome.Mmusculus.UCSC.mm10}\cite{BSMUSMUSCULUS} bei pasirinktų mėginių
\emph{GRanges} objektas. Išgautų nukleotidų sekų \emph{DNAStringSet} objektas
perduotas \emph{Biostrings}\cite{BIOSTRINGS} bibliotekos funkcijai
\emph{countPWM()} kartu su tiriamo transkripcijos faktoriaus pozicine svorių
matrica. Pastaroji funkcija suskaičiavo, kiek išgautose nukleotidų sekose yra
transkripcijos faktoriaus motyvo atitikimų. Šis skaičius padalintas iš bendro
pikų skaičiaus - gauta transkripcijos faktoriaus motyvų procentinė dalis,
kuri vizualizuota, pritaikius pagrindinę \emph{ggplot2} bibliotekos funkciją
\emph{ggplot()} bei \emph{geom\_bar()}.

\subsubsection*{Praturtintų sekų biologinių funkcijų nustatymas}
Praturtintų sekų biologinės funkcijas apibendrinanti lentelė gauta, pritaikius R
bibliotekos \emph{clusterProfiler}\cite{CLUSTERPROFILER} funkciją
\emph{enrichGO()}. Šiai funkcijai perduoti anotuotų ChIP sekoskaitos duomenų
pikų \emph{Entrez} identifikacijos numeriai, organizmo, iš kurio buvo išgautas
pasirinktas mėginys, anotacijos objektas (veikimas tikrintas su naminės pelės
anotacijos objektu \emph{org.Mm.eg.db}), subontologijos (\emph{BP} -
biologiniai procesai, \emph{MF} - molekulinės funkcijos, \emph{CC} - ląstelės
komponentai) parametras bei įvairūs statistiniai įverčiai, reikalingi genų
ontologijos analizės atlikimui.

Biologinius procesus vizualizuojantis aciklinis kryptinis grafas sukonstruotas,
perdavus funkcijos \emph{enrichGO()} rezultato objektą ir bibliotekos
\emph{enrichplot}\cite{ENRICHPLOT} funkcijai \emph{goplot()}.

Funkcijos \emph{enrichGO()} rezultato objektas taip pat buvo panaudotas
atliekant hierarchinį klasterizavimą ir anotuotų pikų genus suskirsčius į
5 skirtingus klasterius, kurie vizualizuoti su \emph{enrichplot} bibliotekos
funkcija \emph{treeplot()}. Pastarosios funkcijos pritaikymui \emph{enrichGO()}
objektas perduotas \emph{pairwise\_termsim()} funkcijai, priklausančiai
\emph{enrichplot} bibliotekai. Šioje funkcijoje realizuotas Žakardo
(angl. \emph{Jaccard}) panašumo indekso tarp skirtingų biologinių procesų
skaičiavimas, leidžiantis suskirstyti biologinius procesus į klasterius.

\subsubsection*{Motyvų paieška \emph{De novo}}
Praturtintos sekos nustatytos, pritaikius komandinės eilutės įrankio
HOMER\cite{HOMER} \emph{Perl} programą \emph{findMotifsGenome}, analizuojančią
ChIP sekoskaitos mėginius, konvertuotus į BED formato failus. Taip pat taikant
šią programą nurodytas organizmo, iš kurio išgauti mėginiai, referentinio genomo
trumpinys (pavyzdžiui, naminės pelės referentinis genomas \emph{mm10}).
Algoritmo veikimo pabaigoje sugeneruotos lentelės, aprašančios informaciją
apie identifikuotus \emph{De novo} motyvus.

\newpage

\subsection{Interaktyvios aplikacijos kūrimas}
Programa, apdorojanti naudotojo įkeltus failus ir atliekanti biologines
analizes, sukurta su R programavimo kalbos biblioteka \emph{Shiny}\cite{SHINY}.
Pastarosios bibliotekos funkcijos leidžia sukurti interaktyvias internetines
aplikacijas (angl. \emph{Interactive Web App}), naudojant R bei internetinių
puslapių kūrimo kalbų - HTML, CSS, JavaScript - funkcijas.

\subsubsection*{Duomenų įkėlimas}
Naudotojas gali įkelti vieną arba daugiau failų, turinčių tabuliacijos
simboliais atskirtus stulpelius. Pastaruosiuose failuose turi būti pateikta
svarbiausia genominių ChIP sekoskaitos duomenų informacija - chromosomos
trumpinys bei piko pradžios ir pabaigos rėžiai. Įkėlus neteisingo formato
duomenis aplikacijoje prieinamos biologinės analizės nėra atliekamos.

Duomenų įkėlimo skiltyje specifikuojami papildomi parametrai, kurie panaudojami
atliekant duomenų kokybės vertinimą, duomenų biologines analizes bei konkretaus
transkripcijos faktoriaus taikinių spėjimą pasirinktame organizme. Programos
naudotojas turi nurodyti, iš kokio organizmo išgauti mėginiai. Tai reikalinga
tam, jog iš JSON\cite{JSON} formato failo būtų nuskaitytos atitinkamos R
bibliotekos bei chromosomų duomenys ir būtų sugeneruoti aiškūs grafikų 
pavadinimai. Taip pat naudotojas turi įrašyti transkripcijos faktoriaus
pavadinimą bei pateikti nurodyto transkripcijos faktoriaus pozicinę svorių
matricą.

\subsubsection*{Duomenų kokybės vertinimas}
Duomenų kokybės skiltyje pateikiama įkeltų ChIP sekoskaitos mėginių
lentelė, kurios elementus naudotojas gali pasirinkti bei vykdyti duomenų
kokybės vertinimo etapus tik pasirinktiems mėginiams. Atliekant GO analizę bei
\emph{De novo} motyvų paiešką aplikacija vienu metu gali analizuoti tik vieną
mėginį.

\subsubsection*{Biologinės analizės}
Biologinių analizių atlikimo lange naudotojas gali pasirinkti, su kokiais
įkeltais mėginiais nori atlikti transkripcijos faktorių apibūdinančios pozicinės
svorių matricos atitikimų skaičiavimą, atlikti \emph{De novo} motyvų paiešką
bei nustatyti, kokios biologinės funkcijos būdingos identifikuotiems motyvams
(GO analizė). \emph{De novo} motyvų paieškos įrankio - HOMER -
pritaikymui, panaudota bazinė R funkcija \emph{system()}, kuri leidžia
R programose įterpti komandas, kurios nepriklauso R bazinių ir įvairių
bibliotekų funkcijų rinkiniui, tačiau yra naudojamos komandinėje eilutėje
(taikant bioinformatinius komandinės eilutės įrankius). Gauti identifikuoti
\emph{De novo} motyvai pateikti lentelės pavidalu, kurią naudotojas gali
atsisiųsti ir naudoti. Komandinės eilutės įrankio HOMER veikimas yra
vienas iš ilgiau trunkančių šiame darbe aprašytų etapų, todėl galutinių
rezultatų pateikimas naudotojui, pateikusiam ir pasirinkusiam mėginius, gali
užtrukti dėl įrankio atliekamų motyvų paieškos skaičiavimų. 

\subsubsection*{Taikinių spėjimas}
Tuo atveju, jei norima vykdyti transkripcijos faktorių taikinių spėjimą,
naudotojas turi specifikuoti organizmą, kurio genome bus spėjamas transkripcijos
faktorių prisijungimas. Atlikus šį pasirinkimą pasirinktiems mėginiams
pritaikomas transkripcijos faktorių taikinių spėjimo metodas. Šio etapo metu
sugeneruojami tarpiniai rezultatų failai: anotuotų pikų genų sekų rinkiniai,
\emph{Blastp} rezultatai, pozicinės svorių matricos atitikimų sekose lentelės
bei pikų, kuriems priskirtų genų atitikimai surasti specifikuotame genome,
rinkiniai. Šiuos tarpinius rezultatus galima per\-žiū\-rė\-ti \emph{Shiny}
aplikacijoje bei juos atsisiųsti.

\subsubsection*{Aplikacijos pritaikymas skirtingiems organizmams}
R \emph{Shiny} aplikacija gali būti panaudota, analizuojant mėginius, išgautus
iš skirtingų or\-ga\-niz\-mų, tačiau šis sąrašas yra ribotas. Realizuojant
aplikaciją taikytos parašytos funkcijos, naudojančios R programavimo kalbos
\emph{BSgenome}, \emph{Txdb} bei \emph{org.x.eg.db} genominių anotacijų
bibliotekas. Dabartinėje R versijoje (4.2.3) yra prieinamos 33 skirtingų
organizmų \emph{BSgenome} bei 12 \emph{Txdb} ir \emph{org.x.eg.db} bibliotekos.
Sukurtoje aplikacijoje galima analizuoti 6 skirtingų organizmų ChIP sekoskaitos
duomenis, kurie anotuojami bei analizuojami, naudojant naujausias organizmų
genomų anotacijų versijas. Šių organizmų genominių anotacijų bibliotekos,
informacija apie kiekvieno organizmo chromosomų rinkinį bei chromosomų ilgius
(informacija apie chromosomas išgauta iš NCBI\cite{NCBI} duomenų bazės)
pateiktos JSON formato pavidalu. Šis formatas sukurtoje
R aplikacijoje apdorotas, pritaikius \emph{rjson}\cite{RJSON} bibliotekos
funkciją \emph{fromJSON()}.

\newpage

%%%%%%%%%%%%%%%%%%%%%%%%%%%%%%%
% REZULTATAI IR JŲ APTARIMAS
%%%%%%%%%%%%%%%%%%%%%%%%%%%%%%%

\section{Rezultatai ir jų aptarimas}
Realizuotas transkripcijos faktorių taikinių spėjimo metodas testuotas su 4
skirtingais \emph{Mus musculus} bei \emph{Homo sapiens} ChIP sekoskaitos
duomenų rinkiniais. Gauti rezultatai aprašyti detaliau.

\subsection{PWM matricos atitikimai}
Realizavus transkripcijos faktorių taikinių spėjimo metodą pradinių
(\emph{Mus musculus}) mėginių sekoms bei gautoms sekoms iš \emph{Homo sapiens}
genomo pritaikyta transkripcijos
faktoriaus pozicinė svorių matrica. Penktame paveiksle (5 pav.) pavaizduotose
stulpelinėse diagramose nurodyta, kokią procentinę dalį sudaro pradinių
mėginių sekose identifikuoti pozicinę svorių matricą atitinkantys fragmentai
su nustatytų atitikimų skaičiumi \emph{Blastp} paieškos metu gautose sekose.

\begin{figure}[H]
    \begin{center}
        \includegraphics[width=1\linewidth]{../Figures/Multiple_min_scores.png}
        \vspace{-1.5\baselineskip}
        \caption*{\small\textbf{5 pav. PWM atitikimų procentinių dalių
                                stulpelinės diagramos}}
    \end{center}
\end{figure}

Remiantis gautu rezultatu galima pastebėti, kad visuose tirtuose
\emph{Mus musculus} mėginiuose buvo nustatyti genai, kuriems buvo surasti didelį
užklausos sekų padengimą bei sekų identiškumą taikinio organizme turintys
\emph{Homo sapiens} genai. Ieškant transkripcijos faktoriaus pozicinę svorių
matricą atitinkančių sekų fragmentų abiejų organizmų genuose buvo nustatyta,
kad užklausos sekose (\emph{Mus musculus}) ir jas atitinkančiose
(\emph{Homo sapiens}) sekose buvo nustatytas didelis pozicinės svorių matricos
atitikimų skaičius: \(\sim\)85\%, kai \emph{min.score = 70\%}, \(\sim\)81\%,
kai \emph{min.score = 80\%} ir \(\sim\)56\%, kai \emph{min.score = 90\%}.

Didinant pozicinės svorių matricos atitikimų skaičiavimo parametrą - minimalų
transkripcijos faktoriaus motyvo ir sekos fragmento atitikimo procentą - 
atitikimų skaičius užklausos bei taikinio sekose sumažėjo, tačiau išskirtys
nepastebėtos.

Suskaičiavus, kiek pozicinės svorių matricos atitikimų yra užklausos bei
taikinio sekose, pasitaikė atvejų, kai pozicinės svorių matricos atitikimų
užklausos sekose nustatyta daugiau nei taikinio sekose. Šių atvejų procentinė
dalis kiekviename mėginyje pavaizduota šeštame paveiksle (6 pav.).

\begin{figure}[ht]
    \begin{center}
        \captionsetup{justification=centering}
        \includegraphics[width=1\linewidth]{../Figures/Less_PWM_hits_in_control.png}
        \vspace{-1.5\baselineskip}
        \caption*{\small\textbf{6 pav. Genų, turinčių mažesnį PWM atitikimų 
                                skaičių taikinio sekose, procentinės dalies
                                stulpelinė diagrama}}
    \end{center}
\end{figure}

Mėginiui \emph{Sample1}, išgautame iš naminės pelės audinių, buvo būdingas
didžiausias anotuotų pikų skaičius. Šiems pikams priskirtiems genams suradus
atitinkamas sekas žmogaus genome dalis žmogaus genų turėjo
mažesnį pozicinę svorių matricą atitinkančių sekų fragmentų skaičių nei
naminės pelės genų sekos - tokių genų sekų nustatyta 18\%. Mažiausias pozicinės
svorių matricos atitikimų skaičius taikinio sekose (16.41\%) nustatytas
\emph{Sample4} mėginyje, kuriam būdingas mažiausias pikų bei jiems priskirtų
genų skaičius.

\newpage

\subsection{Nustatytų genų sutapimai su kontroliniais mėginiais}
Anotavus kontrolinius žmogaus ChIP sekoskaitos duomenis - pikų rinkinius -
nustatyta, kokia anotuotų pikų - genų - dalis sutampa su filtruotu \emph{Blastp}
paieškos metu gautu genų rinkiniu. Gautas sutampančių pikų procentas
pavaizduotas spalvų intensyvumo grafike (7 paveikslas).

\begin{figure}[ht]
    \begin{center}
        \captionsetup{justification=centering}
        \includegraphics[width=0.7\linewidth]{../Figures/Genes_in_control.png}
        \vspace{-1.5\baselineskip}
        \caption*{\small\textbf{7 pav. Anotuotų kontrolinių mėginių pikų ir
                                \emph{Blastp} paieškos metu gautų sutampančių
                                genų spalvų intensyvumo grafikas}}
    \end{center}
\end{figure}

Remiantis gautu grafiku galima pastebėti, kad didžiausias sutampančių genų
skaičius nus\-ta\-ty\-tas tarp \emph{Mus musculus} mėginio \emph{Sample2} ir
\emph{Homo sapiens} mėginio \emph{SRX11080722.05.bed}. Pas\-ta\-ra\-sis žmogaus
ChIP sekoskaitos mėginys turėjo didžiausią anotuotų pikų skaičių, todėl 
buvo nustatyta daugiausiai genų sutapimų su atitinkamomis naminės pelės genų
sekomis. Kituose kontroliniuose žmogaus ir naminės pelės mėginiuose sutampančių
genų procentinė dalis buvo mažesnė dėl mažėjančio kontrolinių mėginių pikų
skaičiaus.

\colorbox{green}{PASTABA: mėginių pavadinimai grafikuose yra koreguojami.}

\newpage

%%%%%%%%%%%%
% IŠVADOS
%%%%%%%%%%%%

\section{Išvados}
Sukūrus transkripcijos faktorių taikinių spėjimo metodą bei realizavus jo
implementaciją R \emph{Shiny} aplinkoje gauti rezultatai buvo apibendrinti:

\begin{itemize}
    \item Sukurtas transkripcijos faktorių taikinių spėjimo metodas leidžia
        nuspėti, kokiuose genominiuose pasirinkto organizmo regionuose
        galimas transkripcijos faktoriaus prisijungimas, naudojant kito
        organizmo genominius duomenis;
    \item Priklausomai nuo organizmų genomų panašumo realizuotas metodas
        leidžia įvertinti transkripcijos faktoriaus pozicinės svorių matricos
        atitikimą su dideliu tikslumu;
    \item Realizuotas metodas gali būti panaudotas, spėjant pasirinkto
        transkripcijos faktoriaus taikinius skirtinguose organizmuose;
    \item Sukurta R \emph{Shiny} aplikacija, kurioje galima įvertinti ChIP
        sekoskaitos duomenų kokybę, atlikti biologines analizes bei pritaikyti
        transkripcijos faktorių taikinių spėjimo metodą grafinėje aplinkoje.
\end{itemize}

\newpage

%%%%%%%%%%%%%%%
% LITERATŪRA
%%%%%%%%%%%%%%%

\bibliographystyle{plain}
\begin{thebibliography}{99}

\bibitem{ARTICLE0} Lee TI, Young RA. Transcriptional regulation and its
misregulation in disease. Cell. 2013 Mar 14;152(6):1237-51.
doi: 10.1016/j.cell.2013.02.014. PMID: 23498934; PMCID: PMC3640494.

\bibitem{BIOTOOLS} Ison, J. et al. (2015). Tools and data services registry: a
community effort to document bioinformatics resources. Nucleic Acids Research.
doi: 10.1093/nar/gkv1116.

\bibitem{RNASEQ} Love MI, Anders S, Kim V, Huber W. RNA-Seqworkflow:
gene-level exploratory analysis and differential expression. F1000Res.
2015 Oct 14;4:1070. doi: 10.12688/f1000research.7035.1. PMID: 26674615;
PMCID: PMC4670015.

\bibitem{GTRD} GTRD: an integrated view of transcription regulation.
Kolmykov S, Yevshin I, Kulyashov M, Sharipov R, Kondrakhin Y, Makeev VJ,
Kulakovskiy IV, Kel A, Kolpakov F Nucleic Acids Res. 2021 Jan
8;49(D1):D104-D111.

\bibitem{CHIPATLAS} Zou, Z., Ohta, T., Miura, F., Oki, S. ChIP-Atlas 2021
update: a data-mining suite for exploring epigenomic landscapes by fully
integrating ChIP-seq, ATAC-seq and Bisulfite-seq data. Nucleic Acids Res.
50(W1), W175-W182, 2022, \newline
\url{http://dx.doi.org/10.1093/nar/gkac199}.

\bibitem{CHIPATLAS2} Zhaonan Zou, Tazro Ohta, Fumihito Miura, Shinya Oki,
ChIP-Atlas 2021 update: a data-mining suite for exploring epigenomic landscapes
by fully integrating ChIP-seq, ATAC-seq and Bisulfite-seq data, Nucleic Acids
Research, Volume 50, Issue W1, 5 July 2022, Pages W175–W182,
\url{https://doi.org/10.1093/nar/gkac199}.

\bibitem{ARTICLE1} Nakato R, Sakata T. Methods for ChIP-seq analysis: A
practical workflow and advanced applications. Methods. 2021 Mar;187:44-53.
doi: 10.1016/j.ymeth.2020.03.005. Epub 2020 Mar 30. PMID: 32240773.

\bibitem{ARTICLE2}
CD Genomics (2023). \emph{ChIP-Seq}.
Prieiga per \url{https://www.cd-genomics.com/chip-seq.html}
[žiūrėta 2023-05-21].

\bibitem{SONICATION}
GoldBio (2023). \emph{How to break up DNA for NGS library prep}.
Prieiga per \url{https://goldbio.com/articles/article/how-to-fragment-DNA-for-NGS}.
[žiūrėta 2023-05-21].

\bibitem{ARTICLE3} Shah, A. Chromatin immunoprecipitation sequencing (ChIP-Seq)
on the SOLiD™ system. Nat Methods 6, ii–iii (2009).
\url{https://doi.org/10.1038/nmeth.f.247}.

\bibitem{ARTICLE4} Liu, E.T., Pott, S. \& Huss, M. Q\&A: ChIP-seq technologies
and the study of gene regulation. BMC Biol 8, 56 (2010).
\url{https://doi.org/10.1186/1741-7007-8-56}.

\bibitem{ARTICLE5}
CD Genomics (2023). \emph{The Advantages and Workflow of ChIP-seq}.
Prieiga per \url{https://www.cd-genomics.com/the-advantages-and-workflow-of-chip-seq.html}.
[žiūrėta 2023-05-21].

\bibitem{ARTICLE6} Langmead, B., Trapnell, C., Pop, M. et al. Ultrafast and
memory-efficient alignment of short DNA sequences to the human genome. Genome
Biol 10, R25 (2009), \newline
\url{https://doi.org/10.1186/gb-2009-10-3-r25}.

\bibitem{ARTICLE7} Li H, Durbin R. Fast and accurate short read alignment with
Burrows-Wheeler transform. Bioinformatics. 2009 Jul 15;25(14):1754-60. doi: 10.
1093/bioinformatics/btp324. Epub 2009 May 18. PMID: 19451168; PMCID: PMC2705234.

\bibitem{ARTICLE9} Lee TI, Young RA. Transcriptional regulation and its
misregulation in disease. Cell. 2013 Mar 14;152(6):1237-51.
doi: 10.1016/j.cell.2013.02.014. PMID: 23498934; PMCID: PMC3640494.

\bibitem{ASP1}
Myriad Genetics (2023). \emph{Autoimmune Polyglandular Syndrome Type 1}.
Prieiga per \url{https://myriad.com/womens-health/diseases/autoimmune-polyglandular-syndrome-type-1/}.
[žiūrėta 2023-05-21].

\bibitem{ARTICLE10} Lambert SA, Jolma A, Campitelli LF, Das PK, Yin Y, Albu M,
Chen X, Taipale J, Hughes TR, Weirauch MT. The Human Transcription Factors.
Cell. 2018 Feb 8;172(4):650-665. doi: 10.1016/j.cell.2018.01.029.
Erratum in: Cell. 2018 Oct 4;175(2):598-599. PMID: 29425488.

\bibitem{ARTICLE11} Mundade R, Ozer HG, Wei H, Prabhu L, Lu T. Role of ChIP-seq
in the discovery of transcription factor binding sites, differential gene
regulation mechanism, epigenetic marks and beyond. Cell Cycle.
2014;13(18):2847-52. doi: 10.4161/15384101.2014.949201. PMID: 25486472;
PMCID: PMC4614920.

\bibitem{ARTICLE12} Zhang, Y., Liu, T., Meyer, C.A. et al. Model-based Analysis
of ChIP-Seq (MACS). Genome Biol 9, R137 (2008).
\url{https://doi.org/10.1186/gb-2008-9-9-r137}.

\bibitem{ARTICLE13}
Merck (2023). \emph{PEAK CALLING FOR ChIP-SEQ}
Prieiga per \url{https://epigenie.com/wp-content/uploads/2013/02/Peak-Calling-for-ChIP-Seq.pdf}.
[žiūrėta 2023-05-21].

\bibitem{ARTICLE14} Guo Y, Papachristoudis G, Altshuler RC, Gerber GK, Jaakkola
TS, Gifford DK, Mahony S. Discovering homotypic binding events at high spatial
resolution. Bioinformatics. 2010 Dec 15;26(24):3028-34.
doi: 10.1093/bioinformatics/btq590. Epub 2010 Oct 21. PMID: 20966006;
PMCID: PMC2995123.

\bibitem{ARTICLE15} Guo Y, Mahony S, Gifford DK. High resolution genome wide
binding event finding and motif discovery reveals transcription factor spatial
binding constraints. PLoS Comput Biol. 2012;8(8):e1002638. doi: 10.1371/journal.
pcbi.1002638. Epub 2012 Aug 9. PMID: 22912568; PMCID: PMC3415389.

\bibitem{ENSEMBLDB} Rainer J, Gatto L, Weichenberger CX (2019) ensembldb: an R
package to create and use Ensembl-based annotation resources. Bioinformatics.
doi:10.1093/bioinformatics/btz031.

\bibitem{ARTICLE16} Ji H, Jiang H, Ma W, Wong WH. Using CisGenome to analyze
ChIP-chip and ChIP-seq data. Curr Protoc Bioinformatics. 2011 Mar;
Chapter 2:Unit2.13. doi: 10.1002/0471250953.bi0213s33. PMID: 21400695;
PMCID: PMC3072298.

\bibitem{ARTICLE17} Enrique Blanco, Luciano Di Croce, Sergi Aranda. Comparative
ChIP-seq (Comp-ChIP-seq): a novel computational methodology for genome-wide
analysis. bioRxiv. 2019 January 29. doi: 10.1101/532622. \newline
\url{https://www.biorxiv.org/content/early/2019/03/26/532622}.

\bibitem{R} R Core Team (2022). R: A language and environment for statistical
computing. R Foundation for Statistical Computing, Vienna, Austria. URL
\url{https://www.R-project.org/}.

\bibitem{GGPLOT2} H. Wickham. ggplot2: Elegant Graphics for Data Analysis.
Springer-Verlag New York, 2016.

\bibitem{SHINY} Chang W, Cheng J, Allaire J, Sievert C, Schloerke B, Xie Y,
Allen J, McPherson J, Dipert A, Borges B (2023). shiny: Web Application
Framework for R. R package version 1.7.4.9002, \url{https://shiny.rstudio.com/}.

\bibitem{CHIPSEEKER} Wang Q, Li M, Wu T, Zhan L, Li L, Chen M, Xie W, Xie Z,
Hu E, Xu S, Yu G (2022). “Exploring epigenomic datasets by ChIPseeker.”
Current Protocols, 2(10), e585. doi: 10.1002/cpz1.585.

\bibitem{GRANGES} Lawrence M, Huber W, Pag\`es H, Aboyoun P, Carlson M,
et al. (2013) Software for Computing and Annotating Genomic Ranges. PLoS Comput
Biol 9(8): e1003118. doi:10.1371/journal.pcbi.1003118.

\bibitem{TXDB_MM} Team BC, Maintainer BP (2019). 
TxDb.Mmusculus.UCSC.mm10.knownGene: Annotation package for TxDb object(s).
R package version 3.10.0.

\bibitem{GGSEQLOGO} Wagih O (2017). ggseqlogo: A 'ggplot2' Extension for
Drawing Publication-Ready Sequence Logos. R package version 0.1, \newline
\url{https://CRAN.R-project.org/package=ggseqlogo}.

\bibitem{BSGENOME} Pagès H (2023). BSgenome: Software infrastructure for
efficient representation of full genomes and their SNPs. R package version
1.66.3, \newline
\url{https://bioconductor.org/packages/BSgenome}.

\bibitem{BSMUSMUSCULUS} Team TBD (2021). BSgenome.Mmusculus.UCSC.mm10: Full
genome sequences for Mus musculus (UCSC version mm10, based on GRCm38.p6). R
package version 1.4.3.

\bibitem{BIOSTRINGS} Pagès H, Aboyoun P, Gentleman R, DebRoy S (2022).
Biostrings: Efficient manipulation of biological strings. R package version
2.66.0, \newline
\url{https://bioconductor.org/packages/Biostrings}.

\bibitem{CLUSTERPROFILER} T Wu, E Hu, S Xu, M Chen, P Guo, Z Dai, T Feng,
L Zhou, W Tang, L Zhan, X Fu, S Liu, X Bo, and G Yu. clusterProfiler 4.0:
A universal enrichment tool for interpreting omics data. The Innovation. 2021,
2(3):100141.

\bibitem{ENRICHPLOT} Yu G (2023). enrichplot: Visualization of Functional
Enrichment Result. R package version 1.18.4, \newline
\url{https://yulab-smu.top/biomedical-knowledge-mining-book}.

\bibitem{JSON} Pezoa, F. et al., 2016. Foundations of JSON schema. In
Proceedings of the 25th International Conference on World Wide Web. pp. 263–273.

\bibitem{RJSON} Couture-Beil A (2022). rjson: JSON for R. R package version
0.2.21, \newline
\url{https://CRAN.R-project.org/package=rjson}.

\bibitem{NCBI} National Center for Biotechnology Information (NCBI)[Internet].
Bethesda (MD): National Library of Medicine (US), National Center for
Biotechnology Information; [1988] – [cited 2023 May 19]. Available from:
\url{https://www.ncbi.nlm.nih.gov/}.

\bibitem{RBLAST} Hahsler M, Nagar A (2019). rBLAST: R Interface for the Basic
Local Alignment Search Tool. R package version 0.99.2, \newline
\url{https://github.com/mhahsler/rBLAST}.

\bibitem{BLAST} Camacho C., Coulouris G., Avagyan V., Ma N., Papadopoulos J.,
Bealer K., Madden T.L. (2008) “BLAST+: architecture and applications.”
BMC Bioinformatics 10:421. PubMed.

\bibitem{HOMER} Heinz S, Benner C, Spann N, Bertolino E et al. Simple
Combinations of Lineage-Determining Transcription Factors Prime cis-Regulatory
Elements Required for Macrophage and B Cell Identities. Mol Cell 2010 May
28;38(4):576-589. PMID: 20513432.

\bibitem{HOCOMOCO} Ivan V. Kulakovskiy; Ilya E. Vorontsov; Ivan S. Yevshin;
Ruslan N. Sharipov; Alla D. Fedorova; Eugene I. Rumynskiy; Yulia A. Medvedeva;
Arturo Magana-Mora; Vladimir B. Bajic; Dmitry A. Papatsenko; Fedor A. Kolpakov;
Vsevolod J. Makeev. Nucl. Acids Res., Database issue, gkx1106
(11 November 2017), doi: 10.1093/nar/gkx1106.

\end{thebibliography}
\newpage

%%%%%%%%%%%%
% PRIEDAI
%%%%%%%%%%%%

\section{Priedas} \label{Priedas}
Priedų sąraše pateikiamos Baigiamojo darbo, Kursinio darbo bei Kursinio projekto
Git repozitorijų nuorodos. Taip pat pateiktos OneDrive nuorodos į
darbo metu analizuotus pavyzdinius ChIP sekoskaitos mėginius bei atrinktų
transkripcijos faktorių pozicines svorių matricas.

\begin{itemize}
    \item \textbf{Baigiamojo darbo Git repozitorija:}\\
        \url{https://github.com/dansta0804/TF\_analysis}
    \item \textbf{Pavyzdinių duomenų aplankas OneDrive:}\\
        \url{https://vult-my.sharepoint.com/:f:/g/personal/daniele_stasiunaite_mif_stud_vu_lt/EtEGQ8POkapLhPv6eHvl48cB-jmes81M0JPW8PVWTz2QgA?e=wjSSKJ}
    \item \textbf{Kursinio projekto analizės Git repozitorija:}\\
        \url{https://github.com/dansta0804/Tbx5\_analysis\_II.git}
    \item \textbf{Kursinio darbo analizės Git repozitorija:}\\
        \url{https://github.com/dansta0804/Tbx5\_analysis.git}
\end{itemize}

\end{document}