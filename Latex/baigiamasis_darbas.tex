\documentclass[12pt]{article}
\usepackage{indentfirst}
\usepackage[utf8x]{inputenc}
\usepackage[T1]{fontenc}
\usepackage[english,lithuanian]{babel}
\usepackage{array}
\usepackage{caption}
\usepackage{subcaption}
\usepackage{makecell}
\usepackage[euler]{textgreek}
\usepackage{multirow}
\usepackage{boldline}
\usepackage{floatrow}
\floatsetup[table]{capposition=top}
\usepackage{amsmath, amsthm, amssymb}
\usepackage{graphicx}
\usepackage{setspace}
\usepackage{verbatim}
\usepackage[left=3cm,top=2cm,right=1.5cm,bottom=2cm]{geometry}
\usepackage{floatrow}
\newfloatcommand{capbtabbox}{table}[][\FBwidth]
\usepackage{blindtext}
\onehalfspacing
\usepackage[hidelinks, unicode]{hyperref}
\usepackage{textcomp}
\usepackage{amsmath}

\newcommand{\EE}{\mathbb{E}\,} % Mean
\newcommand{\ee}{{\mathrm e}}  % nice exponent
\newcommand{\dd}{{\mathrm d}}
\newcommand{\RR}{\mathbb{R}}

\begin{document}
\selectlanguage{lithuanian}

\begin{titlepage}
\vskip 20pt
\begin{center}
\includegraphics[scale=0.5]{MIF}
\end{center}

%%%%%%%%%%%%%%%%%%%%%%%
% TITULINIS PUSLAPIS
%%%%%%%%%%%%%%%%%%%%%%%

\vskip 20pt
\centerline{\bf \large \textbf{VILNIAUS UNIVERSITETAS}}
\bigskip
\centerline{\large \textbf{MATEMATIKOS IR INFORMATIKOS FAKULTETAS}}
\bigskip
\centerline{\large \textbf{BIOINFORMATIKOS BAKALAURO STUDIJŲ PROGRAMA}}

\vskip 90pt
\begin{center}
    {\bf \LARGE Pavadinimas lietuviškai}
\end{center}
\begin{center}
    {\bf \Large Pavadinimas angliškai}
\end{center}
\vskip 20pt
\centerline{\bf \large \textbf{Bakalauro baigiamasis darbas}}
\bigskip
\vskip 40pt

\hskip 140pt {\large Autorė: Danielė Stasiūnaitė}

\hskip 140pt{\large VU el. p.: daniele.stasiunaite@mif.stud.vu.lt}
\bigskip
\vskip 20pt

\hskip 140pt {\large Darbo vadovė: J. m. d. Kotryna Kvederavičiūtė}
\vskip 60pt
\vskip 40pt
\centerline{\large \textbf{Vilnius}}
\centerline{\large \textbf{2023}}
\newpage
\end{titlepage}

\selectlanguage{lithuanian}

%%%%%%%%%%%%%%%%%%%%%
% TURINIO PUSLAPIS
%%%%%%%%%%%%%%%%%%%%%

\tableofcontents
\newpage

%%%%%%%%%%%%%%%%%%%%%%%%%%%%%%%%%%%%
% LIETUVIŠKOS SANTRAUKOS PUSLAPIS
%%%%%%%%%%%%%%%%%%%%%%%%%%%%%%%%%%%%

\section*{Santrauka}
\colorbox{green}{PILDOMA...}

\hfill \break
\textbf{Raktiniai žodžiai:} R, \emph{Shiny}, \emph{Homo sapiens},
\emph{Mus musculus}, ChIP sekoskaita

\newpage

%%%%%%%%%%%%%%%%%%%%%%%%%%%%%%%%%%
% ANGLIŠKOS SANTRAUKOS PUSLAPIS
%%%%%%%%%%%%%%%%%%%%%%%%%%%%%%%%%%

\section*{Summary}
\colorbox{green}{PILDOMA...}

\hfill \break
\textbf{Keywords:} R, \emph{Shiny}, \emph{Homo sapiens},
\emph{Mus musculus}, ChIP-seq

\newpage

%%%%%%%%%%%%%%%%%%%
% ĮVADO PUSLAPIS
%%%%%%%%%%%%%%%%%%%

\section{Įvadas}
\subsection*{Darbo temos aktualumas}
\colorbox{green}{PILDOMA...}

\subsection*{Darbo tikslas}
Sukurti algoritmą bei jo implementaciją R \emph{Shiny}, kurie leistų nuspėti
transkripcijos faktorių prisijungimo vietas, remiantis sekų paašumu.

\subsection*{Uždaviniai}
\begin{itemize}
    \item Išanalizuoti metodus, taikomus praturtintų nuskaitymų - pikų -
          nustatymui;
    \item Įgyvendinti transkripcijos faktorių taikinių spėjimo metodą;
    \item Modifikuoti ir adaptuoti sukurtas R programas, galinčias apdoroti
          ChIP sekoskaitos duomenis iš skirtingų organizmų;          
    \item Sukurti internetinę aplikaciją, leidžiančią įvertinti pateiktų
          ChIP sekoskaitos duomenų kokybę bei atlikti analizes.

\end{itemize}

\newpage

%%%%%%%%%%%%%%%%%%
% TEORINĖ DALIS
%%%%%%%%%%%%%%%%%%

\section{ChIP sekoskaita ir jos vykdymo ypatumai}

\textbf{DNR sekoskaita} - deoksiribonukleorūgšties nukleotidų sekos nustatymas.
Nukleotidų sekų nustatymui gali būti naudojami du pagrindiniai DNR sekvenavimo 
metodai:

\begin{itemize}
    \item \emph{Sendžerio} metodas;
    \item Didelio našumo arba naujos kartos sekoskaita
    (angl. \emph{NGS - \textbf{N}ext \textbf{G}eneration \textbf{S}equencing}).
\end{itemize}

Šiuolaikiniuose tyrimuose labai dažnai taikomos modernios DNR sekos - naujos 
kartos se\-kos\-kai\-tos - technologijos, kurios leidžia nuskaityti didelį kiekį
DNR arba RNR sekų daug sparčiau ir pigiau nei klasikinė \emph{Sendžerio} 
sekoskaita. Pastaroji sekoskaita dažniau naudojama, tiriant mažus duomenų
rinkinius.

\subsection{Transkripcijos faktoriai ir jų prasmė}

\textbf{Transkripcijos faktorius} - ypatingas baltymų tipas. Šio tipo baltymai
atpažįsta specifines DNR sekas ir tokiu būdu kontroliuoja chromatino struktūros
kondensacijos laipsnį bei atitinkamų genų ekspresijos procesus, inicijuojant
arba slopinant genų transkripciją. Šių baltymų sąveika su DNR įtakoja daug
biologiškai svarbių procesų: ląstelių diferenciaciją, ląstelės ciklo eigą, genų
transkripciją, DNR replikaciją, imuninio atsako valdymą ir daugelį kitų
procesų\cite{ARTICLE10, ARTICLE11}.

Transkripcijos faktoriai bei sekos, prie kurių jie jungiasi, gali
mutuoti. Šios pastarųjų baltymų mutacijos nulemia įvairių ligų išsivystymą.
Pavyzdžiui, atsiradusios mutacijos AIRE (autoimuniniame reguliatoriuje)
transkripcijos faktoriuje sukelia I tipo autoimuninį poliendokrinopatijos
sindromą\cite{ARTICLE9} (angl. \emph{APS1 - \textbf{A}utoimmune
\textbf{P}olyendocrinopathy \textbf{S}yndrome type \textbf{I}}). Pasireiškus
šiam sindromui organizmo imuninės ląstelės naikina sveikas, hormonus
išskiriančių liaukų ląsteles\cite{ASP1}, todėl transkripcijos faktorių veikimo
mechanizmų supratimas yra itin svarbus, siekiant diagnozuoti bei išgydyti ligas,
kurios gali būti susijusios su transkripcijos faktorių baltymų mutacijomis.

\subsection{ChIP sekoskaitos apibūdinimas}
\textbf{ChIP sekoskaita} - chromatino imunoprecipitacijos sekoskaita (angl.
\emph{chromatin immunoprecipitation sequencing}). Tai yra viena iš svarbiausių
technologijų, atliekant epigenetikos tyrimus\cite{ARTICLE1}. Šiam metodui
būdingas klasikinio eksperimentinio chromatino precipitacijos metodo derinimas
su naujos kartos (angl. \emph{next generation (NGS) sequencing}) sekoskaita,
siekiant išsiaiškinti baltymų sąveikas su DNR ir nustatyti, kaip transkripcijos
faktoriai ir kiti, su chromatinu susiję baltymai, gali įtakoti įvairius fenotipo
pokyčių mechanizmus\cite{ARTICLE2}.

ChIP sekoskaita neretai taikoma, atliekant: histonų modifikavimo, genų
reguliacijos, transkripcijos komplekso surinkimo, DNR pažaidų taisymo,
vystymosi mechanizmų bei ligų progresavimo tyrimus.

% KABUTĖS: „ “

\subsection{ChIP sekoskaitos vykdymo eiga}

DNR ir baltymų sąveikos tyrimo - ChIP sekoskaitos - eiga pateikta pirmame
paveiksle:

\begin{figure}[ht]
    \begin{center}
        \includegraphics[width=0.7\linewidth]{../Figures/ChIP-seq_workflow.jpg}
        \vspace{-1\baselineskip}
        \caption*{\small\textbf{1 pav. ChIP sekoskaitos vykdymo etapai}}
    \end{center}
\end{figure}

Remiantis pateiktu paveikslu ChIP sekoskaita gali būti suskirstyta į du
pagrindinius etapus:

\begin{enumerate}
    \item Mėginių paruošimas ir sekvenavimas;
    \item Kompiuterinė analizė.
\end{enumerate}

\subsubsection{Mėginių paruošimas ir sekvenavimas}
Mėginių paruošimo bei sekvenavimo etapai įprastai skirstomi į šiuos etapus:

\begin{itemize}
    \item \textbf{Baltymo prijungimas prie DNR.} Šiame etape transkripcijos faktorius susiejamas su DNR, naudojant įvairias chemines medžiagas (formaldehidas naudojamas dažniausiai). Ši baltymo ir DNR fiksacija,
        naudojant chemines medžiagas, padeda išlaikyti baltymo - DNR
        sąsają. Šios sąsajos išsaugojimas yra būtina sąlyga imunoprecipitacijos
        proceso vykdymui.
    \item \textbf{DNR suskaidymas į fragmentus.} \emph{NGS} bibliotekos 
        paruošimui reikalingas DNR suskaidymo į fragmentus etapas. Šiame etape
        DNR dažniausiai suskaidoma į 150 - 500 nukleotidus turinčius fragmentus,
        naudojant ultragarso bangas - sonifikacijos mechanizmą. Pastarojo
        mechanizmo sukurtas vibracijos rezonansas suskaido DNR į fragmentus.
        Fragmentų ilgis gali būti kontroliuojamas sonifikacijos įrenginio
        naudojimo dažniu. Pavyzdžiui, kuo ilgiau trunka vienas ultra garso
        panaudojimo ciklas, tuo trumpesni DNR fragmentai
        gaunami\cite{SONICATION}.
    \item \textbf{Imunoprecipitacijos procesas.} Suskaidytos DNR fragmentai
        inkubuojami su specifiniu antikūnu, galinčiu atpažinti prie DNR
        prisijungusį baltymą - transkripcijos faktorių. Tam, jog ChIP
        sekoskaitos rezultatai būtų patikimi ir tinkami, tinkamas antikūno
        parinkimas ir jo kokybės užtikrinimas yra vienas iš svarbiausių ChIP
        sekoskaitos mėginių paruošimo etapų\cite{ARTICLE3}. Testuojant
        skirtingus antikūnus pasirenkamas tas, kurį panaudojus gaunamas
        didesnis DNR sekų, prie kurių prisijungęs transkripcijos faktorius,
        praturtinimas nei praturtinimas, kuris gautas, naudojant nespecifinį
        antikūną\cite{ARTICLE4} (pavyzdžiui, naudojant tipinį imunoglobulino G
        (IgG) antikūną).
    \item \textbf{Sekvenavimas.} Neretai sekvenavimo įrenginių pritaikymui
        reikalingi trumpų adapterių prijungimo prie gautų DNR fragmentų ir PGR
        amplifikacijos etapai - reikalingas bibliotekos sukonstravimas, kuris
        gali skirtis, priklausomai nuo pasirinktos sekvenavimo platformos
        ir jai specifinės bibliotekos paruošimo protokolo\cite{ARTICLE5}.
        Įvykdžius šiuos etapus gali būti gautas \colorbox{red}{„triukšmas“}\cite{ARTICLE4}
        (angl. \emph{bias}), kuris gali būti mažesnis, atliekant mažiau DNR
        amplifikacijos (padauginimo) ciklų. Sukonstravus biblioteką atliekamas
        \emph{NGS} sekvenavimas.
\end{itemize}

Įgyvendinus mėginių paruošimo ir sekvenavimo etapus atliekama gautų duomenų
kompiuterinė analizė.

\subsubsection{Kompiuterinė analizė}
Gauti ChIP sekoskaitos duomenys apdorojami ir analizuojami, vykdant šiuos
etapus:

\begin{itemize}
    \item \textbf{DNR nuskaitymų kartografavimas.} Nusekvenuoti DNR 
        fragmentai išsaugomi \emph{FASTQ} arba \emph{CSFSATQ} formatais. Šie DNR
        nuskaitymai (angl. \emph{reads} arba \emph{tags}) perkeliami ant genomo,
        naudojant kartografavimo (angl. \emph{mapping}) įrankius, pavyzdžiui,
        \emph{Bowtie}\cite{ARTICLE6}, \emph{Burrows-Wheeler}, kurie leidžia
        nustatyti nuskaityto DNR fragmento poziciją genome, esant kelių
        nukleotidų neatitikimui\cite{ARTICLE7} (angl. \emph{mismatch}). Atlikus
        DNR nuskaitymų priskyrimą gaunami \emph{SAM}, \emph{BAM} (dažniausiai
        naudojamas formatas), \emph{CRAM} arba \emph{tagAlign} formato failai.
    \item \textbf{Normalizavimas.} Normalizavimas reikalingas, siekiant
        sumažinti techninį nuskaitymų variabilumą - sumažinti „triukšmą“,
        atsiradusį dėl sekvenavimo gylio skirtumų tarp mėginių. Normalizavimas
        atliekamas konkrečiose genomo pozicijose esančių nuskaitymų skaičių
        padalinus iš bendro nuskaitymų skaičiaus\cite{ARTICLE17}.
    \item \textbf{Pikų nustatymas (angl. \emph{peak calling}).} Šiame etape
        nustatomi reikšmingai praturtinti genomo lokusai - pikai
        (angl. \emph{peaks}). Įgyvendinus šį etapą dažniausiai sugeneruojami
        \emph{BED} formato failai\cite{ARTICLE1}, kuriuose pateikiamos genominės
        pikų pozicijos, įvairūs statistiniai įverčiai bei identifikacijos kodai,
        kuriuos naudojant galima vykdyti įvairias analizes.
    \item \textbf{Biologinės analizės.} Dažniausiai atliekamos analizės yra
        motyvų analizė bei genų ontologijos analizė\cite{ARTICLE1}
        (angl. \emph{GO - \textbf{G}ene \textbf{O}ntology enrichment analysis}),
        pateikianti biologinių procesų, ląstelinių komponentų ir molekulinių
        funkcijų, kuriose dalyvauja genas, sąrašą. Įvertinus motyvų dažnį,
        konservatyvumą bei biologines funkcijas, susijusias su transkripcijos
        faktorių prisijungimu prie šių motyvų, gali būti identifikuoti
        genominiai regionai, prie kurių gali jungtis transkripcijos faktoriai,
        konservatyvūs motyvai, galintys indikuoti baltymas - baltymas sąveiką,
        bei gali būti analizuojama genų evoliucija.
\end{itemize}

Įvykdžius kompiuterinės analizės etapus gaunami rezultatai, suteikiantys
įžvalgų apie genų reguliavimo mechanizmus bei padedantys nustatyti
transkripcijos faktorių taikinius, siekiant sustabdyti kai kurių ligų
progresavimą.

% Paaiškinti, kas yra ChIP sekoskaita ir koks jos tikslas, kokie jos privalumai.
% Kokie egzistuoja metodai, leidžiantys įvertinti TF taikinius kitame organizme?
% Kas yra transkripcijos faktorius?
% Ką daro TF?
% Kuo yra svarbūs TF?
% Kas yra ChIP sekoskaita?
% Kokiais metodais gali būti nustatyti pikai?
% Kokie yra šių metodų privalumai ir trūkumai?
% Kodėl naudojamas MACS2?

\newpage

\subsection{Pikų nustatymo algoritmai}
Pikų nustatymas yra vienas iš svarbiausių etapų, atliekant DNR ir reguliatorinių
baltymų - transkripcijos faktorių arba histonų - sąveikos tyrimų analizes.
Kuriant pikų nustatymo algoritmus sprendžiamos dvi pagrindinės problemos:
regionų, kurie, tikėtina, yra pikai, nustatymas bei tikėtinų pikų statistinio
reikšmingumo tikrinimas.

Pagrindinė pikų nustatymo algoritmų įvestis yra kartografavimo metu su genomu
išlyginti DNR fragmentų nuskaitymai. Antrajame paveiksle šie fragmentai pažymėti
raudona ir žalia spalvomis.



% \includegraphics[scale=.30]{example-image-b}

% \caption{A caption with long text bla bla bla bla bla bla bla bla
% bla bla bla bla bla bla bla bla bla bla}
% \end{figure}


\begin{figure}[ht]
    \begin{center}
        \captionsetup{justification=centering}
        \includegraphics[width=0.6\linewidth]{../Figures/Read_mapping.png}
        \vspace{-1\baselineskip}
        \caption*{\small\textbf{2 pav. DNR nuskaitymų kartografavimas. Žalia
                                spalva pavaizduoti 5' \(\rightarrow\) 3' DNR
                                galo link kartografuoti nuskaitymai. Raudona
                                spalva - 3' \(\rightarrow\) 5' nuskaitymai.}}
    \end{center}
\end{figure}

Nustačius DNR fragmentų nuskaitymų pozicijas genome kai kurios pastebimos
nuskaitymų sankaupų grupės gali indikuoti, jog toje pozicijoje yra galimas
transkripcijos faktoriaus prisijungimas (nuskaitymų sankaupa yra reikšminga),
tačiau neretai tokios sankaupos - pikai - gali būti laikomos molekuliniu arba
eksperimentiniu „triukšmu“. Taigi kuriant pikų nustatymo algoritmus yra
svarbu, jog algoritmas gebėtų įvertinti, ar pikas yra biologiškai reikšmingas,
ar tai tėra „triukšmas“.

Yra sukurta daugiau nei 30 skirtingų pikų nustatymo algoritmų (angl.
\emph{peak caller}), kurie sprendžia anksčiau minėtas problemas,
tačiau šių problemų sprendimo būdai yra skirtingi. Konkretaus algoritmo
pasirinkimas labai priklauso nuo atliekamo eksperimento tipo ir specialisto,
atliekančio analizę, patirties\cite{ARTICLE13}.

\subsubsection{MACS2}
\textbf{MACS2} - \textbf{M}odel-based \textbf{A}nalysis of
\textbf{C}hIP-\textbf{S}eq. Tai yra populiariausias ir bene seniausias
pikų nustatymo algoritmas.
Transkripcijos faktorių jungimosi prie DNR saitai nustatomi, atsižvelgus į
nuskaitymų pozicijas bei kryptį. Neretai konkrečioje genomo pozicijoje gali
būti priskirti keli nuskaitymai. MACS2 metode yra numatyta palikti tik po
vieną nuskaitymą konkrečioje pozicijoje (pašalinti duplikatus). Duplikatai
nėra šalinami, jeigu tikimasi, kad transkripcijos faktorius jungsis keliose
skirtingose genomo pozicijose. Tose genomo pozicijose, kuriose, tikėtina,
jungiasi transkripcijos faktorius, turi būti pastebimas \emph{Watsono} ir
\emph{Kriko} nuskaitymų išsidėstymas arba \textbf{bimodalinis pasiskirtymas},
kurio grafikas pavaizduotas paveiksle:

\begin{figure}[ht]
    \begin{center}
        \includegraphics[width=0.5\linewidth]{../Figures/Bimodal_pattern.png}
        \vspace{-1\baselineskip}
        \caption*{\small\textbf{3 pav. Bimodalinis pasiskirstymas}}
    \end{center}
\end{figure}

Tam, jog panaši pikų struktūra būtų surasta, MACS2 algoritmas skanuoja visą
kartografuotų nuskaitymų duomenų rinkinį. Algoritmas naudoja dydį, kuris nurodo,
į kokio ilgio nukleotidų fragmentus buvo skaidoma DNR sonifikacijos proceso
metu (angl. \emph{bandwidth}), bei \emph{mfold} vertę. Vykdant algoritmą
atliekamas \emph{2 * bandwidth} ilgio poslinkis ir ieškoma tokių genomo
pozicijų, kuriose nuskaitymų yra daugiau nei naudojant atsitiktinį nuskaitymų
rinkinį (daugiau už \emph{mfold} vertę).

Nustačius aukštos kokybės pikus yra atsitiktinai parenkama 1000 pikų. Turint
šiuos pikus yra atskiriami jų \emph{Watsono} ir \emph{Kriko} (teigiamos ir
neigiamos grandinės) nuskaitymai.
Šių teigiamų ir neigiamų grandinių pikų grupės yra išlyginamos pagal jų
centrus, kaip pavaizduota 4 paveiksle. Atstumas tarp išlygintų nuskaitymų
modų (\emph{d}) nurodo, kokio ilgio yra piko fragmentas.

\begin{figure}[H]
    \begin{center}
        \includegraphics[width=0.5\linewidth]{../Figures/Tag_alignment.png}
        % \vspace{-1\baselineskip}
        \caption*{\small\textbf{4 pav. Nuskaitymų išlyginimas}}
        \label{fig:birds}
    \end{center}
\end{figure}

Algoritmas visiems nuskaitymams atlieka \emph{d/2} 3' DNR galo link
tikėtiniausio DNR ir transk\-rip\-ci\-jos faktoriaus sąveikos regiono poslinkį.
Atlikus šį poslinkį atliekamas \emph{2 * d} poslinkis, jog būtų surastas
statistiškai reikšmingas nuskaitymų praturtinimas, naudojant \emph{Puasono}
skirstinį, kurio parametras \(\lambda_{BG}\) yra tikėtinas nuskaitymų skaičius
atlikus poslinkį. Nepaisant to, \(\lambda_{BG}\) parametras naudojamas,
neatsižvelgus į galimą „triukšmą“, kuris galėjo kilti dėl chromatino struktūros,
DNR amplifikacijos arba sekvenavimo, todėl yra naudojamas parametras
\(\lambda_{local}\), kuris skaičiuojamas kiekvienam tikėtinam
pikui:

\begin{equation} \label{lambda_local}
    \lambda_{local} = max(\lambda_{BG}, \lambda_{5k}, \lambda_{10k})
\end{equation}

čia \(5k\), \(10k\) yra poslinkio plotis.

Parametro \(\lambda_{local}\) naudojimas leidžia aptikti \emph{false positive},
pikus (pikus, kurie atsirado dėl „triukšmo“) ir nustatyti tik tuos pikus,
kurie indikuoja svarbų DNR ir baltymo sąveikos regioną\cite{ARTICLE12}.
  
\subsubsection{GEM}
\textbf{GEM} - \textbf{G}enome wide \textbf{E}vent finding and \textbf{M}otif
discovery. Šis 2012 metais sukurtas algoritmas išsiskiria tuo, jog jame yra
kombinuojama pikų paieška bei motyvų analizė, jog būtų pagerinta galutinių pikų
rezoliucija\footnote{\textbf{Rezoliucija - } genetikoje aukšta rezoliucija
reiškia, jog yra žinoma itin daug molekulinių detalių apie DNR.}.\\

GEM algoritmą sudaro šeši skirtingi etapai\cite{ARTICLE15}:
\begin{enumerate}
    \item \textbf{Baltymo ir DNR sąveikos regionų nustatymas.} Pradiniai
          regionai nustatomi, taikant \emph{GPS} algoritmą\cite{ARTICLE14},
          kuris naudoja \emph{Dirichlė} skirstinį.
    \item \textbf{Praturtintų \emph{k - merų} nustatymas.} Jie nustatomi,
          lyginant \emph{k - merų} dažnius tarp teigiamų sekų ir neigiamų
          kontrolinių sekų. Teigiamos sekos - sekos, kurios sudarytos iš 61
          bazių poros ir yra išsidėsčiusios spėjamų baltymo ir DNR sąveikos
          regionų (gautų pirmajame etape) centruose. Neigiamos kontrolinės
          sekos - 61 bazių porą turinčios sekos, kurios yra nutolusios nuo
          teigiamų sekų per 300 bazių porų. Be to, šios sekos nepersidengia su
          sekomis, esančiomis baltymo - DNR sąveikos sekų centruose. Šiame
          etape yra skaičiuojami \emph{k - merų} fragmentų atitikimai teigiamų
          ir neigiamų sekų rinkiniuose. \emph{K - meras} (sekos fragmentas)
          yra laikomas praturtintu, kai \emph{p} vertė yra mažesnė nei 0.001.
          % ir 3-fold enrichment in terms of positive/negative hit count?
    \item \textbf{Praturtintų \emph{k - merų} klasterizavimas.} Praturtinti
          \emph{k - merai} klasterizuojami į ekvivalentiškumo klases, kurios
          apibūdina panašias DNR sekas, prie kurių jungiasi transkripcijos
          faktorius. Seka atitinka \emph{k - mero} ekvivalentiškumo klasę, kai
          sekoje nustatomas fragmentas yra vienas iš ekvivalentiškumo klasės
          elementų.
    \item \textbf{Išankstinio pasiskirstymo nustatymas.} Labiausiai praturtinta
          \emph{k - merų} klasė yra naudojama \emph{Dirichlė} išankstinio
          pasiskirstymo paskaičiavimui. Šiame etape genomas yra suskaidomas į
          kelis tūkstančius bazių porų turinčius segmentus. Šie segmentai yra
          gaunami atskiriant DNR fragmentus, kuriuose yra daugiau nei 500 bazių
          porų turintys tarpai bei DNR regionai, kuriems buvo priskirta mažiau
          nei 6 DNR nuskaitymai (angl. \emph{reads}). Šie regionai yra
          skanuojami su DNR sekų fragmentais, kurie priklauso atrinktai
          \emph{k - merų} ekvivalentiškumo klasei, \emph{k - merų}
          atitikimai yra skaičiuojami.
    \item \textbf{Tikslesnių baltymo - DNR sąveikos regionų spėjimas.} Tam yra
          panaudojamas 4 etape gautas išankstinis pozicijų pasiskirstymas.
          \colorbox{green}{PAPILDYTI...}
    \item \textbf{2 - 3 etapų kartojimas.} Tam yra panaudojami 5 etape gauti
          patikslinti baltymo - DNR sąveikos regionai.
\end{enumerate}

% \subsubsection{CisGenome}
% Taikant šį pikų nustatymo įrankį DNR nuskaitymai yra kartografuojami. Baltymo -
% DNR sąveikos regionai yra identifikuojami kaip tos genomo sritys, kurioms
% būdingas itin didelis nuskaitymų skaičius. Priklausomai nuo to, ar kontroliniai
% mėginiai yra sekvenuojami, taikant šį įrankį pikai gali būti nustatomi dvejopai.

% Naudojant \emph{CisGenome} įrankį vykdomi šie etapai\cite{ARTICLE16}:
% \begin{enumerate}
%     \item \textbf{\emph{FDR} skaičiavimas.} Pasirinkto organizmo genomas yra
%           suskaidomas į nepersidegiančius pasirinkto dydžio fragmentus - langus
%           (angl. \emph{windows}). Atlikus genomo suskaidymą yra suskaičiuojama,
%           kiek kiekviename fragmente (lange) yra kartografuotų sekų nuskaitymų.
%           \emph{FDR} reikšmė gaunama spėjamą \(k\) nuskaitymų lange proporciją 
%           padalinus iš stebimos \(k\) nuskaitymų lange proporcijos, remiantis
%           neigiamu binominiu modeliu. Šiame etape nustatoma, kokiam
%           kartografuotų nuskaitymų skaičiui būdinga mažesnė nei 0.1 \emph{FDR}
%           reikšmė.
%     \item \textbf{Pikų nustatymas.} Jeigu du pikus skiria mažesnis bazių porų
%           skaičius nei maksimalus jų skaičius, šie du pikai yra sujungiami į
%           vieną piką. Taip pat pikai, kurie yra trumpesni nei nurodytas
%           minimalus piko ilgis, tada tokie pikai nėra įtraukiami į bendrą pikų
%           sąrašą.
% \end{enumerate}

\newpage

%%%%%%%%%%%%%%%%%%%%%%%
% METODO REALIZACIJA
%%%%%%%%%%%%%%%%%%%%%%%
\section{Transkripcijos faktoriaus taikinių spėjimo metodas}

\newpage

%%%%%%%%%%%%%%%%%%%%%%
% MĖGINIŲ APRAŠYMAS
%%%%%%%%%%%%%%%%%%%%%%

\section{Pasirinktų mėginių charakteristika}
Metodo patikimumui ir tikslumui įvertinti naudoti naminės pelės ir žmogaus
ChIP sekoskaitos duomenys, gauti iš nuolat atnaujinamo bei papildomo
ChIP-Atlas\cite{CHIPATLAS} serverio, saugančio ChIP bei kitų sekoskaitų
epigenetinių duomenų rinkinius bei leidžiančio vizualizuoti praturtintų sekų -
pikų - regionus\cite{CHIPATLAS2}.

Metodo testavimui naudoti 8 skirtingi naminių pelių (lot. \emph{Mus musculus})
bei 8 skirtingi žmogaus (lot. \emph{Homo sapiens}) mėginiai, išgauti iš
skirtingų tipų ląstelių: pliuripotentinių kamieninių ląstelių, gebančių
diferencijuoti į visų tipų ląsteles, nervinių bei kraujo ląstelių.

ChIP sekoskaitos mėginiai, išgauti iš naminės pelės ląstelių, aprašyti pirmoje
lentelėje.

\begin{table}[htb]
    \newcolumntype{M}[1]{>{\centering\arraybackslash}m{#1}}
    \small
    \caption*{\small\textbf{1 lentelė. Naminės pelės mėginių charakteristikos}}
    \begin{tabular}{|c|c|c|c|c|}
        \hline
        \textbf{Ląstelių tipas} & \textbf{\thead{Kamienas}} &
        \textbf{\thead{Poveikis}} & \textbf{Antikūnai} &
        \textbf{\thead{ChIP-Atlas ID}} \\
        \hline
        \thead{Nervinės iš mESC} & - & Laukinis tipas (\emph{wt}) &
        \thead{CTCF} &
        \thead{\href{https://chip-atlas.org/view?id=SRX13476140}{SRX13476140}}\\ 
        \hline
        \thead{Nervinės iš mESC} & - & Maz -/- &
        \thead{CTCF} &
        \thead{\href{https://chip-atlas.org/view?id=SRX13476141}{SRX13476141}}\\ 
        \hline
        \thead{Endoderminės\\iš mESC} & DKI &
        \thead{Išgavimas po 3 dienų;\\r1} &
        \thead{anti-Foxa2} &
        \thead{\href{https://chip-atlas.org/view?id=SRX4298469}{SRX4298469}}\\ 
        \hline
        \thead{Endoderminės\\iš mESC} & DKI &
        \thead{Išgavimas po 3 dienų;\\r2} &
        \thead{anti-Foxa2} &
        \thead{\href{https://chip-atlas.org/view?id=SRX4298470}{SRX4298470}}\\ 
        \hline
        \thead{Endoderminės\\iš mESC} & DKI & 
        \thead{Išgavimas po 3 dienų;\\mezendoderminės ląstelės} &
        \thead{anti-Gata4} &
        \thead{\href{https://chip-atlas.org/view?id=SRX4298473}{SRX4298473}}\\ 
        \hline
        \thead{Endoderminės\\iš mESC} & DKI &
        \thead{Išgavimas po 5 dienų;\\endoderminės ląstelės} &
        \thead{anti-Gata4} &
        \thead{\href{https://chip-atlas.org/view?id=SRX4298474}{SRX4298474}}\\ 
        \hline
        \thead{Nervinės\\progenitorinės} & C57BL/6 x DBA  &
        \thead{Tiesioginis ląstelių\\išgavimas} & \thead{anti-Sox2} &
        \thead{\href{https://chip-atlas.org/view?id=SRX2378798}{SRX2378798}}\\ 
        \hline
        \thead{Nervinės\\progenitorinės} & C57BL/6 x DBA &
        \thead{Ląstelių išgavimas\\po disociacijos} & \thead{anti-Sox2} &
        \thead{\href{https://chip-atlas.org/view?id=SRX2749159}{SRX2749159}}\\ 
        \hline
    \end{tabular}
\end{table}

\begin{itemize}
    \item \textbf{mESC} - pelių embrioninės kamieninės ląstelės
          (angl. \emph{\textbf{M}ouse \textbf{E}mbryonic \textbf{S}tem
          \textbf{C}ells}).
    \item \textbf{Maz -/-} - specialus baltymą koduojantis genas, kuris
          nustatomas kartu su CTCF transk\-rip\-ci\-jos faktoriumi,
          kontroliuojant kohezino poziciją ir genomo organizaciją. -/- nurodo,
          kad šio geno kopijos nėra organizme.
    \item \textbf{Mezendoderminės ląstelės} - embrionų ląstelės, kurios
          diferencijuoja į mezodermos ir galutines endodermines ląsteles.
    \item \textbf{Endoderminės ląstelės} - ląstelių tipas, iš kurių 
          susiformuoja įvairių organų epitelinis audinys.
    \item \textbf{DKI} - pelių kamienas, kuriam būdingas dvigubas
          pasirinktų genų sekų įterpimas arba specifinių genų sekų pakeitimas
          kitų genų sekomis.
    \item \textbf{Progenitorinės ląstelės} - ląstelės, gebančios diferencijuoti
          į tam tikro tipo ląsteles (tuo jos panašios į kamienines ląsteles),
          tačiau jų specifiškumo lygis yra didesnis nei mažo specifiškumo
          kamieninių ląstelių.
    \item \textbf{C57BL/6 x DBA} - sukryžminti dažniausiai naudojami inbrydingo
          būdu gauti pelių kamienai.
\end{itemize}

ChIP sekoskaitos mėginiai, išgauti iš žmogaus ląstelių, aprašyti antroje
lentelėje.

\begin{table}[htb]
    \newcolumntype{M}[1]{>{\centering\arraybackslash}m{#1}}
    \small
    \caption*{\small\textbf{2 lentelė. Žmogaus mėginių charakteristikos}}
    \begin{tabular}{|c|c|c|c|}
    \hline
    \textbf{Ląstelių tipas} &
    \textbf{\thead{Poveikis}} & \textbf{Antikūnai} &
    \textbf{\thead{ChIP-Atlas ID}} \\
    \hline
    \thead{Nervinės iš hESC} & \thead{Laukinis tipas (\emph{wt});\\KCl-} &
    \thead{CTCF} &
    \thead{\href{https://chip-atlas.org/view?id=SRX4417526}{SRX4417526}}\\ 
    \hline
    \thead{Nervinės iš hESC} & \thead{Laukinis tipas (\emph{wt});\\KCl+} &
    \thead{CTCF} &
    \thead{\href{https://chip-atlas.org/view?id=SRX4417527}{SRX4417527}}\\ 
    \hline
    \thead{Endoderminės\\iš hESC} & \thead{DE} &
    \thead{FOXA2} &
    \thead{\href{https://chip-atlas.org/view?id=SRX11080722}{SRX11080722}}\\ 
    \hline
    \thead{Endoderminės\\iš hESC} & TKO & \thead{FOXA2} &
    \thead{\href{https://chip-atlas.org/view?id=SRX11080727}{SRX11080727}}\\ 
    \hline
    \thead{Endoderminės\\iš hESC} & Replika 1 & \thead{GATA4} &
    \thead{\href{https://chip-atlas.org/view?id=SRX701989}{SRX701989}}\\ 
    \hline
    \thead{Endoderminės\\iš hESC} & Replika 2 & \thead{GATA4} &
    \thead{\href{https://chip-atlas.org/view?id=SRX701990}{SRX701990}}\\ 
    \hline
    \thead{Nervinės\\progenitorinės} & Laukinis tipas (\emph{wt}) &
    \thead{anti-SOX2} &
    \thead{\href{https://chip-atlas.org/view?id=SRX5716451}{SRX5716451}}\\ 
    \hline
    \thead{Nervinės\\progenitorinės} & K755R/+ & \thead{anti-SOX2} &
    \thead{\href{https://chip-atlas.org/view?id=SRX5716452}{SRX5716452}}\\ 
    \hline
    \end{tabular}
\end{table}

\begin{itemize}
    \item \textbf{hESC} - žmogaus embrioninės kamieninės ląstelės
          (angl. \emph{\textbf{H}uman \textbf{E}mbryonic \textbf{S}tem
          \textbf{C}ells}).
    \item \textbf{KCl (+/-)} - ląstelių stimuliavimas arba nestimuliavimas kalio
          chlorido tirpalu.
    \item \textbf{DE} - galutinė endoderma (angl. \emph{\textbf{d}efinitive
          \textbf{e}ndoderm});
    \item \textbf{TKO} - trijų genų panaikinimas arba jų išaktyvinimas
          (angl. \emph{\textbf{t}riple \textbf{k}nock-\textbf{o}ut}).
    \item \textbf{K755R} - mutacijos nomenklatūra. 755 geno pozicijoje esanti
          lizino (K) aminorūgštis pakeista argininu (R).
\end{itemize}

\newpage

%%%%%%%%%%%%%%%%%%%
% TYRIMO METODAI
%%%%%%%%%%%%%%%%%%%

\section{Tyrimo metodai}
Transkripcijos faktorių taikinių spėjimo pasirinktame organizme metodas
įgyvendintas su R programavimo kalba\cite{R} (4.2.3 versija).

\subsection{Duomenų kokybės įvertinimas}
Genominių duomenų kokybės įvertinimui panaudotos Kursinio darbo bei Kursinio
projekto metu atliktos duomenų kokybės įvertinimo analizės. Taip pat
duomenų kokybės įvertinimo analizių sąrašas papildytas 3 naujomis analizėmis.

\subsubsection*{Pikų skaičius mėginiuose}
Pirmajame duomenų kokybės įvertinimo etape panaudota Kursiniame darbe
įgyvendinta analizė pikų skaičiaus nustatymui mėginiuose pritaikanti bazinę R
programavimo kalbos funkciją \emph{length()}. Pastaroji funkcija leidžia
apskaičiuoti, kiek kiekviename mėginyje yra pikų regionų.

\subsubsection*{Pikų skaičius chromosomose}
Atliekant šį duomenų kokybės įvertinimo etapą panaudota funkcija
\emph{facet\_wrap()}, kuri nustato, kiek pikų yra nustatyta skirtingose
chromosomose. Pikų skaičiaus pasiskirstymą skirtingose chromosomose
vizualizuojančios stulpelinės diagramos sukurtos su \emph{ggplot2}\cite{GGPLOT2}
bibliotekos funkcija \emph{geom\_bar()}.

\subsubsection*{Mėginių panašumas}
Mėginių tarpusavio panašumui įvertinti panaudota Kursinio darbo metu
realizuota modifikuota \emph{Jaccard()} funkcija. Mėginių panašumas
vizualizuotas, panaudojus spalvų intensyvumo grafiką - pritaikius
\emph{ggplot2} bibliotekos funkciją \emph{geom\_tile()}.

\subsubsection*{Genominė distribucija}
Šio Kursinio projekto metu realizuoto duomenų kokybės vertinimo rezultatas,
pikams priskiriant artimiausio geno pavadinimą (anotuojant piką), pateiktas
lentelės pavidalu. Šiame darbe pastarasis rezultato atvaizdavimas patobulintas:
kiekvienam mėginiui sukurtas grafikas, vaizduojantis kiekvieno genominio
elemento procentinę dalį. Pastaroji vizualizacija sukurta, pritaikius
\emph{ChIPseeker}\cite{CHIPSEEKER} bibliotekos funkciją \emph{plotAnnoBar()}.

\subsubsection*{Atstumas iki TSS}
Atstumo iki artimiausio transkripcijos pradžios taško
(angl. \emph{Transcription Start Site}) nustatymas įgyvendintas, pritaikius
\emph{ChIPseeker} bibliotekos funkciją \emph{annotatePeak()}, kuriai argumentų
pavidalu perduoti pikus aprašantys \emph{GRanges}\cite{GRANGES} objektai bei
visus žinomus konkretaus organizmo genus aprašantis \emph{TxDb}\cite{TXDB_MM}
objektas. Atlikus pikų anotavimą gauta genominių elementų procentines dalis
apibendrinanti lentelė, kuri perduota \emph{ChIPseeker} funkcijai
\emph{plotDistToTSS()}. Pastaroji funkcija šiame duomenų kokybės vertinimo
etape sukūrė grafiką, atvaizduojantį kiekvieno mėginio pikų atstumą iki
artimiausio \emph{TSS} regiono.

\subsubsection*{Pikų profilio atvaizdavimas}
Prieš skaičiuojant pikų, kurie jungiasi prie \emph{TSS} regionų, profilį yra
paruošiami tie \emph{TSS} regionai, kurie yra vadinami „šalia \emph{TSS}
einančiais regionais“ (angl. \emph{flanking sequences}). Šie regionai nustatyti
pritaikius \emph{ChIPseeker} bibliotekos funkciją \emph{getPromoters()}.
Pastarajai funkcijai perduotas konkretaus organizmo \emph{TxDb} objektas.
Nustačius šalia \emph{TSS} regionų esančių sekų regionus panaudota
\emph{getTagMatrix()} funkcija, kuri sukuria nuskaitymų, kurie patenka į šalia
\emph{TSS} sričių esančius sekų regionus, matricą. Gautą matricą perdavus
\emph{ChIPseeker} funkcijai \emph{plotAvgProf()} gautas pikų profilis.

\newpage

\subsection{Duomenų analizės atlikimas}
Šiame etape buvo panaudotos patobulintos Kursinio darbo metu įgyvendintos
analizės, kurias galima atlikti su pasirinktais genominiais duomenimis.

\subsubsection*{Transkripcijos faktoriaus motyvo logotipas}
Transkripcijos faktoriaus motyvo sekos logotipas sukurtas su R bibliotekos
\emph{ggseqlogo}\cite{GGSEQLOGO} funkcija \emph{ggseqlogo()}, kuriai perduota
transponuota konkretaus transkripcijos faktoriaus pozicinė svorių matrica.
Šios matricos atsisiųstos iš HOCOMOCO\cite{HOCOMOCO} duomenų bazės (11.0
versija), saugančios 680 žmogaus bei 453 naminės pelės transkripcijos faktorių
pozicines svorių matricas.

\subsubsection*{PWM matricos atitikimų skaičiavimas}
Kursinio darbo metu įgyvendintas pozicinės matricos atitikimų skaičiavimo
metodas patobulintas ir automatizuotas. ChIP sekoskaitos duomenys išsaugoti
\emph{GRanges} objektų pavidalu. Pastarieji objektai, saugantys genomines
sekoskaitos duomenų pozicijas, panaudoti šias pozicijas atitinkančių nukleotidų
sekų išgavimui. Nukleotidai, patenkantys į tam tikrus regionus, išgauti
panaudojus bibliotekos \emph{BSgenome}\cite{BSGENOME} funkciją \emph{getSeq()}.
Pastarajai funkcijai perduotas organizmo genomo (metodas testuotas su naminės
pelės genomu) anotacijos objektas
\emph{BSgenome.Mmusculus.UCSC.mm10}\cite{BSMUSMUSCULUS} bei pasirinktų mėginių
\emph{GRanges} objektas. Išgautų nukleotidų sekų \emph{DNAStringSet} objektas
perduotas \emph{Biostrings}\cite{BIOSTRINGS} bibliotekos funkcijai
\emph{countPWM()} kartu su tiriamo transkripcijos faktoriaus pozicine svorių
matrica. Pastaroji funkcija suskaičiavo, kiek išgautose nukleotidų sekose yra
transkripcijos faktoriaus motyvo atitikimų. Šis skaičius padalintas iš bendro
pikų skaičiaus - gauta transkripcijos faktoriaus motyvų procentinė dalis,
kuri vizualizuota, pritaikius pagrindinę \emph{ggplot2} bibliotekos funkciją
\emph{ggplot()} bei \emph{geom\_bar()}.

\subsubsection*{Praturtintų sekų biologinių funkcijų nustatymas}
\colorbox{green}{PILDOMA...}

\subsubsection*{Motyvų paieška \emph{De novo}}
Praturtintos sekos nustatytos, pritaikius komandinės eilutės įrankio
HOMER\cite{HOMER} \emph{Perl} programą \emph{findMotifsGenome}, analizuojančią
ChIP sekoskaitos mėginius, konvertuotus į BED formato failus. Taip pat taikant
šią programą nurodytas organizmo, iš kurio išgauti mėginiai, referentinio genomo
trumpinys (pavyzdžiui, naminės pelės referentinis genomas \emph{mm10}).

\newpage

\subsection{Interaktyvios aplikacijos kūrimas}
Programa, apdorojanti naudotojo įvestus \emph{BED} formato failus ir atliekanti
papildomas analizes, sukurta su R programavimo kalbos biblioteka
\emph{Shiny}\cite{SHINY}. Pastarosios bibliotekos funkcijos leidžia sukurti
interaktyvias internetines aplikacijas (angl. \emph{Interactive Web App}),
naudojant R bei internetinių puslapių kūrimo kalbų - HTML, CSS, JavaScript -
funkcijas.

\subsubsection*{Duomenų įkėlimas}
Naudotojas gali įkelti vieną arba daugiau \emph{BED} formato failų, saugančių
genominę ChIP sekoskaitos mėginių informaciją. Duomenų įkėlimo skiltyje
specifikuojami papildomi parametrai, kurie panaudojami atliekant duomenų kokybės
vertinimą, duomenų analizes bei konkretaus transkripcijos faktoriaus taikinių
spėjimą pasirinktame organizme. Programos naudotojas gali nurodyti
transkripcijos faktoriaus pavadinimą, pateikti nurodyto transkripcijos
faktoriaus pozicinę svorių matricą bei pasirinkti, kokio organizmo genome
norima atlikti transkripcijos faktoriaus taikinių spėjimą.

\colorbox{green}{PAPILDYTI APRAŠYMU APIE FAILŲ FORMATŲ KONVERSIJOS ETAPĄ...}

\subsubsection*{Duomenų kokybė}
Duomenų kokybės skiltyje pateikiama įkeltų ChIP sekoskaitos mėginių duomenų
lentelė, kurios elementus naudotojas gali pasirinkti bei vykdyti duomenų
kokybės vertinimo etapus tik pasirinktiems mėginiams. Taip pat naudotojas
gali apjungti pasirinktus arba visus įkeltus mėginius į vieną duomenų rinkinį
(šis funkcionalumas įgyvendintas, pritaikius bazinę R funkciją \emph{union()})
arba analizuoti tik pasirinktų arba pažymėtų mėginių persidengiančius pikus
(įgyvendinta, panaudojus funkciją \emph{intersect()}).

\subsubsection*{Analizės}
Analizių atlikimo lange naudotojas gali pasirinkti, su kokiais įkeltais
mėginiais nori atlikti transkripcijos faktorių apibūdinančios pozicinės
svorių matricos atitikimų skaičiavimą, atlikti motyvų paiešką \emph{De novo}
bei nustatyti, kokios biologinės funkcijos būdingos identifikuotiems motyvams
(GO analizė). \emph{De novo} motyvų paieškos įrankio - \emph{HOMER} -
pritaikymui, panaudota bazinė R funkcija \emph{system()}, kuri leidžia
R programose įterpti komandas, kurios nepriklauso R bazinių ir įvairių
bibliotekų funkcijų rinkiniui, tačiau yra naudojamos komandinėje eilutėje
(taikant bioinformatinius komandinės eilutės įrankius). Gauti identifikuoti
\emph{De novo} motyvai pateikti lentelės pavidalu, kurią naudotojas gali
atsisiųsti ir naudoti. Komandinės eilutės įrankio \emph{HOMER} veikimas yra
vienas iš ilgiau trunkančių šiame darbe aprašytų etapų, todėl galutinių
rezultatų pateikimas naudotojui, pateikusiam ir pasirinkusiam mėginius, gali
užtrukti dėl įrankio atliekamų motyvų paieškos skaičiavimų. 

\subsubsection*{Taikinių spėjimas}
\colorbox{green}{PILDOMA...}

\newpage

%%%%%%%%%%%%%%%%%%%%%%%%%%%%%%%
% REZULTATAI IR JŲ APTARIMAS
%%%%%%%%%%%%%%%%%%%%%%%%%%%%%%%

\section{Rezultatai ir jų aptarimas}
\colorbox{green}{PILDOMA...}

\newpage

%%%%%%%%%%%%
% IŠVADOS
%%%%%%%%%%%%

\section{Išvados}
\newpage

%%%%%%%%%%%%%%%
% LITERATŪRA
%%%%%%%%%%%%%%%

\bibliographystyle{plain}
\begin{thebibliography}{99}

\bibitem{GTRD} GTRD: an integrated view of transcription regulation.
Kolmykov S, Yevshin I, Kulyashov M, Sharipov R, Kondrakhin Y, Makeev VJ,
Kulakovskiy IV, Kel A, Kolpakov F Nucleic Acids Res. 2021 Jan
8;49(D1):D104-D111.

\bibitem{CHIPATLAS} Zou, Z., Ohta, T., Miura, F., Oki, S. ChIP-Atlas 2021
update: a data-mining suite for exploring epigenomic landscapes by fully
integrating ChIP-seq, ATAC-seq and Bisulfite-seq data. Nucleic Acids Res.
50(W1), W175-W182, 2022, \newline
\url{http://dx.doi.org/10.1093/nar/gkac199}.

\bibitem{CHIPATLAS2} Zhaonan Zou, Tazro Ohta, Fumihito Miura, Shinya Oki,
ChIP-Atlas 2021 update: a data-mining suite for exploring epigenomic landscapes
by fully integrating ChIP-seq, ATAC-seq and Bisulfite-seq data, Nucleic Acids
Research, Volume 50, Issue W1, 5 July 2022, Pages W175–W182,
\url{https://doi.org/10.1093/nar/gkac199}.

\bibitem{ARTICLE1} Nakato R, Sakata T. Methods for ChIP-seq analysis: A
practical workflow and advanced applications. Methods. 2021 Mar;187:44-53.
doi: 10.1016/j.ymeth.2020.03.005. Epub 2020 Mar 30. PMID: 32240773.

\bibitem{ARTICLE2} \url{https://www.cd-genomics.com/chip-seq.html}
\colorbox{green}{PERTVARKOMA...}

\bibitem{SONICATION}
\url{https://goldbio.com/articles/article/how-to-fragment-DNA-for-NGS}
\colorbox{green}{PERTVARKOMA...}

\bibitem{ARTICLE3} Shah, A. Chromatin immunoprecipitation sequencing (ChIP-Seq)
on the SOLiD™ system. Nat Methods 6, ii–iii (2009).
\url{https://doi.org/10.1038/nmeth.f.247}.

\bibitem{ARTICLE4} Liu, E.T., Pott, S. \& Huss, M. Q\&A: ChIP-seq technologies
and the study of gene regulation. BMC Biol 8, 56 (2010).
\url{https://doi.org/10.1186/1741-7007-8-56}.

\bibitem{ARTICLE5} \url{https://www.cd-genomics.com/the-advantages-and-workflow-of-chip-seq.html}
\colorbox{green}{PERTVARKOMA...}

\bibitem{ARTICLE6} Langmead, B., Trapnell, C., Pop, M. et al. Ultrafast and
memory-efficient alignment of short DNA sequences to the human genome. Genome
Biol 10, R25 (2009), \newline
\url{https://doi.org/10.1186/gb-2009-10-3-r25}.

\bibitem{ARTICLE7} Li H, Durbin R. Fast and accurate short read alignment with
Burrows-Wheeler transform. Bioinformatics. 2009 Jul 15;25(14):1754-60. doi: 10.
1093/bioinformatics/btp324. Epub 2009 May 18. PMID: 19451168; PMCID: PMC2705234.

\bibitem{ARTICLE9} Lee TI, Young RA. Transcriptional regulation and its
misregulation in disease. Cell. 2013 Mar 14;152(6):1237-51.
doi: 10.1016/j.cell.2013.02.014. PMID: 23498934; PMCID: PMC3640494.

\bibitem{ASP1} \url{https://myriad.com/womens-health/diseases/autoimmune-polyglandular-syndrome-type-1/}
\colorbox{green}{PERTVARKOMA...}

\bibitem{ARTICLE10} Lambert SA, Jolma A, Campitelli LF, Das PK, Yin Y, Albu M,
Chen X, Taipale J, Hughes TR, Weirauch MT. The Human Transcription Factors.
Cell. 2018 Feb 8;172(4):650-665. doi: 10.1016/j.cell.2018.01.029.
Erratum in: Cell. 2018 Oct 4;175(2):598-599. PMID: 29425488.

\bibitem{ARTICLE11} Mundade R, Ozer HG, Wei H, Prabhu L, Lu T. Role of ChIP-seq
in the discovery of transcription factor binding sites, differential gene
regulation mechanism, epigenetic marks and beyond. Cell Cycle.
2014;13(18):2847-52. doi: 10.4161/15384101.2014.949201. PMID: 25486472;
PMCID: PMC4614920.

\bibitem{ARTICLE12} Zhang, Y., Liu, T., Meyer, C.A. et al. Model-based Analysis
of ChIP-Seq (MACS). Genome Biol 9, R137 (2008).
\url{https://doi.org/10.1186/gb-2008-9-9-r137}.

\bibitem{ARTICLE13} \url{https://epigenie.com/wp-content/uploads/2013/02/Peak-Calling-for-ChIP-Seq.pdf}
\colorbox{green}{PERTVARKOMA...}

\bibitem{ARTICLE14} Guo Y, Papachristoudis G, Altshuler RC, Gerber GK, Jaakkola
TS, Gifford DK, Mahony S. Discovering homotypic binding events at high spatial
resolution. Bioinformatics. 2010 Dec 15;26(24):3028-34.
doi: 10.1093/bioinformatics/btq590. Epub 2010 Oct 21. PMID: 20966006;
PMCID: PMC2995123.

\bibitem{ARTICLE15} Guo Y, Mahony S, Gifford DK. High resolution genome wide
binding event finding and motif discovery reveals transcription factor spatial
binding constraints. PLoS Comput Biol. 2012;8(8):e1002638. doi: 10.1371/journal.
pcbi.1002638. Epub 2012 Aug 9. PMID: 22912568; PMCID: PMC3415389.

\bibitem{ARTICLE16} Ji H, Jiang H, Ma W, Wong WH. Using CisGenome to analyze
ChIP-chip and ChIP-seq data. Curr Protoc Bioinformatics. 2011 Mar;
Chapter 2:Unit2.13. doi: 10.1002/0471250953.bi0213s33. PMID: 21400695;
PMCID: PMC3072298.

\bibitem{ARTICLE17} Enrique Blanco, Luciano Di Croce, Sergi Aranda. Comparative
ChIP-seq (Comp-ChIP-seq): a novel computational methodology for genome-wide
analysis. bioRxiv. 2019 January 29. doi: 10.1101/532622. \newline
\url{https://www.biorxiv.org/content/early/2019/03/26/532622}.

\bibitem{R} R Core Team (2022). R: A language and environment for statistical
computing. R Foundation for Statistical Computing, Vienna, Austria. URL
\url{https://www.R-project.org/}.

\bibitem{GGPLOT2} H. Wickham. ggplot2: Elegant Graphics for Data Analysis.
Springer-Verlag New York, 2016.

\bibitem{SHINY} Chang W, Cheng J, Allaire J, Sievert C, Schloerke B, Xie Y,
Allen J, McPherson J, Dipert A, Borges B (2023). shiny: Web Application
Framework for R. R package version 1.7.4.9002, \url{https://shiny.rstudio.com/}.

\bibitem{CHIPSEEKER} Wang Q, Li M, Wu T, Zhan L, Li L, Chen M, Xie W, Xie Z,
Hu E, Xu S, Yu G (2022). “Exploring epigenomic datasets by ChIPseeker.”
Current Protocols, 2(10), e585. doi: 10.1002/cpz1.585.

\bibitem{GRANGES} Lawrence M, Huber W, Pag\`es H, Aboyoun P, Carlson M,
et al. (2013) Software for Computing and Annotating Genomic Ranges. PLoS Comput
Biol 9(8): e1003118. doi:10.1371/journal.pcbi.1003118.

\bibitem{TXDB_MM} Team BC, Maintainer BP (2019). 
TxDb.Mmusculus.UCSC.mm10.knownGene: Annotation package for TxDb object(s).
R package version 3.10.0.

\bibitem{GGSEQLOGO} Wagih O (2017). ggseqlogo: A 'ggplot2' Extension for
Drawing Publication-Ready Sequence Logos. R package version 0.1, \newline
\url{https://CRAN.R-project.org/package=ggseqlogo}.

\bibitem{BSGENOME} Pagès H (2023). BSgenome: Software infrastructure for
efficient representation of full genomes and their SNPs. R package version
1.66.3, \newline
\url{https://bioconductor.org/packages/BSgenome}.

\bibitem{BSMUSMUSCULUS} Team TBD (2021). BSgenome.Mmusculus.UCSC.mm10: Full
genome sequences for Mus musculus (UCSC version mm10, based on GRCm38.p6). R
package version 1.4.3.

\bibitem{BIOSTRINGS} Pagès H, Aboyoun P, Gentleman R, DebRoy S (2022).
Biostrings: Efficient manipulation of biological strings. R package version
2.66.0, \newline
\url{https://bioconductor.org/packages/Biostrings}.

\bibitem{HOMER} Heinz S, Benner C, Spann N, Bertolino E et al. Simple
Combinations of Lineage-Determining Transcription Factors Prime cis-Regulatory
Elements Required for Macrophage and B Cell Identities. Mol Cell 2010 May
28;38(4):576-589. PMID: 20513432.

\bibitem{HOCOMOCO} Ivan V. Kulakovskiy; Ilya E. Vorontsov; Ivan S. Yevshin;
Ruslan N. Sharipov; Alla D. Fedorova; Eugene I. Rumynskiy; Yulia A. Medvedeva;
Arturo Magana-Mora; Vladimir B. Bajic; Dmitry A. Papatsenko; Fedor A. Kolpakov;
Vsevolod J. Makeev. Nucl. Acids Res., Database issue, gkx1106
(11 November 2017), doi: 10.1093/nar/gkx1106.

\end{thebibliography}
\newpage

%%%%%%%%%%%%
% PRIEDAI
%%%%%%%%%%%%

\section{Priedas} \label{Priedas}
Priedų sąraše pateikiamos Baigiamojo darbo, Kursinio darbo bei Kursinio projekto
Git repozitorijų nuorodos. Taip pat pateiktos OneDrive nuorodos į
darbo metu analizuotus pavyzdinius ChIP sekoskaitos mėginius bei atrinktų
transkripcijos faktorių pozicines svorių matricas.

\begin{itemize}
    \item \textbf{Baigiamojo darbo Git repozitorija:}\\
        \url{https://github.com/dansta0804/TF\_analysis}
    \item \textbf{Kursinio projekto analizės Git repozitorija:}\\
        \url{https://github.com/dansta0804/Tbx5\_analysis\_II.git}
    \item \textbf{Kursinio darbo analizės Git repozitorija:}\\
        \url{https://github.com/dansta0804/Tbx5\_analysis.git}
\end{itemize}

\end{document}