\documentclass[12pt]{article}
\usepackage{indentfirst}
\usepackage[utf8x]{inputenc}
\usepackage[T1]{fontenc}
\usepackage[english,lithuanian]{babel}
\usepackage{array}
\usepackage{caption}
\usepackage{subcaption}
\usepackage{makecell}
\usepackage[euler]{textgreek}
\usepackage{multirow}
\usepackage{boldline}
\usepackage{floatrow}
\floatsetup[table]{capposition=top}
\usepackage{amsmath, amsthm, amssymb}
\usepackage{graphicx}
\usepackage{setspace}
\usepackage{verbatim}
\usepackage[left=3cm,top=2cm,right=1.5cm,bottom=2cm]{geometry}
\usepackage{floatrow}
\newfloatcommand{capbtabbox}{table}[][\FBwidth]
\usepackage{blindtext}
\onehalfspacing
\usepackage[hidelinks, unicode]{hyperref}
\usepackage{textcomp}
\usepackage{amsmath}
\usepackage{cleveref}
\usepackage[labelfont=bf]{caption}
\captionsetup[table]{font={normalfont},format=plain,labelsep=period}
\captionsetup[figure]{font={normalfont},format=plain,labelsep=space}

\newcommand{\EE}{\mathbb{E}\,}
\newcommand{\ee}{{\mathrm e}}
\newcommand{\dd}{{\mathrm d}}
\newcommand{\RR}{\mathbb{R}}

\begin{document}
\selectlanguage{lithuanian}

\begin{titlepage}
\vskip 20pt
\begin{center}
\includegraphics[scale=0.5]{MIF}
\end{center}

%%%%%%%%%%%%%%%%%%%%%%%
% TITULINIS PUSLAPIS
%%%%%%%%%%%%%%%%%%%%%%%

\vskip 20pt
\centerline{\bf \large \textbf{VILNIAUS UNIVERSITETAS}}
\bigskip
\centerline{\large \textbf{MATEMATIKOS IR INFORMATIKOS FAKULTETAS}}
\bigskip
\centerline{\large \textbf{BIOINFORMATIKOS BAKALAURO STUDIJŲ PROGRAMA}}

% Sukurti algoritmą bei jo implementaciją R \emph{Shiny}, kurie leistų nuspėti
% transkripcijos faktorių prisijungimo vietas, remiantis sekų paašumu.


\vskip 90pt
\begin{center}
    {\bf \LARGE Transkripcijos faktorių taikinių spėjimo metodo kūrimas ir
    implementacija \emph{Shiny} aplikacijoje}
\end{center}
\begin{center}
    {\bf \Large The development and implementation of transcription factors'
    target prediction method in \emph{Shiny} application}
\end{center}
\vskip 20pt
\centerline{\bf \large \textbf{Bakalauro baigiamasis darbas}}
\bigskip
\vskip 40pt

\hskip 140pt {\large Autorė: Danielė Stasiūnaitė}

\hskip 140pt{\large VU el. p.: daniele.stasiunaite@mif.stud.vu.lt}
\bigskip
\vskip 20pt

\hskip 140pt {\large Darbo vadovė: J. m. d. Kotryna Kvederavičiūtė}
\vskip 60pt
\vskip 40pt
\centerline{\large \textbf{Vilnius}}
\centerline{\large \textbf{2023}}
\newpage
\end{titlepage}

\selectlanguage{lithuanian}

%%%%%%%%%%%%%%%%%%%%%
% TURINIO PUSLAPIS
%%%%%%%%%%%%%%%%%%%%%

\tableofcontents
\newpage

%%%%%%%%%%%%%%%%%%%%%%%%%%%%%%%%%%%%
% LIETUVIŠKOS SANTRAUKOS PUSLAPIS
%%%%%%%%%%%%%%%%%%%%%%%%%%%%%%%%%%%%

\section*{Santrauka}
Šiais laikais plačiai atliekami biologinių procesų tyrimai yra itin svarbūs,
siekiant suprasti įvairiuose organizmuose vykstančius gyvybiškai svarbius
biologinius procesus. Pastarųjų
procesų mechanizmų supratimas daro įtaką šiuolaikiniui įvairių žmogaus ligų
gydymui bei prevencijai. Pavyzdžiui, tobulėjantys tyrimų metodai bei įrankiai
padėjo išsiaiškinti, kad vėžiniai susirgimai, neurologiniai sutrikimai, širdies
ir kraujagyslių ligos, diabetas ir net nutukimas gali būti reguliatorinių sekų, 
transkripcijos faktorių bei kitų, genų reguliavime dalyvaujančių elementų,
mutacijų pasekmė  \cite{ARTICLE0}.

Genų reguliavimas yra vienas iš svarbiausių procesų, kurio metu yra
ekspresuojami konkretūs genai. Baltymus koduojančių genų ekspresijos metu yra
susintetinami baltymai, atliekantys įvairias funkcijas: įeinantys į struktūrinių
audinių sudėtį bei organizmų apsauginių elementų rinkinį, turintys medžiagų
pernašos bei organizmo homeostazės palaikymo funkcionalumą, dalyvaujantys
organizmų embriogenezės, tolimesnio vystymosi bei kituose biologiniuose
procesuose, užtikrinančiuose normalų organizmų funkcionavimą. Atitinkamų genų
aktyvavimas bei išaktyvinimas yra valdomas transkripcijos faktorių, kurie lemia
organizmui reikalingų baltymų sintezę, arba, įvykus transkripcijos faktorių
mutacijoms ar pasireiškus nepalankioms sąlygoms, ligų atsiradimą.

Šiame darbe, naudojantis R programavimo kalbos bibliotekomis bei
bioinformatiniais komandinės eilutės įrankiais, sukurtas transkripcijos faktorių
taikinius pasirinkto organizmo genome spėjantis metodas, pritaikantis pozicines
transkripcijos faktorių matricas (PWM). Šis metodas kartu su ChIP sekoskaitos
duomenis apdorojančiomis biologinėmis analizėmis implementuotas \emph{Shiny}
aplikacijoje. Naudojantis sukurta aplikacija galima analizuoti iš skirtingų
organizmų išgautus ChIP sekoskaitos duomenis bei, pagal poreikį, panaudoti
sugeneruotas rezultatų lenteles ir vizualizacijas.

\hfill \break
\textbf{Raktiniai žodžiai:} genų reguliavimas, transkripcijos faktorius, 
transkripcijos faktoriaus taikinys, R, \emph{Shiny} aplikacija, pozicinė svorių
matrica, ChIP sekoskaita.

\newpage

%%%%%%%%%%%%%%%%%%%%%%%%%%%%%%%%%%
% ANGLIŠKOS SANTRAUKOS PUSLAPIS
%%%%%%%%%%%%%%%%%%%%%%%%%%%%%%%%%%

\section*{Summary}
The investigation of biological processes is remarkably important in order to
understand the processes that occur in all living organisms. The understanding
of aforementioned processes and their mechanisms plays a crucial role in
disease treatment and prevention. For instance, the latest advances in research
methods and tools helped to determine that diseases like cancer, neurological
disorders, cardiovascular diseases, diabetes and even obesity can be caused
by mutations that occur in regulatory sequences, transcription factors and
other elements that are involved in gene regulation  \cite{ARTICLE0}.

Gene regulation is one of the biological processes that invokes expression of
particular genes. During the expression of protein coding genes the proteins
that perform different functions are synthethised. The functions involve the
formation of structural frameworks, immune function, substance transportation,
homeostasis, embryogenesis, later development and other biological processes
that maintain proper functioning of all living organisms. The activation and
inactivation of certain genes is controlled by transcription factors that can
either start protein synthesis or, in case of being mutation - affected,
result in the development of the diseases.

In this work, a method that predicts transcription factor target sites in
specified genome by using positional weight matrix (PWM) was created using R
programming language libraries and bioinformatical command line tools. The
method along with biological analyses that process ChIP sequencing data was
implemented in \emph{Shiny} application. The application allows to analyze
sequencing data that is retrieved from various organisms' samples. Furthermore,
the application is capable of generating result tables and visualizations that
can be downloaded and used in research.

\hfill \break
\textbf{Keywords:} gene regulation, transcription factor, transcription factor's
target, R, \emph{Shiny} application, positional weight matrix, ChIP
sequencing.

\newpage

%%%%%%%%%%%%%%%%%%%
% ĮVADO PUSLAPIS
%%%%%%%%%%%%%%%%%%%

\section{Įvadas}
\subsection*{Darbo temos aktualumas}

Įvairūs mokslo laimėjimai bei tobulėjantys tyrimų metodai stipriai prisidėjo
prie priemonių, leidžiančių diagnozuoti ligas bei rasti efektyvius
būdus jų gydymui, vystymo \cite{ARTICLE18}. Nepaisant jau esančių sėkmingai
realizuotų pažangių medicinos diagnostikos bei gydymo metodų, įvairūs tyrimai
tebėra atliekami, siekiant išsiaiškinti skirtingų gyvuosiuose organizmuose
vyks\-tan\-čių biologinių procesų mechanizmus bei juose dalyvaujančius
faktorius.

Biologinių procesų mechanizmuose svarbų vaidmenį atlieka genų reguliavimas,
valdomas transkripcijos faktorių. Dėl to, siekiant geriau suprasti organizmuose
vykstančius procesus yra svarbu išsiaiškinti, kokie bei kaip transkripcijos
faktoriai reguliuoja genų ekspresiją. Kadangi dauguma panašaus pobūdžio tyrimų
yra atliekami su modeliniais organizmais \cite{ARTICLE19} (pavyzdžiui, pelėmis,
vaisinėmis muselėmis, zebražuvėmis, nematodais), yra svarbu išsiaiškinti,
kaip gauta išnagrinėto mechanizmo informacija gali būti įvertinta organizme,
su kuriuo tyrimai nėra arba negali būti atliekami. Dėl šios priežasties šio
darbo metu sukurta populiarios ir neretai biologinius duomenis analizuojančių
specialistų naudojamos \emph{Shiny} aplinkos aplikacija, kuri leidžia
ne tik atlikti įvairius ChIP sekoskaitos duomenų analizės etapus, bet ir
pritaikyti sukurtą metodą, nuspėjantį galimas transkripcijos faktoriaus
prisijungimo sritis organizmų, kurių tyrimų duomenys nėra žinomi, genomuose.

\subsection*{Darbo tikslas}
Sukurti algoritmą bei jo implementaciją R \emph{Shiny}, kurie leistų nuspėti
transkripcijos faktorių prisijungimo vietas, remiantis sekų panašumu.

\subsection*{Uždaviniai}
\begin{itemize}
    \item Išanalizuoti metodus, taikomus praturtintų nuskaitymų - pikų -
          nustatymui;
    \item Įgyvendinti transkripcijos faktorių taikinių spėjimo metodą;
    \item Modifikuoti ir adaptuoti sukurtas R programas, galinčias apdoroti
          ChIP sekoskaitos duomenis iš skirtingų organizmų;          
    \item Sukurti aplikaciją, leidžiančią įvertinti pateiktų ChIP sekoskaitos
          duomenų kokybę bei atlikti analizes.
\end{itemize}

\newpage

%%%%%%%%%%%%%%%%%%
% TEORINĖ DALIS
%%%%%%%%%%%%%%%%%%

\section{Literatūros apžvalga}
\textbf{ChIP sekoskaita} yra viena iš svarbiausių technologijų, atliekant
epigenetikos tyrimus \cite{ARTICLE1}. Ši sekoskaita derina klasikinį chromatino
imunoprecipitacijos metodą su naujos kartos sekoskaita (NKS)
(angl. \emph{NGS} - \emph{\textbf{N}ext \textbf{G}eneration \textbf{S}equencing}).
ChIP sekoskaita taikoma, analizuojant tam tikro tipo baltymų - transkripcijos
faktorių - sąveiką su DNR, siekiant geriau suprasti genų ekspresijos reguliavimo
mechanizmus.

\subsection{Transkripcijos faktoriai ir jų reikšmė}
\textbf{Transkripcijos faktorius} - ypatingas baltymų tipas. Šio tipo baltymai
atpažįsta specifines DNR sekas ir tokiu būdu kontroliuoja chromatino struktūros
kondensacijos laipsnį bei atitinkamų genų ekspresijos procesus, inicijuojant
arba slopinant genų transkripciją. Šių baltymų sąveika su DNR veikia daug
biologiškai svarbių procesų: ląstelių diferenciaciją, ląstelės ciklo eigą, genų
transkripciją, DNR replikaciją, imuninio atsako valdymą ir daugelį kitų
procesų \cite{ARTICLE10, ARTICLE11}.

Transkripcijos faktoriai bei sekos, prie kurių jie jungiasi, gali
mutuoti. Šios pastarųjų baltymų mutacijos nulemia įvairių ligų išsivystymą.
Pavyzdžiui, atsiradusios mutacijos AIRE (autoimuniniame reguliatoriuje)
transkripcijos faktoriuje sukelia I tipo autoimuninį poliendokrinopatijos
sindromą \cite{ARTICLE9} (angl. \emph{APS1 - \textbf{A}utoimmune
\textbf{P}olyendocrinopathy \textbf{S}yndrome type \textbf{I}}). Pasireiškus
šiam sindromui organizmo imuninės ląstelės naikina sveikas, hormonus
išskiriančių liaukų ląsteles \cite{ASP1}, todėl transkripcijos faktorių veikimo
mechanizmų supratimas yra itin svarbus, siekiant diagnozuoti bei išgydyti ligas,
kurios gali būti susijusios su transkripcijos faktorių baltymų mutacijomis.

\subsection{ChIP sekoskaita ir jos vykdymo eiga}
Taikant chromatino imunoprecipitacijos sekoskaitos, kuri neretai žymima
\emph{ChIP-seq}, technologiją galima išsiaiškinti baltymų sąveikas su DNR ir
nustatyti, kaip transkripcijos faktoriai ir kiti, su chromatinu susiję baltymai,
gali įtakoti įvairius fenotipo pokyčių mechanizmus \cite{ARTICLE2}. Taip pat
ši technologija neretai taikoma atliekant: histonų modifikavimo, genų
reguliacijos, trans\-krip\-ci\-jos komplekso surinkimo, DNR pažaidų taisymo,
vystymosi mechanizmų bei ligų progresavimo tyrimus.

Viena iš svarbiausių priežasčių, kodėl chromatino imunoprecipitacijos metodas
derinamas su NKS sekoskaita, yra ta, jog naujos kartos sekoskaitos technologijos
leidžia nuskaityti didelį kiekį DNR arba RNR sekų daug sparčiau ir pigiau nei
klasikinė \emph{Sendžerio} sekoskaita ar kiti eksperimentiniai metodai. Dėl to,
derinant chromatino imunoprecipitacijos metodą su NKS galima gauti tikslius ir
prasmingus rezultatus. Remiantis gautais rezultatais galima daryti išvadas apie
baltymų ir DNR sąveikas bei kitus tyrimų metu analizuojamus procesus.

% KABUTĖS: „ “

\newpage

DNR ir baltymų sąveikos tyrimo - chromatino imunoprecipitacijos sekoskaitos -
eiga pateikta pirmame paveiksle (\hyperref[fig:image1]{1 pav.}):

\begin{figure}[ht]
    \begin{center}
        \captionsetup{justification=centering}
        \includegraphics[width=0.7\linewidth]{../Figures/ChIP-seq_workflow.jpg}
        \vspace{-1\baselineskip}
        \caption{\small\textbf{ChIP sekoskaitos vykdymo etapai.}\\
            \textbf{(A)} etape vaizduojamas mėginių paruošimo etapas:
            transkripcijos faktoriaus prijungimas prie DNR, DNR suskaidymas į
            fragmentus, kurie yra sekvenuojami bei kartografuojami.\\
            \textbf{(B)} etape su kartografuotais fragmentų nuskaitymais
            atliekama kompiuterinė analizė: nuskaitymų pasiskirstymo
            vizualizavimas, pikų nustatymas, pikų anotavimas bei sekų motyvų
            nustatymas, kuriam gali būti atlikta GO analizė, arba chromatino
            būsenos anotavimas ir klasterizavimas. Adaptuota pagal Ryuichiro 
            Nakato, ir Toyonori Sakata, 2020,
            \url{https://doi.org/10.1016/j.ymeth.2020.03.005}.}
        \label{fig:image1}
    \end{center}
\end{figure}

Remiantis pateiktu paveikslu ChIP sekoskaita gali būti suskirstyta į du
pagrindinius etapus:

\begin{enumerate}
    \item Mėginių paruošimas ir sekvenavimas;
    \item Kompiuterinė analizė.
\end{enumerate}

\subsubsection{Mėginių paruošimas ir sekvenavimas}
Mėginių paruošimo bei sekvenavimo etapai įprastai skirstomi į šiuos etapus:

\begin{itemize}
    \item \textbf{Baltymo prijungimas prie DNR.} Šiame etape transkripcijos faktorius susiejamas su DNR, naudojant įvairias chemines medžiagas (formaldehidas naudojamas dažniausiai). Ši baltymo ir DNR fiksacija,
        naudojant chemines medžiagas, padeda išlaikyti baltymo - DNR
        sąsają. Šios sąsajos išsaugojimas yra būtina sąlyga imunoprecipitacijos
        proceso vykdymui.
    \item \textbf{DNR suskaidymas į fragmentus.} \emph{NGS} bibliotekos 
        paruošimui reikalingas DNR suskaidymo į fragmentus etapas. Šiame etape
        DNR dažniausiai suskaidoma į 150 - 500 nukleotidus turinčius fragmentus,
        naudojant ultragarso bangas - sonifikacijos principą. Pastarojo
        mechanizmo sukurtas vibracijos rezonansas suskaido DNR į fragmentus,
        kurių ilgis gali būti kontroliuojamas sonifikacijos ciklo ilgiu.
        Pavyzdžiui, kuo ilgiau trunka vienas ultra garso panaudojimo ciklas,
        tuo trumpesni DNR fragmentai gaunami \cite{SONICATION}.
    \item \textbf{Imunoprecipitacijos procesas.} Suskaidytos DNR fragmentai
        inkubuojami su specifiniu antikūnu, galinčiu atpažinti prie DNR
        prisijungusį baltymą - transkripcijos faktorių. Tam, jog ChIP
        sekoskaitos rezultatai būtų patikimi ir tinkami, tinkamas antikūno
        parinkimas ir jo kokybės užtikrinimas yra vienas iš svarbiausių ChIP
        sekoskaitos mėginių paruošimo etapų \cite{ARTICLE3}. Testuojant
        skirtingus antikūnus pasirenkamas tas, kurį panaudojus gaunamas
        didesnis DNR sekų, prie kurių prisijungęs transkripcijos faktorius,
        praturtinimas nei praturtinimas, kuris gautas, naudojant nespecifinį
        antikūną \cite{ARTICLE4} (pavyzdžiui, naudojant tipinį imunoglobulino G
        (IgG) antikūną).
    \item \textbf{Sekvenavimas.} Neretai sekvenavimo įrenginių pritaikymui
        reikalingi trumpų adapterių prijungimo prie gautų DNR fragmentų ir PGR
        amplifikacijos etapai - reikalingas bibliotekos sukonstravimas, kuris
        gali skirtis, priklausomai nuo pasirinktos sekvenavimo platformos
        ir jai specifinės bibliotekos paruošimo protokolo \cite{ARTICLE5}.
        Įvykdžius šiuos etapus gali būti gauti tendencingumai \cite{ARTICLE4}
        (angl. \emph{bias}), kurie gali būti mažesni, atliekant mažiau DNR
        amplifikacijos (padauginimo) ciklų. Sukonstravus biblioteką atliekama
        sekoskaita.
\end{itemize}

Įgyvendinus mėginių paruošimo ir sekvenavimo etapus atliekama gautų duomenų
kompiuterinė analizė.

\subsubsection{Kompiuterinė analizė}
Gauti ChIP sekoskaitos duomenys apdorojami ir analizuojami, vykdant šiuos
etapus:

\begin{itemize}
    \item \textbf{DNR nuskaitymų kartografavimas.} Nusekvenuoti DNR 
        fragmentai išsaugomi \emph{FASTQ} arba \emph{CSFSATQ} formatais. Šie DNR
        nuskaitymai (angl. \emph{reads} arba \emph{tags}) perkeliami ant genomo,
        naudojant kartografavimo (angl. \emph{mapping}) įrankius bei metodus,
        pavyzdžiui, programą \emph{Bowtie} \cite{ARTICLE6}, kuri paremta
        \emph{Burrows-Wheeler} metodu. \emph{Bowtie} programa leidžia nustatyti
        nuskaityto DNR fragmento poziciją genome,
        esant kelių nukleotidų neatitikimui \cite{ARTICLE7}
        (angl. \emph{mismatch}). Atlikus DNR nuskaitymų priskyrimą gaunami
        \emph{SAM}, \emph{BAM} (dažniausiai naudojamas formatas), \emph{CRAM}
        arba \emph{tagAlign} formato failai.
    \item \textbf{Normalizavimas.} Normalizavimas reikalingas, siekiant
        sumažinti techninį nuskaitymų variabilumą - sumažinti „triukšmą“,
        atsiradusį dėl sekvenavimo gylio skirtumų tarp mėginių. Normalizavimas
        atliekamas konkrečiose genomo pozicijose esančių nuskaitymų skaičių
        padalinus iš bendro nuskaitymų skaičiaus \cite{ARTICLE17}.
    \item \textbf{Pikų nustatymas.} Šiame etape
        nustatomi reikšmingai praturtinti genomo lokusai - pikai
        (angl. \emph{peaks}). Įgyvendinus šį etapą dažniausiai sugeneruojami
        \emph{BED} formato failai \cite{ARTICLE1}, kuriuose pateikiamos genominės
        pikų pozicijos, įvairūs statistiniai įverčiai bei identifikacijos kodai,
        kuriuos naudojant galima vykdyti tolimesnes analizes.
    \item \textbf{Biologinės analizės.} Dažniausiai atliekamos analizės yra
        motyvų analizė bei genų ontologijos
        (angl. \emph{GO - \textbf{G}ene \textbf{O}ntology enrichment analysis})
        praturtinimo analizė \cite{ARTICLE1},
        pateikianti biologinių procesų, ląstelinių komponentų ir molekulinių
        funkcijų, kuriose dalyvauja genas, sąrašą. Įvertinus motyvų dažnį,
        konservatyvumą bei biologines funkcijas, susijusias su transkripcijos
        faktorių prisijungimu prie šių motyvų, gali būti identifikuoti
        genominiai regionai, prie kurių gali jungtis transkripcijos faktoriai,
        konservatyvūs motyvai, galintys indikuoti baltymas - baltymas sąveiką,
        bei gali būti analizuojama genų evoliucija.
\end{itemize}

\colorbox{red}{Įvykdžius kompiuterinės analizės etapus gaunami rezultatai, suteikiantys
įžvalgų apie genų reguliavimo mechanizmus, padedantys nustatyti
transkripcijos faktorių taikinius ir geriau suprasti ligų progresavimą bei
ieškoti ligos gydymo būdų.}

\newpage

\subsection{Pikų nustatymo algoritmai}
Pikų nustatymas yra vienas iš svarbiausių etapų, atliekant DNR ir reguliatorinių
baltymų - transkripcijos faktorių arba histonų - sąveikos tyrimų analizes.
Kuriant pikų nustatymo algoritmus sprendžiamos dvi pagrindinės problemos:
genominių koordinačių, kuriose, tikėtina, yra pikai, nustatymas bei tikėtinų
pikų statistinio reikšmingumo tikrinimas.

Pagrindinė pikų nustatymo algoritmų įvestis yra kartografavimo metu su genomu
išlyginti DNR fragmentų nuskaitymai. Antrajame paveiksle
(\hyperref[fig:image2]{2 pav.}) šie fragmentai pažymėti raudona ir žalia
spalvomis.

\begin{figure}[ht]
    \begin{center}
        \captionsetup{justification=centering}
        \includegraphics[width=0.6\linewidth]{../Figures/Read_mapping.png}
        \vspace{-1\baselineskip}
        \caption{\small\textbf{DNR nuskaitymų kartografavimas.}\\Žalia
            spalva pavaizduoti 5' \(\rightarrow\) 3' DNR galo link kartografuoti
            nuskaitymai. Raudona spalva - 3' \(\rightarrow\) 5' nuskaitymai.}
        \label{fig:image2}
    \end{center}
\end{figure}

Nustačius DNR fragmentų nuskaitymų pozicijas genome kai kurios pastebimos
nuskaitymų sankaupų grupės gali indikuoti, jog toje pozicijoje yra galimas
transkripcijos faktoriaus prisijungimas (nuskaitymų sankaupa yra reikšminga),
tačiau neretai tokios sankaupos - pikai - gali būti laikomos molekuliniu arba
eksperimentiniu „triukšmu“. Taigi kuriant pikų nustatymo algoritmus yra
svarbu, jog algoritmas gebėtų įvertinti, ar pikas yra biologiškai reikšmingas,
ar tai tėra „triukšmas“.

Yra sukurta daugiau nei 30 skirtingų pikų nustatymo algoritmų (angl.
\emph{peak caller}), kurie sprendžia anksčiau minėtas problemas,
tačiau šių problemų sprendimo būdai yra skirtingi. Konkretaus algoritmo
pasirinkimas labai priklauso nuo atliekamo eksperimento tipo ir specialisto,
atliekančio analizę, patirties \cite{ARTICLE13}.

\subsubsection{MACS}
\textbf{MACS} - \emph{\textbf{M}odel-based \textbf{A}nalysis of
\textbf{C}hIP-\textbf{S}eq}. Tai yra populiariausias ir bene seniausias
(publikuotas 2008 metais \cite{ARTICLE20}) pikų nustatymo algoritmas.
Transkripcijos faktorių jungimosi prie DNR sritys nustatomos, atsižvelgus į
nuskaitymų pozicijas bei kryptį. Tose genomo pozicijose, kuriose, tikėtina,
jungiasi transkripcijos faktorius, turi būti pastebimas \emph{Watsono} ir
\emph{Kriko} nuskaitymų išsidėstymas arba \textbf{bimodalinis pasiskirtymas},
kurio grafikas pavaizduotas trečiame paveiksle (\hyperref[fig:image3]{3 pav.}).

\begin{figure}[ht]
    \begin{center}
        \captionsetup{justification=centering}
        \includegraphics[width=0.5\linewidth]{../Figures/Bimodal_pattern.png}
        \vspace{-1\baselineskip}
        \caption{\small\textbf{Bimodalinis pasiskirstymas.}\\Grafikui
            būdingos dvi viršūnės - dvi modos. Rausva ir melsva spalvomis
            atskirti 5' \(\rightarrow\) 3' ir 3' \(\rightarrow\) 5'
            DNR galų link kartografuoti nuskaitymai.}
        \label{fig:image3}
    \end{center}
\end{figure}

Tam, jog panaši pikų struktūra būtų surasta, MACS2 algoritmas skanuoja visą
kartografuotų nuskaitymų duomenų rinkinį. Algoritmas naudoja dydį, kuris nurodo,
į kokio ilgio nukleotidų fragmentus buvo skaidoma DNR sonifikacijos proceso
metu (angl. \emph{bandwidth}), bei \emph{mfold} vertę. Vykdant algoritmą
atliekamas \emph{2 * bandwidth} ilgio poslinkis ir ieškoma tokių genomo
pozicijų, kuriose nuskaitymų yra daugiau nei naudojant atsitiktinį nuskaitymų
rinkinį (daugiau už \emph{mfold} vertę).

Nustačius aukštos kokybės pikus yra atsitiktinai parenkama 1000 pikų. Turint
šiuos pikus yra atskiriami jų \emph{Watson} ir \emph{Crick} (teigiamos ir
neigiamos grandinės) nuskaitymai.
Šių teigiamų ir neigiamų grandinių pikų grupės yra išlyginamos pagal jų
centrus, kaip pavaizduota ketvirtame paveiksle (\hyperref[fig:image4]{4 pav.}).
Atstumas tarp išlygintų nuskaitymų modų (\emph{d}) nurodo, kokio ilgio yra piko
fragmentas.

\begin{figure}[ht]
    \begin{center}
        \captionsetup{justification=centering}
        \includegraphics[width=0.5\linewidth]{../Figures/Tag_alignment.png}
        \vspace{-1\baselineskip}
        \caption{\small\textbf{4 pav. Nuskaitymų išlyginimas.}\\Raudona spalva
            vaizduojami \emph{Watson} nuskaitymai (5' \(\rightarrow\) 3' DNR
            galo link kartografuoti nuskaitymai),
            mėlyna spalva - \emph{Crick} nuskaitymai (3' \(\rightarrow\) 5').}
        \label{fig:image4}
    \end{center}
\end{figure}

Algoritmas visiems nuskaitymams atlieka \emph{d/2} 3' DNR galo link
tikėtiniausio DNR ir transk\-rip\-ci\-jos faktoriaus sąveikos regiono poslinkį.
Atlikus šį poslinkį atliekamas \emph{2 * d} poslinkis, jog būtų surastas
statistiškai reikšmingas nuskaitymų praturtinimas, naudojant \emph{Puasono}
skirstinį, kurio parametras \(\lambda_{BG}\) yra tikėtinas nuskaitymų skaičius
atlikus poslinkį. Nepaisant to, \(\lambda_{BG}\) parametras naudojamas,
neatsižvelgus į galimą „triukšmą“, kuris galėjo kilti dėl chromatino struktūros,
DNR amplifikacijos arba sekvenavimo, todėl yra naudojamas parametras
\(\lambda_{local}\), kuris skaičiuojamas kiekvienam tikėtinam
pikui:

\begin{equation} \label{lambda_local}
    \lambda_{local} = max(\lambda_{BG}, \lambda_{5k}, \lambda_{10k})
\end{equation}

čia \(5k\), \(10k\) yra poslinkio plotis.

Parametro \(\lambda_{local}\) naudojimas leidžia aptikti \emph{false positive},
pikus (pikus, kurie atsirado dėl „triukšmo“) ir nustatyti tik tuos pikus,
kurie indikuoja svarbų DNR ir baltymo sąveikos regioną \cite{ARTICLE12}.
  
\subsubsection{GEM}
\textbf{GEM} - \textbf{G}enome wide \textbf{E}vent finding and \textbf{M}otif
discovery. Šis 2012 metais sukurtas algoritmas išsiskiria tuo, jog jame yra
kombinuojama pikų paieška bei motyvų analizė, jog būtų pagerinta galutinių pikų
rezoliucija\footnote{\textbf{Rezoliucija - } genetikoje aukšta rezoliucija
reiškia, jog yra žinoma itin daug molekulinių detalių apie DNR.}.\\

GEM algoritmą sudaro šeši skirtingi etapai \cite{ARTICLE15}:
\begin{enumerate}
    \item \textbf{Baltymo ir DNR sąveikos regionų nustatymas.} Pradiniai
        regionai nustatomi, taikant \emph{GPS} algoritmą \cite{ARTICLE14},
        kuris naudoja \emph{Dirichlė} skirstinį.
    \item \textbf{Praturtintų \emph{k - merų} nustatymas.} Jie nustatomi,
        lyginant \emph{k - merų} dažnius tarp teigiamų sekų ir neigiamų
        kontrolinių sekų. Teigiamos sekos - sekos, kurios sudarytos iš 61
        bazių poros ir yra išsidėsčiusios spėjamų baltymo ir DNR sąveikos
        regionų (gautų pirmajame etape) centruose. Neigiamos kontrolinės
        sekos - 61 bazių porą turinčios sekos, kurios yra nutolusios nuo
        teigiamų sekų per 300 bazių porų. Be to, šios sekos nepersidengia su
        sekomis, esančiomis baltymo - DNR sąveikos sekų centruose. Šiame
        etape yra skaičiuojami \emph{k - merų} fragmentų atitikimai teigiamų
        ir neigiamų sekų rinkiniuose. \emph{K - meras} (sekos fragmentas)
        yra laikomas praturtintu, kai \emph{p} vertė yra mažesnė nei 0.001.
        % ir 3-fold enrichment in terms of positive/negative hit count?
    \item \textbf{Praturtintų \emph{k - merų} klasterizavimas.} Praturtinti
        \emph{k - merai} klasterizuojami į ekvivalentiškumo klases, kurios
        apibūdina panašias DNR sekas, prie kurių jungiasi transkripcijos
        faktorius. Seka atitinka \emph{k - mero} ekvivalentiškumo klasę, kai
        sekoje nustatomas fragmentas yra vienas iš ekvivalentiškumo klasės
        elementų.
    \item \textbf{Išankstinio pasiskirstymo nustatymas.} Labiausiai praturtinta
        \emph{k - merų} klasė yra naudojama \emph{Dirichlė} išankstinio
        pasiskirstymo paskaičiavimui. Šiame etape genomas yra suskaidomas į
        kelis tūkstančius bazių porų turinčius segmentus. Šie segmentai yra
        gaunami atskiriant DNR fragmentus, kuriuose yra daugiau nei 500 bazių
        porų turintys tarpai bei DNR regionai, kuriems buvo priskirta mažiau
        nei 6 DNR nuskaitymai (angl. \emph{reads}). Šie regionai yra
        skanuojami su DNR sekų fragmentais, kurie priklauso atrinktai
        \emph{k - merų} ekvivalentiškumo klasei, \emph{k - merų}
        atitikimai yra skaičiuojami.
    \item \textbf{Tikslesnių baltymo - DNR sąveikos regionų spėjimas.} Tam yra
        panaudojamas 4 etape gautas išankstinis pozicijų pasiskirstymas.
    \item \textbf{2 - 3 etapų kartojimas.} Tam yra panaudojami 5 etape gauti
        patikslinti baltymo - DNR sąveikos regionai.
\end{enumerate}

\newpage

%%%%%%%%%%%%%%%%%%%%%%%
% METODO REALIZACIJA
%%%%%%%%%%%%%%%%%%%%%%%

% \section{Transkripcijos faktorių taikinių spėjimo metodas}
\section{Tyrimo metodai}
ChIP sekoskaitos duomenų kokybės vertinimas, jų apdorojimas, atliekant
biologines analizes, bei transkripcijos faktorių taikinių spėjimo pasirinktame
organizme metodas įgyvendintas su R programavimo kalba \cite{R} (4.2.3 versija).
Realizuoti R programiniai kodai patalpinti Git repozitorijoje:
\small{\url{https://github.com/dansta0804/TF\_analysis}}.


%%%%%%%%%%%%%%%%%%%
% TYRIMO METODAI
%%%%%%%%%%%%%%%%%%%

\subsection{Duomenų kokybės vertinimas}
Atsisiuntus ChIP sekoskaitos mėginius (\ref{table:mouse_samples},
\ref{table:human_samples}) yra atliekamas mėginių duomenų kokybės
vertinimas, siekiant geriau suprasti, kokia genominė informacija gauta
eksperimentų metu bei ar gauta informacija yra kokybiška. Nustačius, kad
analizuojamuose mėginiuose nėra išskirčių arba sunkiai paaiškinamų rezultatų,
gali būti atliekamos biologinės analizės, kurių rezultatas, tikėtina, yra
teisingas bei interpretuojamas.

\subsubsection*{Pikų skaičius mėginiuose}
Pirmajame duomenų kokybės įvertinimo etape pikų skaičiaus nustatymui mėginiuose
pritaikyta bazinė R programavimo kalbos funkcija \emph{length()}. Pastaroji
funkcija leidžia apskaičiuoti, kiek kiekviename mėginyje yra pikų regionų.
Pikų skaičius mėginiuose vizualizuotas su \emph{ggplot2}
\cite{GGPLOT2} bibliotekos funkcija \emph{geom\_bar()}.

\subsubsection*{Pikų skaičius chromosomose}
Atliekant šį duomenų kokybės įvertinimo etapą panaudota funkcija
\emph{facet\_wrap()}, kuri nustato, kiek pikų yra nustatyta skirtingose
chromosomose. Pikų skaičiaus pasiskirstymą skirtingose chromosomose
vizualizuojančios stulpelinės diagramos sukurtos su \emph{ggplot2} funkcija
\emph{geom\_bar()}.

\subsubsection*{Mėginių panašumas}
Mėginių tarpusavio panašumui įvertinti realizuota modifikuota
\emph{Jaccard()} funkcija, kur mėginių A ir B sutampančių duomenų ilgis 
padalintas iš A duomenų rinkinio ilgio:

\[ J(A, B) = \frac{|A \cap B|}{|A|} \]

Mėginių panašumas vizualizuotas, panaudojus spalvų intensyvumo grafiką -
pritaikius \emph{ggplot2} bibliotekos funkciją \emph{geom\_tile()}.

\newpage

\subsubsection*{Genominė distribucija}
Šio duomenų kokybės vertinimo etapo metu kiekvienam mėginiui sukurtas grafikas,
vaizduojantis kiekvieno genominio elemento (promotorių, intronų, egzonų,
tarpgeninių sričių) procentinę dalį. Tokio tipo vizualizacija sukurta,
pritaikius \emph{ChIPseeker} \cite{CHIPSEEKER} bibliotekos funkciją
\emph{plotAnnoBar()}.

\subsubsection*{Atstumas iki TSS}
Atstumo iki artimiausio transkripcijos pradžios taško
(angl. \emph{Transcription Start Site}) nustatymas įgyvendintas, pritaikius
\emph{ChIPseeker} bibliotekos funkciją \emph{annotatePeak()}, kuriai argumentų
pavidalu perduoti pikus aprašantys \emph{GRanges} \cite{GRANGES} objektai bei
visus žinomus konkretaus organizmo genus aprašantis \emph{TxDb} \cite{TXDB_MM}
objektas. Atlikus pikų anotavimą gauta genominių elementų procentines dalis
apibendrinanti lentelė, kuri perduota \emph{ChIPseeker} funkcijai
\emph{plotDistToTSS()}. Pastaroji funkcija sukūrė grafiką, atvaizduojantį
kiekvieno mėginio pikų atstumą iki artimiausio \emph{TSS} regiono.

\subsubsection*{Pikų profilio atvaizdavimas}
Prieš skaičiuojant pikų, kurie jungiasi prie \emph{TSS} regionų, profilį yra
paruošiami tie \emph{TSS} regionai, kurie yra vadinami „\emph{TSS}
supančiais regionais“ (angl. \emph{flanking sequences}). Šie regionai nustatyti,
pritaikius \emph{ChIPseeker} bibliotekos funkciją \emph{getPromoters()}.
Pastarajai funkcijai perduotas konkretaus organizmo \emph{TxDb} objektas.
Nustačius šalia \emph{TSS} regionų esančių sekų regionus panaudota
\emph{getTagMatrix()} funkcija, sukurianti nuskaitymų, kurie patenka į šalia
\emph{TSS} sričių esančius sekų regionus, matricą. Gautą matricą perdavus
\emph{ChIPseeker} funkcijai \emph{plotAvgProf()} gautas pikų profilis.

\newpage

\subsection{Biologinių analizių atlikimas}
Įvertinus ChIP sekoskaitos duomenų kokybę vykdomos biologinės analizės,
kurių metu, pasitelkus įvairias R bibliotekas, galima gauti rezultatus, kurie
gali būti panaudoti atliekamuose tyrimuose. Taip pat šiame etape gauti
rezultatai gali padėti geriau suprasti genų reguliavimą, prie konkrečių
genominių sričių jungiantis transkripcijos faktoriams, bei su kokiomis
biologinėmis funkcijomis gali būti susiję anotuoti pikai.

\subsubsection*{Transkripcijos faktoriaus motyvo logo}
Transkripcijos faktoriaus motyvo sekos logotipas sukurtas su R bibliotekos
\emph{ggseqlogo} \cite{GGSEQLOGO} funkcija \emph{ggseqlogo()}, kuriai perduota
transponuota konkretaus transkripcijos faktoriaus pozicinė svorių matrica.
Šios matricos atsisiųstos iš HOCOMOCO \cite{HOCOMOCO} duomenų bazės (11.0
versija), saugančios 680 žmogaus (lot. \emph{Homo sapiens}) bei 453 naminės
pelės (lot. \emph{Mus musculus}) transkripcijos faktorių pozicines svorių
matricas.

\subsubsection*{PWM matricos atitikimų skaičiavimas}
ChIP sekoskaitos duomenys R saugomi \emph{GRanges} objektų pavidalu.
Pastarieji objektai, saugantys genomines sekoskaitos duomenų pozicijas,
panaudoti šias pozicijas atitinkančių nukleotidų sekų išgavimui. Nukleotidai,
patenkantys į tam tikrus regionus, išgauti panaudojus bibliotekos
\emph{BSgenome} \cite{BSGENOME} funkciją \emph{getSeq()}. Pastarajai funkcijai
perduotas organizmo genomo anotacijos objektas
\emph{BSgenome.Mmusculus.UCSC.mm10} \cite{BSMUSMUSCULUS} bei pasirinktų
mėginių (\ref{table:mouse_samples}) \emph{GRanges} objektas. Išgautų nukleotidų
sekų \emph{DNAStringSet} objektas perduotas \emph{Biostrings} \cite{BIOSTRINGS} 
bibliotekos funkcijai \emph{countPWM()} kartu su tiriamo transkripcijos
faktoriaus pozicine svorių matrica (pozicinių svorių matricos failai prieinami
\href{https://vult-my.sharepoint.com/:f:/g/personal/daniele_stasiunaite_mif_stud_vu_lt/EtEGQ8POkapLhPv6eHvl48cB-jmes81M0JPW8PVWTz2QgA?e=wjSSKJ}{\textbf{\emph{šioje}}}
\emph{OneDrive} saugykloje).
Pastaroji funkcija suskaičiavo, kiek išgautose nukleotidų sekose yra
transkripcijos faktoriaus motyvo atitikimų. Šis skaičius padalintas iš bendro
pikų skaičiaus - gauta transkripcijos faktoriaus motyvų procentinė dalis, kuri
vizualizuota, pritaikius \emph{geom\_bar()}.

\subsubsection*{Praturtintų sekų biologinių funkcijų nustatymas}
Praturtintų sekų biologines funkcijas apibendrinanti lentelė gauta, pritaikius R
bibliotekos \emph{clusterProfiler} \cite{CLUSTERPROFILER} funkciją
\emph{enrichGO()}. Šiai funkcijai perduoti anotuotų ChIP sekoskaitos duomenų
pikų \emph{Entrez} identifikacijos numeriai, organizmo, iš kurio buvo išgautas
pasirinktas mėginys, anotacijos objektas (veikimas tikrintas su naminės pelės
anotacijos objektu \emph{org.Mm.eg.db}), subontologijos (\emph{BP} -
biologiniai procesai, \emph{MF} - molekulinės funkcijos, \emph{CC} - ląstelės
komponentai) parametras bei įvairūs statistiniai įverčiai, reikalingi genų
ontologijos analizės atlikimui.

Biologinius procesus vizualizuojantis aciklinis kryptinis grafas sukonstruotas,
perdavus funkcijos \emph{enrichGO()} rezultato objektą bibliotekos
\emph{enrichplot} \cite{ENRICHPLOT} funkcijai \emph{goplot()}.

Taip pat funkcijos \emph{enrichGO()} rezultato objektas buvo panaudotas,
atliekant hierarchinį klasterizavimą ir anotuotų pikų genus suskirsčius į
5 skirtingus klasterius, kurie vizualizuoti su \emph{enrichplot} bibliotekos
funkcija \emph{treeplot()}. Pastarosios funkcijos pritaikymui \emph{enrichGO()}
objektas perduotas \emph{pairwise\_termsim()} funkcijai, priklausančiai
\emph{enrichplot} bibliotekai. Šioje funkcijoje realizuotas originalaus
\emph{Jaccard} panašumo indekso tarp skirtingų biologinių procesų skaičiavimas,
leidžiantis suskirstyti biologinius procesus į klasterius.

\subsubsection*{Motyvų paieška \emph{de novo}}
Praturtintos sekos nustatytos, pritaikius komandinės eilutės įrankio
HOMER \cite{HOMER} \emph{Perl} programą \emph{findMotifsGenome}, analizuojančią
ChIP sekoskaitos mėginius (\ref{table:mouse_samples}). Taip pat taikant
šią programą nurodytas organizmo, iš kurio išgauti mėginiai, referentinio genomo
trumpinys (pavyzdžiui, naminės pelės referentinis genomas \emph{mm10}).
Algoritmo veikimo pabaigoje sugeneruotos lentelės, aprašančios informaciją
apie identifikuotus \emph{de novo} motyvus.

\newpage

\subsection{Transkripcijos faktorių taikinių spėjimo metodas}
Transkripcijos faktorių taikinių spėjimo metodas leidžia nustatyti, kokiose
pasirinkto organizmo genomo srityse (genuose) yra įmanomas nurodyto
transkripcijos faktoriaus prisijungimas, naudojant kito organizmo genomines
sekas. Metodas realizuotas, įgyvendinus aprašytus etapus.

\subsubsection{Metodo implementacijos schema}
Penktame paveiksle (\hyperref[fig:image5]{5 pav.}) vaizduojama metodo
implementacijos etapus apibendrinanti schema:

\begin{figure}[ht]
    \begin{center}
        \captionsetup{justification=centering}
        \includegraphics[width=0.75\linewidth]{../Figures/Schema.png}
        \vspace{-1\baselineskip}
        \caption{\small\textbf{Metodo implementacijos etapai.}\\Čia
        \emph{organizmas 1} - organizmas, iš kurio išgauti ChIP sekoskaitos
        mėginiai; \emph{organizmas 2} - organizmas, kuriame spėjami
        transkripcijos faktoriaus taikiniai.}
    \label{fig:image5}
    \end{center}
\end{figure}

\newpage

\subsubsection{Pikų anotavimas}
Šiame etape yra anotuojami ChIP sekoskaitos mėginiai, išgauti iš organizmo,
kurio duomenys panaudojami, atliekant transkripcijos faktorių taikinių spėjimą.
Taip pat anotuojami ir kontroliniai ChIP sekoskaitos mėginiai, kurie išgauti iš
organizmo, kurio genome spėjamos transkripcijos faktoriaus prisijungimo vietos.
Kontroliniai mėginiai reikalingi tam, jog būtų galima įvertinti metodo tikslumą.
ChIP sekoskaitos duomenys - pikų rinkiniai - anotuojami, pritaikant sukurtą
\emph{annotate\_peak()} funkciją. Šiai funkcijai turi būti perduoti šie
parametrai:

\begin{itemize}
    \item \textbf{\emph{Pikų rinkinys:}} ChIP sekoskaitos duomenų rinkinys,
        saugomas \emph{GRanges} objektų sąraše (\emph{GRangesList}).
    \item \textbf{\emph{Ensembl genomo anotacija:}} organizmų genominės
        anotacijos iš R \emph{annotables} bibliotekos.
    \item \textbf{\emph{TxDb objektas:}} pasirinktų organizmų duomenų bazės,
        kuriose saugoma informacija įprastai atitinka GTF/GFF failų informaciją.
    \item \textbf{\emph{org.x.eg.db anotacija:}} \emph{x} atitinka organizmų
        lotyniškų pavadinimų santrumpą (pavyzdžiui, \emph{Mus musculus} -
        \emph{Mm}).
\end{itemize}

Perduoti parametrai panaudojami \emph{ChIPseeker} \cite{CHIPSEEKER} bibliotekos
funkcijos \emph{annotatePeak()}, kuri pikams priskiria artimiausius genus.
Taip pat šiame etape anotuotiems pikams priskiriami genų identifikacijos
numeriai bei genų simboliai.

\subsubsection{Aminorūgščių sekų išgavimas}
Anotuotų pikų rinkinių genus atitinkančios aminorūgščių sekos išgaunamos,
panaudojus \emph{ensembldb} \cite{ENSEMBLDB} bibliotekos funkciją 
\emph{proteins()}. Šiai funkcijai perduoti šie parametrai:

\begin{itemize}
    \item \textbf{\emph{Ensembl duomenų bazės objektas:}} specifikavus
        konkretaus organizmo anotacijos objektą, įvykdomas SQL susiejimas su
        duomenų baze, kurioje saugoma konkretaus organizmo \emph{Ensembl}
        anotacija.
    \item \textbf{\emph{Filtras:}} jis taikomas, kad iš \emph{Ensembl} duomenų
        bazės būtų išgauta tik tam tikra informacija. Realizuojant
        transkripcijos faktorių taikinių paieškos metodą taikytas
        \emph{GeneNameFilter()}, kuriam perduodamas pikui priskirto geno
        pavadinimas.
    \item \textbf{\emph{Grąžinamo rezultato tipas:}} siekiant išgauti
        aminorūgščių sekas formatu, kuris būtų paprastai apdorojamas
        tolimesniuose etapuose, pasirinktas \emph{AAStringSet} grąžinamo
        rezultato tipas.
\end{itemize}

\subsubsection{\emph{Blastp} paieška}
Paieškos atlikimui panaudota \emph{rBlast} \cite{RBLAST} bibliotekos funkcija
\emph{blast()}, kuriai perduotas pasirinkto organizmo baltymų duomenų bazės
failas bei nurodytas \emph{Blast} paieškos tipas - \emph{blastp}.
\emph{rBlast} biblioteka leidžia naudotis lokaliai instaliuotu
\emph{BLAST+} \cite{BLAST}, naudojant \emph{Bioconductor} in\-fra\-struk\-tū\-rą.
Specifikavus duomenų bazę, kurioje atliekama paieška, taikoma funkcija
\emph{predict()}. Šiai funkcijai perduodamas duomenų bazės objektas, pradinio
organizmo genų aminorūgščių sekų rinkinys, \emph{Blast} paieškai naudojamų gijų
skaičius bei galutinio rezultato formatas - identifikacijos numerius, sekų
ilgius, sekų identiškumo bei ieškomų sekų padengimo procentus specifikuojantys
stulpeliai.

\subsubsection{Paieškos rezultatų apdorojimas}
Gauti \emph{Blast} paieškos rezultatai sugrupuojami pagal identifikacijos
numerius ir atrenkamos tos rezultatų eilutės, kurios atitinka dvejus
\emph{dplyr} bibliotekos filtrus: maksimalų ieškomų sekų padengimo procentą bei
maksimalų sekų identiškumą. \colorbox{red}{KODĖL MAX???}

Po filtrų pritaikymo gavus mažesnį rezultatų rinkinį ieškomų sekų \emph{Ensembl}
identifikacijos numeriams priskiriami NCBI genų identifikacijos numeriai bei
genų simboliai. Taip pat šiame metodo realizavimo etape ieškomas sekas
atitinkančioms sekoms priskiriami NCBI identifikacijos numeriai bei genų
simboliai, naudojant bibliotekos \emph{clusterProfiler} \cite{CLUSTERPROFILER}
funkciją \emph{bitr}. Pastaroji funkcija surastoms sekoms priskiria
identifikacijos numerius, naudojant organizmo, kurio genome siekiama atlikti
transkripcijos faktorių taikinių spėjimą, genomo anotacijos objektą
\emph{org.x.eg.db}, kur \emph{x} - organizmo lotyniško pavadinimo pirmosios
raidės.

Ieškomų sekų informacija sujungiama su sekų, turinčių didelį sekų identiškumą
bei sekų padengimą, informacija.

\subsubsection{Duomenų rinkinio sekų išgavimas}
Šiame etape naudojami du \emph{TxDb} bibliotekos organizmų objektai,
saugantys nurodytų or\-ga\-niz\-mų genominę informaciją. Nukleotidų sekų
išgavimui iš konkretaus organizmo genomo naudojama \emph{BSgenome} bibliotekos
funkcija \emph{getSeq()}. Šiai funkcijai perduodamas pasirinkto organizmo
\emph{BSgenome} objektas bei \emph{GRanges} objektas, kuriame išsaugota
genominių regionų informacija - genominės pozicijos tų genų, kurie specifikuoti
apdorotame duomenų rinkinyje bei patenka į organizmų \emph{TxDb} objektų duomenų
rinkinį.

\newpage

\subsubsection{PWM atitikimų nustatymas}
Išgauti kiekvieno organizmo sekų rinkiniai analizuojami atskirai. Kiekvienai
sekų rinkiniui priklausančiai sekai pritaikoma
\emph{Biostrings} \cite{BIOSTRINGS} bibliotekos
funkcija \emph{countPWM()}, kuriai pateikiama pasirinkto transkripcijos
faktoriaus pozicinė svorių matrica bei \emph{min.score} vertė procentais,
nurodanti minimalų sekos fragmento bei pozicinės svorių matricos atitikimą.
Grąžintas pozicinės svorių matricos atitikimų skaičius bei geno pavadinimas
patalpinamas į \emph{data.frame} duomenų struktūrą.

Nustačius, kiek genų sekose esama pozicijų, kuriose galimas konkretaus
transkripcijos faktoriaus prisijungimas, gautos duomenų struktūros sujungiamos
į bendrą duomenų struktūrą, kuriai sukuriamas papildomas stulpelis,
specifikuojantis užklausos sekose nustatytų transkripcijos faktorių
prisijungimo procentą, palyginus transkripcijos faktorių taikinių skaičių
užklausos sekose su skaičiumi, gautu \emph{Blast} surastose sekose. 

\newpage

\subsection{Interaktyvios aplikacijos kūrimas}
Programa, vertinanti naudotojo įkeltų mėginių failų
(\ref{table:mouse_samples}, \ref{table:human_samples}) kokybę, atliekanti
biologines analizes bei realizuojanti transkripcijos faktorių taikinių spėjimo
metodą sukurta su R programavimo kalbos biblioteka \emph{Shiny} \cite{SHINY}.
Pastarosios bibliotekos funkcijos leidžia sukurti interaktyvias internetines
aplikacijas (angl. \emph{Interactive Web App}), naudojant R bei internetinių
puslapių kūrimo kalbų - HTML, CSS, JavaScript - funkcijas.

\subsubsection*{Duomenų įkėlimas}
Naudotojas gali įkelti vieną arba daugiau ChIP sekoskaitos duomenų failų,
turinčių tabuliacijos simboliais atskirtus stulpelius. Pastaruosiuose failuose
turi būti pateikta svarbiausia genominių ChIP sekoskaitos duomenų informacija -
chromosomos trumpinys bei piko pradžios ir pabaigos rėžiai. Įkėlus neteisingo
formato duomenis aplikacijoje atliekamas duomenų kokybės vertinimas, biologinės
analizės bei transkripcijos faktorių taikinių spėjimas nėra galimas.

Duomenų įkėlimo skiltyje specifikuojami papildomi parametrai, kurie panaudojami,
atliekant duomenų kokybės vertinimą, duomenų biologines analizes bei konkretaus
transkripcijos faktoriaus taikinių spėjimą pasirinktame organizme. Programos
naudotojas turi nurodyti, iš kokio organizmo išgauti mėginiai. Tai reikalinga
tam, jog iš sukurto JSON \cite{JSON} formato failo būtų nuskaityti atitinkamo
organizmo R genominių anotacijų objektai, informacija apie chromosomas ir būtų
sugeneruoti aiškūs grafikų pavadinimai. Taip pat naudotojas turi specifikuoti
transkripcijos faktoriaus pavadinimą bei pateikti nurodyto transkripcijos
faktoriaus pozicinę svorių matricą.

\subsubsection*{Duomenų kokybės vertinimas}
Duomenų kokybės vertinimo skiltyje pateikiama įkeltų ChIP sekoskaitos mėginių
(\ref{table:mouse_samples}) lentelė, kurios elementus naudotojas gali
pasirinkti bei vykdyti duomenų kokybės vertinimo etapus tik pasirinktiems
mėginiams. Atliekant duomenų kokybės vertinimą galima pasirinkti daugiau negu
vieną mėginį - visiems pasirinktiems mėginiams sugeneruojamas bendras grafikas
arba atskira kiekvieno mėginio vizualizacija.

\subsubsection*{Biologinės analizės}
Biologinių analizių atlikimo lange naudotojas gali pasirinkti, su kokiais
įkeltais mėginiais nori atlikti transkripcijos faktorių apibūdinančios pozicinės
svorių matricos atitikimų skaičiavimą. Atliekant GO analizę bei \emph{de novo}
motyvų paiešką aplikacija vienu metu gali analizuoti tik vieną mėginį.
\emph{De novo} motyvų paieškos įrankio - HOMER - pritaikymui, panaudota bazinė R
funkcija \emph{system()}, kuri leidžia R programose įterpti komandas, kurios
nepriklauso R bazinių ir įvairių bibliotekų funkcijų rinkiniui, tačiau yra
naudojamos komandinėje eilutėje (pavyzdžiui, taikant bioinformatinius komandinės
eilutės įrankius). Gauti identifikuoti \emph{de novo} motyvai pateikti
lentelės pavidalu, kurią naudotojas gali atsisiųsti ir naudoti. \emph{De novo}
motyvų paieška yra vienas iš ilgiau trunkančių šiame darbe aprašytų etapų, todėl
galutinių rezultatų pateikimas naudotojui, pateikusiam ir pasirinkusiam
mėginius, gali užtrukti dėl komandinės eilutės įrankio HOMER atliekamų motyvų
paieškos skaičiavimų.  

\subsubsection*{Taikinių spėjimas}
Tuo atveju, jei norima vykdyti transkripcijos faktorių taikinių spėjimą,
naudotojas turi specifikuoti organizmą, kurio genome bus spėjamas transkripcijos
faktorių prisijungimas. Taip pat naudotojas gali nurodyti minimalų pozicinės
svorių matricos bei sekų atitikimo procentą (\emph{min.score}), kuris
pritaikomas skaičiuojant PWM matricos atitikimus sekose. Atlikus šį pasirinkimą
pasirinktiems mėginiams (\ref{table:mouse_samples}) pritaikomas transkripcijos
faktorių taikinių spėjimo metodas. Šio etapo metu sugeneruojami tarpiniai
rezultatų failai: anotuotų pikų genų sekų rinkiniai, \emph{Blastp} rezultatai,
pozicinės svorių matricos atitikimų sekose lentelės bei pikų, kuriems priskirtų
genų atitikimai surasti specifikuotame genome, rinkiniai. Šiuos tarpinius
rezultatus galima per\-žiū\-rė\-ti \emph{Shiny} aplikacijoje bei juos
atsisiųsti.

\subsubsection*{Aplikacijos pritaikymas skirtingiems organizmams}
R \emph{Shiny} aplikacija gali būti panaudota, analizuojant mėginius, išgautus
iš skirtingų or\-ga\-niz\-mų, tačiau šis sąrašas yra ribotas. Realizuojant
aplikaciją taikytos parašytos funkcijos, naudojančios R programavimo kalbos
\emph{BSgenome}, \emph{Txdb} bei \emph{org.x.eg.db} genominių anotacijų
bibliotekas. Dabartinėje R versijoje (4.2.3) yra prieinamos 33 skirtingų
organizmų \emph{BSgenome} bei 12 \emph{Txdb} ir \emph{org.x.eg.db} bibliotekos.
Sukurtoje aplikacijoje galima analizuoti 6 skirtingų organizmų ChIP sekoskaitos
duomenis, kurie anotuojami bei analizuojami, naudojant naujausias organizmų
genomų anotacijų versijas. Šių organizmų genominių anotacijų bibliotekos,
informacija apie kiekvieno organizmo chromosomų rinkinį bei chromosomų ilgius
(informacija apie chromosomas išgauta iš NCBI \cite{NCBI} duomenų bazės)
pateiktos JSON formato pavidalu. Šis formatas sukurtoje
R aplikacijoje apdorotas, pritaikius \emph{rjson} \cite{RJSON} bibliotekos
funkciją \emph{fromJSON()}.

\newpage

%%%%%%%%%%%%%%%%%%%%%%
% MĖGINIŲ APRAŠYMAS
%%%%%%%%%%%%%%%%%%%%%%

\subsection{Analizuotų mėginių charakteristika}
Metodo patikimumui ir tikslumui įvertinti naudoti naminės pelės ir žmogaus
ChIP se\-kos\-kai\-tos duomenys, gauti iš nuolat atnaujinamo bei papildomo
ChIP-Atlas \cite{CHIPATLAS} serverio, saugančio ChIP bei kitų sekoskaitų
epigenetinių duomenų rinkinius bei leidžiančio vizualizuoti praturtintų sekų -
pikų - regionus \cite{CHIPATLAS2}.

Metodo testavimui naudoti 4 skirtingi \emph{Mus musculus}
bei 4 skirtingi \emph{Homo sapiens} mėginiai, išgauti iš
skirtingų tipų ląstelių: pliuripotentinių kamieninių ląstelių, gebančių
diferencijuoti į visų tipų ląsteles, bei nervinių ląstelių.

Iš ChIP-Atlas duomenų bazės atsisiųsti tik statistiškai patikimi ChIP
sekoskaitos duomenys, atrinkti įvertinus statistinio \emph{q} įverčio reikšmes.
Įvertinus genominių regionų \emph{q} vertes gautas mažesnis duomenų rinkinys,
kuris gali būti apdorojamas, nesukeliant programos veikimo trikdžių bei
nesumažinant rezultatų generavimo spartos. Statistiškai patikimi duomenys
atsisiųsti BED formato pavidalu. Mėginiai gali būti atsisiųsti iš
\emph{OneDrive} \href{https://vult-my.sharepoint.com/:f:/g/personal/daniele_stasiunaite_mif_stud_vu_lt/EtEGQ8POkapLhPv6eHvl48cB-jmes81M0JPW8PVWTz2QgA?e=wjSSKJ}{\emph{\textbf{saugyklos}}}.

ChIP sekoskaitos mėginiai, išgauti iš naminės pelės ląstelių, aprašyti pirmoje
lentelėje (\hyperref[table:mouse_samples]{Lentelė 1}).

\begin{table}[htb]
    \newcolumntype{M}[1]{>{\centering\arraybackslash}m{#1}}
    \small
    \captionsetup{justification=centering}
    \caption{\small\textbf{Naminės pelės mėginių
              charakteristikos.}\\ \emph{mESC} - pelių embrioninės kamieninės
              ląstelės; \emph{endoderminės ląstelės} - ląstelės, iš
              kurių susiformuoja organų epitelinis audinys;
              \emph{DKI} - pelių kamienas, kuriam būdingas dvigubas pasirinktų
              genų sekų įterpimas arba specifinių genų sekų pakeitimas kitų genų
              sekomis; \emph{progenitorinės ląstelės} - specifiškos ląstelės,
              gebančios diferencijuoti į tam tikro tipo ląsteles;
              \emph{C57BL/6 x DBA} - sukryžminti dažniausiai naudojami pelių
              kamienai.}
    \begin{tabular}{|c|c|c|c|c|}
        \hline
        \textbf{Ląstelių tipas} & \textbf{\thead{Kamienas}} &
        \textbf{\thead{Poveikis}} & \textbf{Antikūnai} &
        \textbf{\thead{ChIP-Atlas ID}} \\
        \hline
        \thead{Nervinės iš mESC} & - & Laukinis tipas (\emph{wt}) &
        \thead{CTCF} &
        \thead{\href{https://chip-atlas.org/view?id=SRX13476140}{SRX13476140}}\\ 
        \hline
        \thead{Endoderminės iš mESC} & DKI &
        \thead{Išgavimas po 3 dienų; r2} &
        \thead{anti-Foxa2} &
        \thead{\href{https://chip-atlas.org/view?id=SRX4298470}{SRX4298470}}\\
        \hline
        \thead{Endoderminės iš mESC} & DKI &
        \thead{Išgavimas po 5 dienų;\\endoderminės ląstelės} &
        \thead{anti-Gata4} &
        \thead{\href{https://chip-atlas.org/view?id=SRX4298474}{SRX4298474}}\\ 
        \hline
        \thead{Nervinės progenitorinės} & C57BL/6 x DBA &
        \thead{Ląstelių išgavimas\\po disociacijos} & \thead{anti-Sox2} &
        \thead{\href{https://chip-atlas.org/view?id=SRX2749159}{SRX2749159}}\\
        \hline
    \end{tabular}
    \label{table:mouse_samples}
\end{table}

ChIP sekoskaitos mėginiai, išgauti iš žmogaus ląstelių, aprašyti antroje
lentelėje (\hyperref[table:human_samples]{Lentelė 2}).

\begin{table}[htb]
    \newcolumntype{M}[1]{>{\centering\arraybackslash}m{#1}}
    \small
    \captionsetup{justification=centering}
    \caption{\small\textbf{Žmogaus mėginių charakteristikos.}\\
            \emph{hESC} - žmogaus embrioninės kamieninės ląstelės;
            \emph{KCl (+/-)} - ląstelių stimuliavimas arba nestimuliavimas
            kalio chlorido tirpalu; \emph{DE} - galutinė endoderma.}
    \begin{tabular}{|c|c|c|c|}
    \hline
    \textbf{Ląstelių tipas} &
    \textbf{\thead{Poveikis}} & \textbf{Antikūnai} &
    \textbf{\thead{ChIP-Atlas ID}} \\
    \hline
    \thead{Nervinės iš hESC} & \thead{Laukinis tipas (\emph{wt}); KCl-} &
    \thead{CTCF} &
    \thead{\href{https://chip-atlas.org/view?id=SRX4417526}{SRX4417526}}\\ 
    \hline
    \thead{Endoderminės iš hESC} & \thead{DE} &
    \thead{FOXA2} &
    \thead{\href{https://chip-atlas.org/view?id=SRX11080722}{SRX11080722}}\\ 
    \hline
    \thead{Endoderminės iš hESC} & Replika 1 & \thead{GATA4} &
    \thead{\href{https://chip-atlas.org/view?id=SRX701989}{SRX701989}}\\ 
    \hline
    \thead{Nervinės progenitorinės} & Laukinis tipas (\emph{wt}) &
    \thead{anti-SOX2} &
    \thead{\href{https://chip-atlas.org/view?id=SRX5716451}{SRX5716451}}\\ 
    \hline
    \end{tabular}
    \label{table:human_samples}
\end{table}

\newpage

%%%%%%%%%%%%%%%%%%%%%%%%%%%%%%%
% REZULTATAI IR JŲ APTARIMAS
%%%%%%%%%%%%%%%%%%%%%%%%%%%%%%%

\section{Rezultatai ir jų aptarimas}
Darbo metu realizuota \emph{Shiny} aplikacija, kurioje implementuotas sukurtas
transkripcijos faktoriaus taikinių spėjimo metodas, duomenų kokybės vertinimas
bei biologinių analizių atlikimas.

\subsection{\emph{Shiny} aplikacijos apžvalga}
Sukurta \emph{Shiny} aplikacija yra suskirstyta į 4 atskiras skiltis, kurias
pasirinkus galima atlikti skirtingus darbe aprašytus duomenų kokybės
vertinimo, biologinių analizių atlikimo bei traskripcijos faktorių taikinių
spėjimo etapus.

\subsubsection*{Duomenų įkėlimas}
Pradinis sukurtos \emph{Shiny} aplikacijos langas pavaizduotas šeštame
paveiksle (\hyperref[fig:image6]{6 pav.}).
Atskiri pradinio lango komponentai pažymėti tamsiai mėlyname fone pavaizduotais
baltos spalvos skaičiais. Šiuos skaičius atitinkančių komponentų aprašymai
pateikti po paveikslu.

\begin{figure}[ht]
    \begin{center}
        \captionsetup{justification=centering}
        \includegraphics[width=1\linewidth]{../Figures/Primary_panel.png}
        \vspace{-1.5\baselineskip}
        \caption{\small\textbf{Pradinio \emph{Shiny} lango iliustracija.}\\
            \emph{1} - mygtukų juosta, leidžianti atlikti parašytą etapą;
            \emph{2, 3, 4, 5} - įvesties laukai, kuriuose įkeliami duomenys,
                nurodomas organizmo bei transkripcijos faktoriaus pavadinimai;
            \emph{6} - pavyzdinių duomenų saugyklą atveriantis mygtukas;
            \emph{7} - automatiškai generuojama įvestus duomenis apibendrinanti
                skiltis;
            \emph{8} - įkeltus ChIP sekoskaitos mėginius aprašanti lentelė.}
        \label{fig:image6}
    \end{center}
\end{figure}

\newpage

Pradinį langą sudarančių komponentų apibūdinimai:
\begin{itemize}
    \item \textbf{\large{1}}: juosta, kurioje galima pasirinkti ChIP
        sekoskaitos duomenų apdorojimo etapus, paspaudžiant ant vieno iš
        mygtukų. Paspaudus atitinkamą mygtuką jis išryškinamas tamsiai raudona
        spalva.
    \item \textbf{\large{2}}: failo įvestis. Aplikacijoje galima įkelti daugiau
        nei vieną mėginį. Pavyzdyje nurodyta, kad įkelti 4 ChIP sekoskaitos
        mėginiai. Tai yra privalomas įvesties laukas. Tuo atveju, jei
        įkeltuose failuose nėra tabuliacijos simboliais atskirtų stulpelių,
        kuriuose pateiktos chromosomų santrumpos bei pikų regionai, nei vienas
        duomenų kokybės vertinimo, biologinių analizių atlikimo ar sukurto
        metodo etapas nėra įvykdomas.
    \item \textbf{\large{3}}: išskleidžiamas organizmų sąrašas, iš kurio
        galima pasirinkti vieno organizmo genomą. Priklausomai nuo to, koks
        organizmas pasirenkamas, R programa naudoja atitinkamo organizmo
        genomines anotacijas bei jų objektus. Pavyzdyje pasirinktas
        \emph{Mus musculus} organizmas.
    \item \textbf{\large{4}}: failo įvestis, kur turi būti įkelta viena
        transkripcijos faktoriaus pozicinė svorių matrica. Pavyzdyje įkelta
        Tbx5 transkripcijos faktoriaus pozicinė svorių matrica.
    \item \textbf{\large{5}}: teksto įvesties laukas, kuriame nurodomas
        įkelto transkripcijos faktoriaus pavadinimas.
    \item \textbf{\large{6}}: mygtukas, kurį paspaudus atveriamas pavyzdinių
        duomenų \emph{OneDrive} aplankas, iš kurio galima atsisiųsti darbo
        metu analizuotus mėginius.
    \item \textbf{\large{7}}: priklausomai nuo to, kokia informacija buvo
        įvesta, šiame lange pateikiama apibendrinta informacija bei
        pavaizduojamas įkeltą transkripcijos faktoriaus pozicinę svorių
        matricą atitinkantis sekos logotipas.
    \item \textbf{\large{8}}: lentelė, kurioje pateikta informacija apie įkeltus
        mėginius: originalūs įkeltų failų pavadinimai, sugeneruoti pavadinimai,
        kurie naudojami grafikuose. Pavadinimo „mm“ dalis kinta priklausomai
        nuo to, koks organizmas pasirinktas 3 numeriu pažymėtame įvesties
        lauke. Kadangi pasirinktas \emph{Mus musculus} organizmas, prie grafikų
        pavadinimo pridėtos dvi pirmosios organizmo lotyniško pavadinimo raidės.
        Trečioje lentelės skiltyje pateikiamas mėginio dydis, kuris
        specifikuotas mėginyje esančių pikų skaičiumi. Ši lentelė gali būti
        rikiuojama pagal visus lentelės stulpelius.
\end{itemize}

\newpage

\subsubsection*{Kokybės vertinimas}
Paspaudus pradiniame lange (\hyperref[fig:image6]{6 pav.}) pavaizduotą duomenų
kokybės vertinimo mygtuką pasirodo septintame paveiksle
(\hyperref[fig:image7]{7 pav.}) pavaizduotas vaizdas.

\begin{figure}[ht]
    \begin{center}
        \captionsetup{justification=centering}
        \includegraphics[width=1\linewidth]{../Figures/Data_quality_tab.png}
        \vspace{-1.5\baselineskip}
        \caption{\small\textbf{Duomenų kokybės vertinimo skilties
                                iliustracija.}\\
            \emph{1} - įkeltų mėginių pasirinkimo lentelė;
            \emph{2} - mygtukų juosta, iš kurios galima pasirinkti ir atlikti
                skirtngus ChIP sekoskaitos duomenų kokybės vertinimo etapus;
            \emph{3} - mygtukas, kuris leidžia atsiųsti sukurtą vizualizaciją.}
    \end{center}
    \label{fig:image7}
\end{figure}

Paveiksle išryškintų skilties elementų apibūdinimai:
\begin{itemize}
    \item \textbf{\large{1}}: įkeltų mėginių lentelė, kurios pirmame stulpelyje
        nurodytas sugeneruotas mėginio pavadinimas (kartais originalus mėginio
        pavadinimas gali būti ilgas, todėl originalaus pavadinimo naudojimas
        yra ne visada patogus ir aiškus). Atliekant duomenų kokybės vertinimą
        šioje lentelėje galima pasirinkti norimus mėginius, kaip pavaizduota
        dešimtame paveiksle esančiame pavyzdyje.
    \item \textbf{\large{2}}: įvairių grafikų, padedančių įvertinti įkeltų
        duomenų kokybę, juosta. Pateiktame pavyzdyje aktyvi „Mėginių
        panašumas“ skiltis - ji išryškinta tamsiai pilka spalva. Pasirinkus
        šią skiltį po grafikų pasirinkimo juosta pateikiamas kairėje pusėje
        esančioje lentelėje pasirinktų mėginių spalvų intensyvumo grafikas.
    \item \textbf{\large{3}}: mygtukas, kurį paspaudus inicijuojamas sugeneruoto
        paveikslėlio atsisiuntimas. Šis mygtukas būdingas daugeliui aplikacijoje
        sugeneruotų grafikų.
\end{itemize}

\newpage

\subsubsection*{Biologinės analizės}
Pasirinkus biologinių analizių atlikimo skiltį ankstesnis vaizdas pakeičiamas
aštuntame paveiksle (\hyperref[fig:image8]{8 pav.}) pavaizduotu vaizdu.
Pasirodžiusioje skiltyje vaizduojama mėginių lentelė bei trys biologinės
analizės, kurias galima atlikti su lentelėje pasirinktais mėginiais.

\begin{figure}[ht]
    \begin{center}
        \captionsetup{justification=centering}
        \includegraphics[width=1\linewidth]{../Figures/Biological_analysis_tab.png}
        \vspace{-1.5\baselineskip}
        \caption{\small\textbf{Biologinių analizių atlikimo skilties
                                iliustracija.}\\
            \emph{1} - mygtukų juosta, kurioje skilčiai „GO analizė“ būdinga 3
                vaikinių mygtukų elementų juosta;
            \emph{2} - subontologijos rezultato lentelė, kuri gali būti
                rikiuojama pagal visus lentelės stulpelius;
            \emph{3} - mygtukas, išskleidžiantis konkrečiam GO rezultatui
                priskirtų genų sąrašą lentelės pavidalu.}
        \label{fig:image8}
    \end{center}
\end{figure}

Paveiksle pavaizduotų elementų apibūdinimai:
\begin{itemize}
    \item \textbf{\large{1}}: pasirinkus skiltį „GO analizė“ pateikiamos
        trys vaikinės skiltys. Pasirinkus atitinkamą skiltį gali būti
        sugeneruotos GO analizės kiekvienos subontologijos (biologinių procesų,
        molekulinių funkcijų ir ląstelės komponentų) lentelės, kryptiniai
        acikliniai grafai bei medžio struktūros grafikai.
    \item \textbf{\large{2}}: sugeneruota GO analizės lentelė. Lentelės
        sugeneruojamos kiekvienai subontologijai atskirai.
    \item \textbf{\large{3}}: kiekvienas lentelės įrašas gali būti išskleistas,
        pasirinkus penktame lentelės stulpelyje esantį mygtuką „Peržiūrėti genų
        sąrašą“. Paspaudus šį mygtuką pateikiama anotuotų pikų genų, kuriems
        būdinga antroje lentelės skiltyje įrašyta funkcija ar priklausymas
        tam tikram ląstelės komponentui. Dar kartą paspaudus mygtuką genų
        lentelė suskleidžiama.
\end{itemize}

\newpage

Atliekant GO analizę bei \emph{de novo} motyvų paiešką galima analizuoti tik
vieną mėginį. Pasirinkus keletą mėginių išvedamas stilizuotas klaidos
pranešimas (\hyperref[fig:image9]{9 pav.}). Pasirodžius šiam pranešimui
analizės nėra vykdomos.

\begin{figure}[ht]
    \begin{center}
        \captionsetup{justification=centering}
        \includegraphics[width=1\linewidth]{../Figures/Too_many_samples.png}
        \vspace{-1.5\baselineskip}
        \caption{\small\textbf{Stilizuotas klaidos pranešimas,
                                pasirodantis GO analizės vykdymo ir
                                \emph{de novo} motyvų paieškos metu pasirinkus
                                daugiau nei vieną mėginį}}
        \label{fig:image9}
    \end{center}
\end{figure}

\newpage

\subsubsection*{Transkripcijos faktorių taikinių spėjimas}
\colorbox{red}{PILDOMA...}

\newpage

\subsection{Analizuotų mėginių rezultatų apžvalga}
Sukurta \emph{Shiny} aplikacija buvo panaudota realių (\hyperref[table:mouse_samples]{Lentelė 1}) ChIP sekoskitos duomenų analizei.

\subsection*{Duomenų kokybės vertinimas}
Duomenų kokybės vertinimo etape buvo analizuoti visi mėginiai, kuriems buvo
būdingas skirtingas pikų skaičius.
% Taip pat itin išsiskiriantis atrankumas chromosomų atžvilgiu
% nebuvo nustatytas, todėl vizualizavus pikų pasiskirstymą chromosomose galima
% teigti, kad duomenų problematiškumas nepastebimas.

\subsubsection*{Genominė distribucija}
Įvertinus bendrą pikų skaičių chromosomose bei nustačius mėginių panašumą
atliktas genominės mėginių distribucijos tyrimas. Gautas rezultatas pateiktas
dešimtame paveiksle (\hyperref[fig:image10]{10 pav.}).

\begin{figure}[ht]
    \begin{center}
        \captionsetup{justification=centering}
        \includegraphics[width=0.9\linewidth]{../Figures/Genomic_distribution.png}
        \vspace{-1.5\baselineskip}
        \caption{\small\textbf{Mėginių genominės distribucijos
                                grafikas}}
        \label{fig:image10}
    \end{center}
\end{figure}

Remiantis gautu rezultatu galima pastebėti, kad mėginiai \emph{Sample1\_mm},
\emph{Sample2\_mm} ir \emph{Sample3\_mm} turėjo labai panašų genominių elementų
pasiskirstymo procentą, tačiau mėginyje \emph{Sample4\_mm}, turinčiam mažiausią,
724, pikų skaičių, nustatytas itin didelis promotorių sekų, kurių ilgis mažesnis
nei 1 kilobazės, procentas, kuris viršijo 62\%. Taip pat pastarajame mėginyje
nustatyta mažesnė nei \emph{Sample1\_mm}, \emph{Sample2\_mm} ir
\emph{Sample3\_mm} mėginiuose esanti kitų intronų (rausva spalva) bei
tarpgeninių sričių (gelsva spalva) procentinė dalis. Šie genominių elementų
procentinių dalių skirtumai gali būti susiję su tuo, kad visi mėginiai buvo
išgauti iš nevienodų naminės pelės kamienų ląstelių, kurios buvo veiktos
skirtingais poveikiais bei transkripcijos faktoriais.

\newpage

\subsubsection*{Atstumas iki TSS}
Patikrinus mėginių genominių elementų pasiskirtymą ir pastebėjus
\emph{Sample4\_mm} mėginio išsiskyrimą sukurtas grafikas, vaizduojantis mėginių
atstumo iki artimiausių genų transkripcijos pradžios sričių procentinę dalį
5' \(\rightarrow\) 3' (angl. \emph{downstream}) ir 3' \(\rightarrow\) 5'
(angl. \emph{upstream}) DNR galų kryptimis. Gautas grafikas pateiktas
vienuoliktame paveiksle (\hyperref[fig:image1]{11 pav.}).

\begin{figure}[ht]
    \begin{center}
        \captionsetup{justification=centering}
        \includegraphics[width=0.9\linewidth]{../Figures/Distance_to_TSS.png}
        \vspace{-1.5\baselineskip}
        \caption{\small\textbf{Mėginių atstumai iki TSS
                                (5' \(\rightarrow\) 3' ir
                                3' \(\rightarrow\) 5' kryptimis)}}
        \label{fig:image11}
    \end{center}
\end{figure}

Iš pateikto grafiko labiausiai išsiskyrė genominių elementų procentinės
dalies nustatymo metu pastebėtas \emph{Sample4\_mm} mėginys. Pateiktame grafike
matoma, kad šiame mėginyje buvo nustatyta daugiausiai pikų, kurių atstumas iki
artimiausios TSS srities neviršija 1 kilobazės. Programa nustatė, kad
mėginyje buvo 40\% 5' \(\rightarrow\) 3' ir \(\sim\)25\% 3' \(\rightarrow\) 5'
kryptimis išsidėsčiusių pikų. Kituose mėginiuose 0 - 1 kilobazių atstumas iki
artimiausio TSS regiono nustatytas mažiau nei 10\% pikų.

\emph{Sample1\_mm}, \emph{Sample2\_mm} ir \emph{Sample3\_mm} mėginiuose
buvo nustatytas didžiausias procentas pikų, kurie nutolę nuo artimiausio
TSS regiono per 10 - 100 kilobazių. Tokie pikai sudarė \(\sim\)24\% visų
\emph{downstream} ir \emph{upstream} kryptimis išsidėsčiusių pikų.

\newpage

\subsubsection*{Pikų profilis}
Iš kitų mėginių išsiskyrus \emph{Sample4\_mm} mėginiui palygintas jo ir kitų
mėginių pikų profilis. Gautas rezultatas pateiktas dvyliktame paveiksle
(\hyperref[fig:image12]{12 pav.}).

\begin{figure}[ht]
    \begin{center}
        \captionsetup{justification=centering}
        \includegraphics[width=1\linewidth]{../Figures/Peak_profiles.png}
        \vspace{-1.5\baselineskip}
        \caption{\small\textbf{Mėginių pikų profiliai}}
    \label{fig:image12}
    \end{center}
\end{figure}

Gautuose pikų profiliuose nebuvo pastebėtos duomenų anomalijos - išskirtys,
kurios negali būti paaiškintos. Profiliams būdinga viena viršūnė.
\emph{Sample2\_mm} ir \emph{Sample3\_mm} mėginiams būdingas didesnis dantytumas,
kuris pastebimas tolstant nuo TSS, tačiau toks dantytumas nereiškia, kad
duomenys yra nekokybiški.

Mėginiui \emph{Sample4\_mm}, kuris akstesniuose
duomenų kokybės vertinimo etapuose išsiskyrė itin stipriai, pikų profilio
vaizdavimo etape buvo būdingas normalusis skirstinys, primenantis varpo formą,
kaip pastebima šešiolikto paveikslo paskutiniame grafike.

\newpage

\subsection*{Biologinės analizės}
Įvertinus naminės pelės mėginių kokybę ir nepastebėjus duomenų problematiškumo
vykdytos biologinės analizės.

\subsubsection*{GO analizė}
Naminės pelės mėginiams (\hyperref[table:mouse_samples]{Lentelė 1}) atlikus genų
ontologijos analizę buvo gautos rezultatų lentelės, iš kurių atrinktos funkcijos
bei ląsteliniai komponentai, turintys didžiausią priskirtų genų skaičių.
Trečioje lentelėje (\hyperref[table:table:go_results]{Lentelė 3}) pateiktos
lentelių ištraukos.

\begin{table}[htb]
    \newcolumntype{M}[1]{>{\centering\arraybackslash}m{#1}}
    \small
    \captionsetup{justification=centering}
    \caption{\small\textbf{GO analizės rezultatų ištraukos}}
    \begin{tabular}{|c|c|c|c|}
        \hline
            \thead{\textbf{Mėginys}} &
            \textbf{\thead{Biologinis procesas;\\genų santykis}} &
            \textbf{\thead{Molekulinė funkcija;\\genų santykis}} &
            \textbf{\thead{Ląstelės komponentas;\\genų santykis}}\\
        \hline
            \emph{Sample1\_mm} &
            \thead{Aksonų formavimasis\\327/9917} &
            \thead{Jungimasis prie fosfolipidų\\293/9857} &
            \thead{Postsinapsinė specializacija\\317/9953}\\
        \hline
            \emph{Sample2\_mm} &
            \thead{Aksonų formavimasis\\115/2492} &
            \thead{GTPazės reguliavimas\\87/2481} &
            \thead{Asimetrinė sinapsė\\104/2502}\\
        \hline
            \emph{Sample3\_mm} &
            \thead{Epitelinių vamzdelių morfogenezė\\143/3048} &
            \thead{Jungimasis prie aktinų\\99/3006} &
            \thead{Postsinapsinė specializacija\\140/3045}\\
        \hline
            \emph{Sample4\_mm} &
            \thead{Autofagija\\31/601} &
            \thead{-} &
            \thead{Asimetrinė sinapsė\\24/594}\\
        \hline
    \end{tabular}
    \label{table:go_results}
\end{table}

Dviejuose mėginiuose - \emph{Sample1\_mm} ir \emph{Sample2\_mm} - daugiausiai
anotuotų pikų genų buvo priskirta nervinių darinių, aksonų, formavimosi
biologinei funkcijai. Pusėje mėginių daugiausiai genų kodavo baltymus,
atliekančius jungimosi prie fosfolipidų dvisluoksnio molekulinę funkciją.
Kadangi \emph{Sample4\_mm} turėjo mažą pikų skaičių  (724 pikai), šiame mėginyje
GO analizės metu nebuvo gautas molekulinių funkcijos rezultatas. Šis rezultatas
nėra neįprastas, kadangi šiame mėginyje buvo naudotas anti-Sox2 antikūnas, o
Sox2 yra žinomas nervinių ląstelių markerinis žymuo \cite{ARTICLE21}.

Remiantis lentelės (\hyperref[table:go_results]{Lentelė 3}) ketvirto stulpelio
duomenimis pastebėta, kad tarp anotuotų pikų buvo daug genų, kurių koduojami
baltymai įeina į nervinių darinių komponentų sudėtį.

Gauti genų ontologijos rezultatai vizualizuoti acikliniais kryptiniais grafais
bei medžio struk\-tū\-ro\-mis. Pastarasis grafiko tipas pavaizduotas 
\emph{Sample2\_mm} mėginio biologinių procesų subontologijai tryliktame
paveiksle (\hyperref[fig:image13]{13 pav.}).

\begin{figure}[H]
    \begin{center}
        \includegraphics[width=1\linewidth]{../Figures/Sample2_treeplot.png}
        \vspace{-1.5\baselineskip}
        \caption{\small\textbf{\emph{Sample2\_mm} mėginio biologinių
                                procesų subontologijos medžio struktūra}}
        \label{fig:image13}
    \end{center}
\end{figure}

Pateiktoje vizualizacijoje biologiniai procesai suskaidyti į 5 klasterius:
raumens diferenciacijos, aksonų formavimosi, ląstelių migracijos amebos
judėjimo principu, \emph{Wnt} signalų perdavimo ir sinapsės jungčių formavimosi.
Grafike pažymėtas aksonų formavimosi biologinis procesas, kuriam priskirtas
didžiausias genų skaičius. Atsižvelgus į medžio lapuose pateiktų apskritimų
dydį galima įvertinti kitus biologinius procesus, kuriems būdingas mažiausias
arba didžiausias priskirtų genų skaičius. Taip pat remiantis gautu grafiku
galima pastebėti, kad daugiausiai genų nustatyta raumens diferenciacijos
klasteryje. Mažiausiai genų priskirta ląstelių migracijos, naudojant
specialias išaugas, pseudopodijas, biologiniui procesui.

\subsubsection*{Motyvų paieška \emph{de novo}}
Pritaikius HOMER komandinės eilutės įrankį \emph{de novo} motyvų mėginiuose
paieškai gautų \emph{Sample4\_mm} mėginio rezultatų ištrauka pateikta
keturioliktame paveiksle (\hyperref[fig:image14]{14 pav.}).

\begin{figure}[H]
    \begin{center}
        \includegraphics[width=1\linewidth]{../Figures/Homer_result.png}
        \vspace{-1.5\baselineskip}
        \caption{\small\textbf{\emph{Sample4\_mm} mėginio
                                \emph{de novo} motyvų paieškos rezultato
                                ištrauka}}
        \label{fig:image14}
    \end{center}
\end{figure}

Pateiktame paveiksle rezultatai išrikiuoti pagal sekų, kuriose identifikuotas
pirmame lentelės stulpelyje nurodytas motyvas, skaičių mažėjančia tvarka.
Lentelės ištraukoje matoma, kad \emph{Sample4\_mm} mėginio pikams priskirtų
genų sekose daugiausiai identifikuota KLF14 transkripcijos faktoriaus taikinių -
iš viso 42.60\% genų sekų, kurioms būdingas motyvas.

\newpage

\subsection*{Transkripcijos faktorių taikinių spėjimas}

\subsection*{PWM matricos atitikimai}
Realizavus transkripcijos faktorių taikinių spėjimo metodą pradinių
(\emph{Mus musculus}) mėginių sekoms bei gautoms sekoms iš \emph{Homo sapiens}
genomo pritaikyta transkripcijos faktoriaus pozicinė svorių matrica.
Penkioliktame paveiksle (\hyperref[fig:image15]{15 pav.}) pavaizduotose
stulpelinėse diagramose nurodyta, kokią procentinę dalį sudaro pradinių mėginių
anouotų pikų - genų - sekose identifikuoti pozicinę svorių matricą atitinkantys
fragmentai su nustatytų atitikimų skaičiumi \emph{Blastp} paieškos metu gautose
\emph{Homo sapiens} sekose.

\begin{figure}[H]
    \begin{center}
        \includegraphics[width=1\linewidth]{../Figures/Multiple_min_scores.png}
        \vspace{-1.5\baselineskip}
        \caption{\small\textbf{PWM atitikimų procentinių dalių
                                stulpelinės diagramos}}
        \label{fig:image15}
    \end{center}
\end{figure}

Remiantis gautu rezultatu galima pastebėti, kad visuose tirtuose
\emph{Mus musculus} mėginiuose buvo nustatyti genai, kuriems buvo surasti didelį
užklausos sekų padengimą bei sekų identiškumą taikinio organizme turintys
\emph{Homo sapiens} genai. Ieškant transkripcijos faktoriaus pozicinę svorių
matricą atitinkančių sekų fragmentų abiejų organizmų genuose buvo nustatyta,
kad užklausos sekose (\emph{Mus musculus}) ir jas atitinkančiose
(\emph{Homo sapiens}) sekose buvo nustatytas didelis pozicinės svorių matricos
atitikimų skaičius: \(\sim\)85\%, kai \emph{min.score = 70\%}, \(\sim\)81\%,
kai \emph{min.score = 80\%} ir \(\sim\)56\%, kai \emph{min.score = 90\%}.

Didinant pozicinės svorių matricos atitikimų skaičiavimo parametrą - minimalų
transkripcijos faktoriaus motyvo ir sekos fragmento atitikimo procentą - 
atitikimų skaičius užklausos bei taikinio sekose sumažėjo, tačiau išskirtys
nepastebėtos.

Suskaičiavus, kiek pozicinės svorių matricos atitikimų yra užklausos bei
taikinio sekose, pasitaikė atvejų, kai pozicinės svorių matricos atitikimų
užklausos sekose nustatyta daugiau nei taikinio sekose. Šių atvejų procentinė
dalis kiekviename mėginyje pavaizduota šešioliktame paveiksle
(\hyperref[fig:image16]{16 pav.}).

\begin{figure}[ht]
    \begin{center}
        \captionsetup{justification=centering}
        \includegraphics[width=1\linewidth]{../Figures/Less_PWM_hits_in_control.png}
        \vspace{-1.5\baselineskip}
        \caption{\small\textbf{Genų, turinčių mažesnį PWM atitikimų 
                                skaičių taikinio sekose, procentinės dalies
                                stulpelinė diagrama}}
        \label{fig:image16}
    \end{center}
\end{figure}

Mėginiui \emph{Sample1}, išgautame iš naminės pelės audinių, buvo būdingas
didžiausias anotuotų pikų skaičius. Šiems pikams priskirtiems genams suradus
atitinkamas sekas žmogaus genome dalis žmogaus genų turėjo
mažesnį pozicinę svorių matricą atitinkančių sekų fragmentų skaičių nei
naminės pelės genų sekos - tokių genų sekų nustatyta 18\%. Mažiausias pozicinės
svorių matricos atitikimų skaičius taikinio sekose (16.41\%) nustatytas
\emph{Sample4} mėginyje, kuriam būdingas mažiausias pikų bei jiems priskirtų
genų skaičius.

\newpage

\subsection*{Nustatytų genų sutapimai su kontroliniais mėginiais}
Anotavus kontrolinius žmogaus ChIP sekoskaitos pikų rinkinius
(\ref{table:human_samples}) nustatyta, kokia anotuotų pikų - genų - dalis
sutampa su filtruotu \emph{Blastp} paieškos metu gautu genų rinkiniu. Gautas
sutampančių pikų procentas pavaizduotas spalvų intensyvumo grafike
(\hyperref[fig:image17]{17 pav.}).

\begin{figure}[ht]
    \begin{center}
        \captionsetup{justification=centering}
        \includegraphics[width=0.7\linewidth]{../Figures/Genes_in_control.png}
        \vspace{-1.5\baselineskip}
        \caption{\small\textbf{Anotuotų kontrolinių mėginių pikų ir
                                \emph{Blastp} paieškos metu gautų sutampančių
                                genų spalvų intensyvumo grafikas}}
        \label{fig:image17}
    \end{center}
\end{figure}

Remiantis gautu grafiku galima pastebėti, kad didžiausias sutampančių genų
skaičius nus\-ta\-ty\-tas tarp \emph{Mus musculus} mėginio \emph{Sample2} ir
\emph{Homo sapiens} mėginio \emph{SRX11080722.05.bed}. Pas\-ta\-ra\-sis žmogaus
ChIP sekoskaitos mėginys turėjo didžiausią anotuotų pikų skaičių, todėl 
buvo nustatyta daugiausiai genų sutapimų su atitinkamomis naminės pelės genų
sekomis. Kituose kontroliniuose žmogaus ir naminės pelės mėginiuose sutampančių
genų procentinė dalis buvo mažesnė dėl mažėjančio kontrolinių mėginių pikų
skaičiaus.

\colorbox{green}{PASTABA: mėginių pavadinimai grafikuose yra koreguojami.}

%%%%%%%%%%%%
% IŠVADOS
%%%%%%%%%%%%

\section{Išvados}
Sukūrus transkripcijos faktorių taikinių spėjimo metodą bei realizavus jo
implementaciją R \emph{Shiny} aplinkoje gauti rezultatai buvo apibendrinti:

\begin{itemize}
    \item Sukurtas transkripcijos faktorių taikinių spėjimo metodas leidžia
        nuspėti, kokiuose genominiuose pasirinkto organizmo regionuose
        galimas transkripcijos faktoriaus prisijungimas, naudojant kito
        organizmo genominius duomenis;
    \item Priklausomai nuo organizmų genomų panašumo realizuotas metodas
        leidžia įvertinti transkripcijos faktoriaus pozicinės svorių matricos
        atitikimą su dideliu tikslumu;
    \item Realizuotas metodas gali būti panaudotas, spėjant pasirinkto
        transkripcijos faktoriaus taikinius skirtinguose organizmuose;
    \item Sukurta R \emph{Shiny} aplikacija, kurioje galima įvertinti ChIP
        sekoskaitos duomenų kokybę, atlikti biologines analizes bei pritaikyti
        transkripcijos faktorių taikinių spėjimo metodą grafinėje aplinkoje.
\end{itemize}

\newpage

%%%%%%%%%%%%%%%
% LITERATŪRA
%%%%%%%%%%%%%%%

\section{Literatūros sąrašas}

\bibliographystyle{plain}
\begin{thebibliography}{99}

\bibitem{ARTICLE18} Horton RH, Lucassen AM. Recent developments in
genetic/genomic medicine. Clin Sci (Lond). 2019 Mar 5;133(5):697-708.
doi: 10.1042/CS20180436. PMID: 30837331; PMCID: PMC6399103.

\bibitem{ARTICLE19} Tang J, Xu Z, Huang L, Luo H, Zhu X. Transcriptional
regulation in model organisms: recent progress and clinical implications.
Open Biol. 2019 Nov 29;9(11):190183. doi: 10.1098/rsob.190183.
Epub 2019 Nov 20. PMID: 31744421; PMCID: PMC6893401.

\bibitem{ARTICLE0} Lee TI, Young RA. Transcriptional regulation and its
misregulation in disease. Cell. 2013 Mar 14;152(6):1237-51.
doi: 10.1016/j.cell.2013.02.014. PMID: 23498934; PMCID: PMC3640494.

\bibitem{BIOTOOLS} Ison, J. et al. (2015). Tools and data services registry: a
community effort to document bioinformatics resources. Nucleic Acids Research.
doi: 10.1093/nar/gkv1116.

\bibitem{RNASEQ} Love MI, Anders S, Kim V, Huber W. RNA-Seqworkflow:
gene-level exploratory analysis and differential expression. F1000Res.
2015 Oct 14;4:1070. doi: 10.12688/f1000research.7035.1. PMID: 26674615;
PMCID: PMC4670015.

\bibitem{GTRD} GTRD: an integrated view of transcription regulation.
Kolmykov S, Yevshin I, Kulyashov M, Sharipov R, Kondrakhin Y, Makeev VJ,
Kulakovskiy IV, Kel A, Kolpakov F Nucleic Acids Res. 2021 Jan
8;49(D1):D104-D111.

\bibitem{CHIPATLAS} Zou, Z., Ohta, T., Miura, F., Oki, S. ChIP-Atlas 2021
update: a data-mining suite for exploring epigenomic landscapes by fully
integrating ChIP-seq, ATAC-seq and Bisulfite-seq data. Nucleic Acids Res.
50(W1), W175-W182, 2022, \newline
\url{http://dx.doi.org/10.1093/nar/gkac199}.

\bibitem{CHIPATLAS2} Zhaonan Zou, Tazro Ohta, Fumihito Miura, Shinya Oki,
ChIP-Atlas 2021 update: a data-mining suite for exploring epigenomic landscapes
by fully integrating ChIP-seq, ATAC-seq and Bisulfite-seq data, Nucleic Acids
Research, Volume 50, Issue W1, 5 July 2022, Pages W175–W182,
\url{https://doi.org/10.1093/nar/gkac199}.

\bibitem{ARTICLE1} Nakato R, Sakata T. Methods for ChIP-seq analysis: A
practical workflow and advanced applications. Methods. 2021 Mar;187:44-53.
doi: 10.1016/j.ymeth.2020.03.005. Epub 2020 Mar 30. PMID: 32240773.

\bibitem{ARTICLE2}
CD Genomics (2023). \emph{ChIP-Seq}.
Prieiga per \url{https://www.cd-genomics.com/chip-seq.html}
[žiūrėta 2023-05-21].

\bibitem{SONICATION}
GoldBio (2023). \emph{How to break up DNA for NGS library prep}.
Prieiga per \url{https://goldbio.com/articles/article/how-to-fragment-DNA-for-NGS}.
[žiūrėta 2023-05-21].

\bibitem{ARTICLE3} Shah, A. Chromatin immunoprecipitation sequencing (ChIP-Seq)
on the SOLiD™ system. Nat Methods 6, ii–iii (2009).
\url{https://doi.org/10.1038/nmeth.f.247}.

\bibitem{ARTICLE4} Liu, E.T., Pott, S. \& Huss, M. Q\&A: ChIP-seq technologies
and the study of gene regulation. BMC Biol 8, 56 (2010).
\url{https://doi.org/10.1186/1741-7007-8-56}.

\bibitem{ARTICLE5}
CD Genomics (2023). \emph{The Advantages and Workflow of ChIP-seq}.
Prieiga per \url{https://www.cd-genomics.com/the-advantages-and-workflow-of-chip-seq.html}.
[žiūrėta 2023-05-21].

\bibitem{ARTICLE6} Langmead, B., Trapnell, C., Pop, M. et al. Ultrafast and
memory-efficient alignment of short DNA sequences to the human genome. Genome
Biol 10, R25 (2009), \newline
\url{https://doi.org/10.1186/gb-2009-10-3-r25}.

\bibitem{ARTICLE7} Li H, Durbin R. Fast and accurate short read alignment with
Burrows-Wheeler transform. Bioinformatics. 2009 Jul 15;25(14):1754-60. doi: 10.
1093/bioinformatics/btp324. Epub 2009 May 18. PMID: 19451168; PMCID: PMC2705234.

\bibitem{ARTICLE9} Lee TI, Young RA. Transcriptional regulation and its
misregulation in disease. Cell. 2013 Mar 14;152(6):1237-51.
doi: 10.1016/j.cell.2013.02.014. PMID: 23498934; PMCID: PMC3640494.

\bibitem{ASP1}
Myriad Genetics (2023). \emph{Autoimmune Polyglandular Syndrome Type 1}.
Prieiga per \url{https://myriad.com/womens-health/diseases/autoimmune-polyglandular-syndrome-type-1/}.
[žiūrėta 2023-05-21].

\bibitem{ARTICLE10} Lambert SA, Jolma A, Campitelli LF, Das PK, Yin Y, Albu M,
Chen X, Taipale J, Hughes TR, Weirauch MT. The Human Transcription Factors.
Cell. 2018 Feb 8;172(4):650-665. doi: 10.1016/j.cell.2018.01.029.
Erratum in: Cell. 2018 Oct 4;175(2):598-599. PMID: 29425488.

\bibitem{ARTICLE11} Mundade R, Ozer HG, Wei H, Prabhu L, Lu T. Role of ChIP-seq
in the discovery of transcription factor binding sites, differential gene
regulation mechanism, epigenetic marks and beyond. Cell Cycle.
2014;13(18):2847-52. doi: 10.4161/15384101.2014.949201. PMID: 25486472;
PMCID: PMC4614920.

\bibitem{ARTICLE12} Zhang, Y., Liu, T., Meyer, C.A. et al. Model-based Analysis
of ChIP-Seq (MACS). Genome Biol 9, R137 (2008).
\url{https://doi.org/10.1186/gb-2008-9-9-r137}.

\bibitem{ARTICLE13}
Merck (2023). \emph{PEAK CALLING FOR ChIP-SEQ}
Prieiga per \url{https://epigenie.com/wp-content/uploads/2013/02/Peak-Calling-for-ChIP-Seq.pdf}.
[žiūrėta 2023-05-21].

\bibitem{ARTICLE14} Guo Y, Papachristoudis G, Altshuler RC, Gerber GK, Jaakkola
TS, Gifford DK, Mahony S. Discovering homotypic binding events at high spatial
resolution. Bioinformatics. 2010 Dec 15;26(24):3028-34.
doi: 10.1093/bioinformatics/btq590. Epub 2010 Oct 21. PMID: 20966006;
PMCID: PMC2995123.

\bibitem{ARTICLE15} Guo Y, Mahony S, Gifford DK. High resolution genome wide
binding event finding and motif discovery reveals transcription factor spatial
binding constraints. PLoS Comput Biol. 2012;8(8):e1002638. doi: 10.1371/journal.
pcbi.1002638. Epub 2012 Aug 9. PMID: 22912568; PMCID: PMC3415389.

\bibitem{ENSEMBLDB} Rainer J, Gatto L, Weichenberger CX (2019) ensembldb: an R
package to create and use Ensembl-based annotation resources. Bioinformatics.
doi:10.1093/bioinformatics/btz031.

\bibitem{ARTICLE16} Ji H, Jiang H, Ma W, Wong WH. Using CisGenome to analyze
ChIP-chip and ChIP-seq data. Curr Protoc Bioinformatics. 2011 Mar;
Chapter 2:Unit2.13. doi: 10.1002/0471250953.bi0213s33. PMID: 21400695;
PMCID: PMC3072298.

\bibitem{ARTICLE17} Enrique Blanco, Luciano Di Croce, Sergi Aranda. Comparative
ChIP-seq (Comp-ChIP-seq): a novel computational methodology for genome-wide
analysis. bioRxiv. 2019 January 29. doi: 10.1101/532622. \newline
\url{https://www.biorxiv.org/content/early/2019/03/26/532622}.

\bibitem{ARTICLE20} Zhang, Y., Liu, T., Meyer, C.A. et al.
Model-based Analysis of ChIP-Seq (MACS). Genome Biol 9, R137 (2008).
\url{https://doi.org/10.1186/gb-2008-9-9-r137}.

\bibitem{ARTICLE21} Mercurio S, Serra L, Nicolis SK. More than just Stem Cells:
Functional Roles of the Transcription Factor Sox2 in Differentiated Glia and
Neurons. Int J Mol Sci. 2019 Sep 13;20(18):4540. doi: 10.3390/ijms20184540.
PMID: 31540269; PMCID: PMC6769708.

\bibitem{R} R Core Team (2022). R: A language and environment for statistical
computing. R Foundation for Statistical Computing, Vienna, Austria. URL
\url{https://www.R-project.org/}.

\bibitem{GGPLOT2} H. Wickham. ggplot2: Elegant Graphics for Data Analysis.
Springer-Verlag New York, 2016.

\bibitem{SHINY} Chang W, Cheng J, Allaire J, Sievert C, Schloerke B, Xie Y,
Allen J, McPherson J, Dipert A, Borges B (2023). shiny: Web Application
Framework for R. R package version 1.7.4.9002, \url{https://shiny.rstudio.com/}.

\bibitem{CHIPSEEKER} Wang Q, Li M, Wu T, Zhan L, Li L, Chen M, Xie W, Xie Z,
Hu E, Xu S, Yu G (2022). “Exploring epigenomic datasets by ChIPseeker.”
Current Protocols, 2(10), e585. doi: 10.1002/cpz1.585.

\bibitem{GRANGES} Lawrence M, Huber W, Pag\`es H, Aboyoun P, Carlson M,
et al. (2013) Software for Computing and Annotating Genomic Ranges. PLoS Comput
Biol 9(8): e1003118. doi:10.1371/journal.pcbi.1003118.

\bibitem{TXDB_MM} Team BC, Maintainer BP (2019). 
TxDb.Mmusculus.UCSC.mm10.knownGene: Annotation package for TxDb object(s).
R package version 3.10.0.

\bibitem{GGSEQLOGO} Wagih O (2017). ggseqlogo: A 'ggplot2' Extension for
Drawing Publication-Ready Sequence Logos. R package version 0.1, \newline
\url{https://CRAN.R-project.org/package=ggseqlogo}.

\bibitem{BSGENOME} Pagès H (2023). BSgenome: Software infrastructure for
efficient representation of full genomes and their SNPs. R package version
1.66.3, \newline
\url{https://bioconductor.org/packages/BSgenome}.

\bibitem{BSMUSMUSCULUS} Team TBD (2021). BSgenome.Mmusculus.UCSC.mm10: Full
genome sequences for Mus musculus (UCSC version mm10, based on GRCm38.p6). R
package version 1.4.3.

\bibitem{BIOSTRINGS} Pagès H, Aboyoun P, Gentleman R, DebRoy S (2022).
Biostrings: Efficient manipulation of biological strings. R package version
2.66.0, \newline
\url{https://bioconductor.org/packages/Biostrings}.

\bibitem{CLUSTERPROFILER} T Wu, E Hu, S Xu, M Chen, P Guo, Z Dai, T Feng,
L Zhou, W Tang, L Zhan, X Fu, S Liu, X Bo, and G Yu. clusterProfiler 4.0:
A universal enrichment tool for interpreting omics data. The Innovation. 2021,
2(3):100141.

\bibitem{ENRICHPLOT} Yu G (2023). enrichplot: Visualization of Functional
Enrichment Result. R package version 1.18.4, \newline
\url{https://yulab-smu.top/biomedical-knowledge-mining-book}.

\bibitem{JSON} Pezoa, F. et al., 2016. Foundations of JSON schema. In
Proceedings of the 25th International Conference on World Wide Web. pp. 263–273.

\bibitem{RJSON} Couture-Beil A (2022). rjson: JSON for R. R package version
0.2.21, \newline
\url{https://CRAN.R-project.org/package=rjson}.

\bibitem{NCBI} National Center for Biotechnology Information (NCBI)[Internet].
Bethesda (MD): National Library of Medicine (US), National Center for
Biotechnology Information; [1988] – [cited 2023 May 19]. Available from:
\url{https://www.ncbi.nlm.nih.gov/}.

\bibitem{RBLAST} Hahsler M, Nagar A (2019). rBLAST: R Interface for the Basic
Local Alignment Search Tool. R package version 0.99.2, \newline
\url{https://github.com/mhahsler/rBLAST}.

\bibitem{BLAST} Csamacho C., Coulouris G., Avagyan V., Ma N., Papadopoulos J.,
Bealer K., Madden T.L. (2008) “BLAST+: architecture and applications.”
BMC Bioinformatics 10:421. PubMed.

\bibitem{HOMER} Heinz S, Benner C, Spann N, Bertolino E et al. Simple
Combinations of Lineage-Determining Transcription Factors Prime cis-Regulatory
Elements Required for Macrophage and B Cell Identities. Mol Cell 2010 May
28;38(4):576-589. PMID: 20513432.

\bibitem{HOCOMOCO} Ivan V. Kulakovskiy; Ilya E. Vorontsov; Ivan S. Yevshin;
Ruslan N. Sharipov; Alla D. Fedorova; Eugene I. Rumynskiy; Yulia A. Medvedeva;
Arturo Magana-Mora; Vladimir B. Bajic; Dmitry A. Papatsenko; Fedor A. Kolpakov;
Vsevolod J. Makeev. Nucl. Acids Res., Database issue, gkx1106
(11 November 2017), doi: 10.1093/nar/gkx1106.

\end{thebibliography}

\end{document}